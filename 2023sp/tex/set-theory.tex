% define new conditionals: \if<conditional>
\newif\ifbook  % whether or not to use book styles
\newif\ifdate  % whether or not to include the current date header
\newif\ifalgorithms  % whether or not to define the algorithm environment
\newif\ifdaggerfootnotes  % whether or not to have fnsymbol footnotes or numerical ones

% set the conditionals: \<conditional>true / \<conditional>false
\bookfalse
\datefalse
\algorithmsfalse
\daggerfootnotestrue

\ifbook
    \documentclass[letterpaper]{book}
    \usepackage{arydshln, chngcntr}
\else
    \documentclass[letterpaper]{article}
\fi

% math
\usepackage{amsmath, amsfonts, amssymb, amstext, amscd, amsthm, mathrsfs, mathtools, xfrac}
% fonts
\usepackage{bbm, CJKutf8, caption, dsfont, marvosym, stmaryrd}
% tables
\usepackage{booktabs, colortbl, makecell}
% colors
\usepackage{color, soul, xcolor}
% references
\usepackage{xr-hyper, hyperref, url}
% figures
\usepackage{graphicx, float, subcaption, tikz}
% headers and footers
\usepackage{fancyhdr, lastpage}
% miscellaneous
\usepackage{enumerate, ifthen, lipsum, listings, makeidx, parskip, ulem, verbatim, xargs}
\usepackage[nodayofweek]{datetime}

\usepackage[left=2cm,top=2cm,right=2cm,bottom=2cm,bindingoffset=0cm]{geometry}
\usepackage[group-separator={,},group-minimum-digits={3}]{siunitx}
\usepackage[shortlabels]{enumitem}
\setlist[enumerate]{topsep=0ex,itemsep=0ex,partopsep=1ex,parsep=1ex}
\setlist[itemize]{topsep=0ex,itemsep=0ex,partopsep=1ex,parsep=1ex}

\usepackage[math]{cellspace}
\cellspacetoplimit 1pt
\cellspacebottomlimit 1pt

\definecolor{gruvred}{HTML}{CC214D}
\definecolor{gruvorange}{HTML}{D65D0E}
\definecolor{gruvaqua}{HTML}{689D6A}
\definecolor{gruvpurple}{HTML}{B16286}
\definecolor{colorblack}{HTML}{252422}
\definecolor{colorgrey}{HTML}{f4efef}
\definecolor{colorblue}{HTML}{045275}
\definecolor{colorteal}{HTML}{089099}
\definecolor{colorgreen}{HTML}{7ccba2}
\definecolor{coloryellow}{HTML}{ffc61e}  % fcde9c % ffc61e  % b8860b
\definecolor{colororange}{HTML}{f0746e}
\definecolor{colorred}{HTML}{dc3977}
\definecolor{colorpurple}{HTML}{7c1d6f}

\hypersetup{
    colorlinks=true,
    linkcolor=gruvorange,
    citecolor=gruvaqua,
    urlcolor=gruvpurple
}

\allowdisplaybreaks
\newdateformat{verbosedate}{\ordinal{DAY} of \monthname[\THEMONTH], \THEYEAR}
\verbosedate

\pagestyle{fancy}
% \fancyfoot[C]{--~\thepage~--}
\fancyfoot[C]{\tiny \thepage\ / \pageref*{LastPage}}
\ifbook
    \fancypagestyle{plain}{%
        \fancyhead[L]{}
        \ifdate
            \fancyhead[R]{\textsc{\today}}
        \else
            \fancyhead[R]{}
        \fi
        \renewcommand{\headrulewidth}{0pt}
    }

    \let\cleardoublepage=\clearpage
\else
    \fancypagestyle{plain}{}
    \renewcommand{\headrulewidth}{0pt}
\fi

\ifdaggerfootnotes
    \renewcommand{\thefootnote}{\fnsymbol{footnote}}
\else
\fi

\delimitershortfall=-1pt
\normalem

\newlist{detail}{itemize}{2}
\setlist[detail]{label={\boldmath$\cdot$},topsep=0pt,leftmargin=*,noitemsep}

\ifalgorithms
    \newcounter{nalg}[chapter]
    \renewcommand{\thenalg}{\thechapter.\arabic{nalg}}
    \DeclareCaptionLabelFormat{algocaption}{\it Algorithm \thenalg}

    \lstnewenvironment{algorithm}[1][]
    {
        \refstepcounter{nalg}
        \captionsetup{labelformat=algocaption,labelsep=colon}
        \lstset{
            mathescape=true,
            frame=tB,
            numbers=left,
            numberstyle=\tiny,
            basicstyle=\scriptsize,
            keywordstyle=\color{black}\bfseries\em,
            keywords={,input, output, return, datatype, function, in, if, else, elif, for, foreach, while, not, begin, end, true, false, null, break, continue, let, and, or, }
            numbers=left,
            xleftmargin=.04\textwidth,
            #1
        }
    }
    {}
\else
\fi

\ifindentproofs
    % begin new proof environment
    \expandafter\let\expandafter\oldproof\csname\string\proof\endcsname
    \let\oldendproof\endproof

    \renewenvironment{proof}[1][\proofname]{%
        \ifindenttheorems
            \vspace{-\abovedisplayskip}
        \else
        \fi
        \oldproof[#1]\quote~\vspace{-\parskip}

    }{%
        %\endquote\oldendproof
        \endquote\vspace{-\parskip}\qed
    }
    % end new proof environment
\else
\fi

\ifindenttheorems
    \newtheorem{pretheorem}{Theorem}
    \newtheorem{prelemma}{Lemma}
    \newtheorem{preproposition}{Proposition}
    \newtheorem{precorollary}{Corollary}
    \newtheorem{preclaim}{Claim}
    \newtheorem{preconjecture}{Conjecture}
    \newtheorem{prejustification}{Justification}

    \newtheorem{preaxiom}{Axiom}
    \newtheorem{predefinition}{Definition}
    \newtheorem{prenotation}{Notation}
    \newtheorem{preexercise}{Exercise}
    \newtheorem{preexample}{Example}
    \newtheorem{precounterexample}{Counterexample}

    \newtheorem{preidea}{Idea}
    \newtheorem*{preremark}{Remark}
    \newtheorem*{prenote}{Note}

    % theorem
    \NewDocumentEnvironment{theorem}{O{} O{}}
        {\begin{pretheorem}[#1]~#2\quote\vspace{-0.75\parskip}}
        {\endquote\end{pretheorem}}
    % lemma
    \NewDocumentEnvironment{lemma}{O{} O{}}
        {\begin{prelemma}[#1]~#2\quote\vspace{-0.75\parskip}}
        {\endquote\end{prelemma}}
    % proposition
    \NewDocumentEnvironment{proposition}{O{} O{}}
        {\begin{preproposition}[#1]~#2\quote\vspace{-0.75\parskip}}
        {\endquote\end{preproposition}}
    % corollary
    \NewDocumentEnvironment{corollary}{O{} O{}}
        {\begin{precorollary}[#1]~#2\quote\vspace{-0.75\parskip}}
        {\endquote\end{precorollary}}
    % claim
    \NewDocumentEnvironment{claim}{O{} O{}}
        {\begin{preclaim}[#1]~#2\quote\vspace{-0.75\parskip}}
        {\endquote\end{preclaim}}
    % conjecture
    \NewDocumentEnvironment{conjecture}{O{} O{}}
        {\begin{preconjecture}[#1]~#2\quote\vspace{-0.75\parskip}}
        {\endquote\end{preconjecture}}
    % justification
    \NewDocumentEnvironment{justification}{O{} O{}}
        {\begin{prejustification}[#1]~#2\quote\vspace{-0.75\parskip}}
        {\endquote\end{prejustification}}

    % axiom
    \NewDocumentEnvironment{axiom}{O{} O{}}
        {\begin{preaxiom}[#1]~#2\quote\normalfont\vspace{-0.75\parskip}}
        {\endquote\end{preaxiom}}
    % definition
    \NewDocumentEnvironment{definition}{O{} O{}}
        {\begin{predefinition}[#1]~#2\quote\normalfont\vspace{-0.75\parskip}}
        {\endquote\end{predefinition}}
    % notation
    \NewDocumentEnvironment{notation}{O{} O{}}
        {\begin{prenotation}[#1]~#2\quote\normalfont\vspace{-0.75\parskip}}
        {\endquote\end{prenotation}}
    % exercise
    \NewDocumentEnvironment{exercise}{O{} O{}}
        {\begin{preexercise}[#1]~#2\quote\normalfont\vspace{-0.75\parskip}}
        {\endquote\end{preexercise}}
    % example
    \NewDocumentEnvironment{example}{O{} O{}}
        {\begin{preexample}[#1]~#2\quote\normalfont\vspace{-0.75\parskip}}
        {\endquote\end{preexample}}
    % counterexample
    \NewDocumentEnvironment{counterexample}{O{} O{}}
        {\begin{precounterexample}[#1]~#2\quote\normalfont\vspace{-0.75\parskip}}
        {\endquote\end{precounterexample}}

    % idea
    \NewDocumentEnvironment{idea}{O{} O{}}
        {\begin{preidea}[#1]~#2\normalfont}
        {\end{preidea}}
    % remark
    \NewDocumentEnvironment{remark}{O{} O{}}
        {\begin{preremark}[#1]~#2\normalfont}
        {\end{preremark}}
    % note
    \NewDocumentEnvironment{note}{O{} O{}}
        {\begin{prenote}[#1]~#2\normalfont}
        {\end{prenote}}
\else
    \theoremstyle{thm}% style for theorems
    \newtheorem{theorem}{Theorem}
    \newtheorem{lemma}{Lemma}
    \newtheorem{proposition}{Proposition}
    \newtheorem{corollary}{Corollary}
    \newtheorem{claim}{Claim}
    \newtheorem{conjecture}{Conjecture}
    \newtheorem{justification}{Justification}

    \theoremstyle{dfn}% style for definitions
    \newtheorem{axiom}{Axiom}
    \newtheorem{definition}{Definition}
    \newtheorem{notation}{Notation}
    \newtheorem{exercise}{Exercise}
    \newtheorem{example}{Example}
    \newtheorem{counterexample}{Counterexample}

    \theoremstyle{rmk}% style for remarks
    \newtheorem{idea}{Idea}
    \newtheorem*{remark}{Remark}
    \newtheorem*{note}{Note}
\fi

\newenvironment{case}[1][Case]
    {\textbf{#1:}\quote\vspace{-0.75\parskip}}
    {\endquote}

\def\lstlistingautorefname{Algorithm}
\def\itemautorefname{Section}
\renewcommand{\chapterautorefname}{Chapter}
\renewcommand{\sectionautorefname}{Section}
\newcommand{\pretheoremautorefname}{Theorem}
\newcommand{\preaxiomautorefname}{Axiom}
\newcommand{\prelemmaautorefname}{Lemma}
\newcommand{\prepropositionautorefname}{Proposition}
\newcommand{\precorollaryautorefname}{Corollary}
\newcommand{\preclaimautorefname}{Claim}
\newcommand{\preconjectureautorefname}{Conjecture}
\newcommand{\prejustificationautorefname}{Justification}
\newcommand{\predefinitionautorefname}{Definition}
\newcommand{\prenotationautorefname}{Notation}
\newcommand{\preexampleautorefname}{Example}
\newcommand{\precounterexampleautorefname}{Counterexample}
\newcommand{\preideaautorefname}{Idea}
\newcommand{\axiomautorefname}{Axiom}
\newcommand{\lemmaautorefname}{Lemma}
\newcommand{\propositionautorefname}{Proposition}
\newcommand{\corollaryautorefname}{Corollary}
\newcommand{\claimautorefname}{Claim}
\newcommand{\conjectureautorefname}{Conjecture}
\newcommand{\justificationautorefname}{Justification}
\newcommand{\definitionautorefname}{Definition}
\newcommand{\notationautorefname}{Notation}
\newcommand{\exampleautorefname}{Example}
\newcommand{\counterexampleautorefname}{Counterexample}
\newcommand{\ideaautorefname}{Idea}

\ifbook
    \renewcommand{\theequation}{\thechapter.\arabic{equation}}
    \renewcommand{\thepretheorem}{\thechapter.\arabic{pretheorem}}
    \renewcommand{\theprelemma}{\thechapter.\arabic{prelemma}}
    \renewcommand{\thepreproposition}{\thechapter.\arabic{preproposition}}
    \renewcommand{\theprecorollary}{\thechapter.\arabic{precorollary}}
    \renewcommand{\thepreclaim}{\thechapter.\arabic{preclaim}}
    \renewcommand{\thepreconjecture}{\thechapter.\arabic{preconjecture}}
    \renewcommand{\theprejustification}{\thechapter.\arabic{prejustification}}
    \renewcommand{\thepredefinition}{\thechapter.\arabic{predefinition}}
    \renewcommand{\theprenotation}{\thechapter.\arabic{prenotation}}
    \renewcommand{\thepreexample}{\thechapter.\arabic{preexample}}
    \renewcommand{\theprecounterexample}{\thechapter.\arabic{precounterexample}}
    \counterwithin*{equation}{chapter}
    \counterwithin*{pretheorem}{chapter}
    \counterwithin*{prelemma}{chapter}
    \counterwithin*{preproposition}{chapter}
    \counterwithin*{precorollary}{chapter}
    \counterwithin*{preclaim}{chapter}
    \counterwithin*{preconjecture}{chapter}
    \counterwithin*{prejustification}{chapter}
    \counterwithin*{predefinition}{chapter}
    \counterwithin*{prenotation}{chapter}
    \counterwithin*{preexercise}{chapter}
    \counterwithin*{preexample}{chapter}
    \counterwithin*{precounterexample}{chapter}
\else
\fi

\newcommand*{\xline}[1][3em]{\rule[0.5ex]{#1}{0.55pt}}

\newcommand{\isomorphic}{\cong}
\newcommand{\iffdefn}{~\mathrel{\vcentcolon\Leftrightarrow}~}
\newcommand{\iffbydefn}{\(\mathrel{\vcentcolon\Leftrightarrow}\)\ }
\newcommand{\niff}{\mathrel{{\ooalign{\hidewidth$\not\phantom{"}$\hidewidth\cr$\iff$}}}}
\renewcommand{\implies}{\Rightarrow}
\renewcommand{\iff}{\Leftrightarrow}
\newcommand{\proves}{\vdash}
\newcommand{\satisfies}{\models}
\renewcommand{\qedsymbol}{\sc q.e.d.}

\renewcommand{\restriction}[1]{\downharpoonright_{#1}}
\renewcommand{\leq}{\leqslant}
\renewcommand{\geq}{\geqslant}

\newcommand{\meet}{\wedge}
\newcommand{\join}{\vee}
\newcommand{\conjunct}{\wedge}
\newcommand{\disjunct}{\vee}
\newcommand{\bigmeet}{\bigwedge}
\newcommand{\bigjoin}{\bigvee}
\newcommand{\bigconjunct}{\bigwedge}
\newcommand{\bigdisjunct}{\bigvee}
\newcommand{\defn}{\coloneqq}
\newcommand{\xor}{\oplus}
\newcommand{\nand}{\uparrow}
\newcommand{\nor}{\downarrow}

\newcommand{\compose}{\circ}
\newcommand{\divides}{~|~}
\newcommand{\notdivides}{\not|~}
\newcommand{\given}{~\middle|~}
\newcommand{\suchthat}{~\middle|~}
\newcommand{\contradiction}{~\smash{\text{\raisebox{-0.6ex}{\Large \Lightning}}}~}

\newcommand{\conjugate}[1]{\overline{#1}}
\newcommand{\mean}[1]{\overline{#1}}

\newcommand*\diff{\mathop{}\!\mathrm{d}}
\newcommand{\integral}[1]{\smashoperator{\int_{#1}}}
\newcommand{\E}[1]{\mathbb{E}\sq*{#1}}
\newcommand{\Esub}[2]{\mathbb{E}_{#1}\sq*{#2}}
\newcommand{\var}[1]{\mathrm{Var}\pn*{#1}}
\newcommand{\cov}[2]{\mathrm{Cov}\pn*{#1, #2}}
\newcommand{\der}[2]{\frac{\diff{#1}}{\diff{#2}}}
\newcommand{\dern}[3]{\frac{\diff^{#3}{#1}}{\diff{#2}^{#3}}}
\newcommand{\derm}[3]{\frac{\diff^{#3}{#1}}{\diff{#2}}}
\newcommand{\prt}[2]{\frac{\partial{#1}}{\partial{#2}}}
\newcommand{\prtn}[3]{\frac{\partial^{#3}{#1}}{\partial{#2}^{#3}}}
\newcommand{\prtm}[3]{\frac{\partial^{#3}{#1}}{\partial{#2}}}
\newcommand{\modulo}[1]{~\pn{\mathrm{mod}~#1}}

\newcommand{\inj}{\hookrightarrow}
\newcommand{\injection}{\hookrightarrow}

\newcommand{\surj}{\twoheadrightarrow}
\newcommand{\surjection}{\twoheadrightarrow}

\newcommand{\bij}{\lhook\joinrel\twoheadrightarrow}
\newcommand{\bijection}{\lhook\joinrel\twoheadrightarrow}

\newcommand{\monic}{\hookrightarrow}
\newcommand{\monomorphism}{\hookrightarrow}

\newcommand{\epic}{\twoheadrightarrow}
\newcommand{\epimorphism}{\twoheadrightarrow}

\newcommand{\iso}{\lhook\joinrel\twoheadrightarrow}
\newcommand{\isomorphism}{\lhook\joinrel\twoheadrightarrow}
\newcommand{\immersion}{\looprightarrow}

\renewcommand{\O}[1]{\mathcal{O}\pn*{#1}}
\renewcommand{\P}[1]{\mathbb{P}\pn*{#1}}
\newcommand{\power}[1]{\mathcal{P}\pn*{#1}}
\newcommand{\successor}[1]{\mathcal{S}\pn*{#1}}
\newcommand{\C}{\mathbb{C}}
\newcommand{\N}{\mathbb{N}}
\newcommand{\Q}{\mathbb{Q}}
\newcommand{\R}{\mathbb{R}}
\newcommand{\Z}{\mathbb{Z}}

% these don't need {} after them since they should be followed by text
\newcommand{\cf}{\textit{c.f.},\ }
\newcommand{\eg}{\textit{e.g.},\ }
\newcommand{\ie}{\textit{i.e.},\ }
\newcommand{\aka}{\textit{a.k.a.}\ }
\newcommand{\viz}{\textit{viz.}\ }
\newcommand{\vide}{\textit{v.}\ }
\newcommand{\ifandonlyif}{\textit{iff}\ }

% these need {} after them
\newcommand{\etal}{\textit{et al.}}
\newcommand{\wff}{\textit{wff}}

\DeclareMathOperator{\lcm}{lcm}
\DeclareMathOperator*{\argmin}{arg\!\min}
\DeclareMathOperator*{\argmax}{arg\!\max}

\let\originalleft\left
\let\originalright\right
\renewcommand{\left}{\mathopen{}\mathclose\bgroup\originalleft}
\renewcommand{\right}{\aftergroup\egroup\originalright}

\newcommand{\zh}[1]{\begin{CJK}{UTF8}{gbsn}#1\end{CJK}}
\newcommand{\jp}[1]{\begin{CJK}{UTF8}{gbsn}#1\end{CJK}}

\DeclarePairedDelimiterX \inner[2]{\langle}{\rangle}{#1,#2}
\DeclarePairedDelimiter \bra{\langle}{\rvert}
\DeclarePairedDelimiter \ket{\lvert}{\rangle}
\DeclarePairedDelimiter \abs{\lvert}{\rvert}
\DeclarePairedDelimiter \cardinality{\lvert}{\rvert}
\DeclarePairedDelimiter \norm{\lVert}{\rVert}
\DeclarePairedDelimiter \set{\lbrace}{\rbrace}
\DeclarePairedDelimiter \seq{\langle}{\rangle}
\DeclarePairedDelimiter \pn{(}{)}
\DeclarePairedDelimiter \sq{[}{]}
\DeclarePairedDelimiter \curly{\lbrace}{\rbrace}
\DeclarePairedDelimiter \bracket{\langle}{\rangle}

\let\oldemptyset\emptyset
\let\emptyset\varnothing
\let\union\cup
\let\intersection\cap
\let\intersect\cap


\begin{document}

\title{Discrete Mathematics}
\author{Daniel Gonzalez Cedre}
\date{University of Notre Dame \\ Spring of 2023}
\maketitle

\setcounter{chapter}{2}
\chapter{Zermelo-Fraenkel Set Theory}
\begin{quote}
    ``No one shall expel us from the paradise that Cantor has created.''
    \begin{flushright}
        ---David Hilbert
    \end{flushright}
\end{quote}

\section{The Language of Set Theory}
In order to use our first-order logic as a language with which to talk about math,
we need to specify: what is our universe of discourse $\Omega$, and what are our fundamental predicate symbols?
% \begin{enumerate}
%     \item
%         What is our universe of discourse $\Omega$?
%     \item
%         What are our fundamental predicate symbols?
% \end{enumerate}
% With these two questions answered,
% we will be able to take objects from $\Omega$ and make atomic formul{\ae} out of them,
% which we can then use to construct our {\wff}.
The analogy drawn in class between the structure of the study of mathematics
and the abstract structure of a modern computer
should hopefully communicate how natural and common the notion is of having one \emph{type} of object
that implements other, more complicated objects.
However, this decision actually goes back to the beginning of
this $20$\textsuperscript{th} century revolution in mathematics.
The initial solution people came up with to the fundamental logical and foundational problems they had discovered
has to use a \emph{ramified}---or \emph{typed}---ontology,
where different objects had different \emph{types} in a hierarchy and there were rules governing how objects
could be manipulated based on their \emph{type}.
The problem with this approach is that it becomes very syntactically-cumbersome for humans
(the primary practitioners of mathematics) to deal with directly,
so it was quickly abandoned for an \emph{untyped} approach.
\vspace{-\parskip}
\begin{note}
    The converse is true about the \emph{$\lambda$-calculus},
    which is perhaps the most famous mathematical model of computation after the Turing machine.
    Although the untyped $\lambda$-calculus is more expressive (\ie stronger)
    than any of the typed $\lambda$-calculi (and is thus more interesting to study for mathematicians),
    the modern functional programming languages we have today (\eg Haskell, the LISP dialects, F\#)
    are actually implementations of typed $\lambda$-calculi
    because computers have no issues dealing with the syntactic complications of a typed theory.
\end{note}
\vspace{-\parskip}
Our universe of discourse will consist of \emph{those objects that we can prove exist}
using the rules of inference and the \emph{axioms of set theory} (which we will develop in this chapter).
The axioms of set theory will be sentences in the \emph{language of Zermelo-Fraenkel set theory}
that describe \emph{what exactly sets are} and \emph{how they work}.
The language of set theory will have two predicate symbols, defined below.

\begin{definition}[Equality]\label{def:equal}
    We define the binary predicate $=$ to mean that
    its left argument is \emph{identical to} its right argument;
    \ie we say $x = y$
    when we mean that the symbols $x$ and $y$ refer to the same object.
    We will take the following axioms for equality:
    \[
        \forall x \pn*{x = x}
        ~~~~~~~~~~~~~~~~
        \forall x \forall y \pn*{\pn*{x = y} \iff \pn*{y = x}}
        ~~~~~~~~~~~~~~~~
        \forall x \forall y \forall z \pn*{\pn*{x = y \meet y = x} \implies \pn*{x = z}}
    \]
\end{definition}

\begin{definition}[Elementhood]\label{def:element}
    We define the binary predicate $\in$ to mean that its left argument is
    contained in its right argument as an element.
    So, if $x$ and $y$ are sets, then the phrase $x \in y$
    conveys that $x$ is an element of $y$.
\end{definition}

\begin{definition}[Language of Set Theory]
    The \emph{language of Zermelo-Fraenkel set theory}
    consists of the first-order logic
    along with
    \begin{enumerate}
        \item[I.]
            a universe of discourse consisting of
            those things that provably exist from the axioms (\autoref{sec:axioms}),
        \item[II.]
            the binary predicates for equality ($=$, \autoref{def:equal})
            and elementhood ($\in$, \autoref{def:element}).
    \end{enumerate}
\end{definition}

\section{The Axioms of Set Theory}\label{sec:axioms}
\setcounter{axiom}{-1}

\subsection{The Axiom of Existence}
\begin{axiom}[Existence]\label{ax:existence}
    ~\vspace{-\baselineskip}
    \[
        \exists x \pn*{x = x}
    \]
    This axiom asserts that our universe of discourse is non-empty.
    Assuming this axiom lets us know for sure that when we make claims about sets,
    those claims are actually in reference to objects that provably exist
    (because we can use this axiom as an assumption in any proof).
\end{axiom}

\subsection{The Axiom of Extensionality}
\begin{axiom}[Extensionality]\label{ax:extensionality}
    ~\vspace{-\baselineskip}
    \[
        \forall x \forall y \pn*{\pn*{x = y} \iff \forall z \pn*{z \in x \iff z \in y}}
    \]
    This axiom states that sets are equal \ifandonlyif they have the same elements,
    capturing what we mean by two sets being (or not being) equal.
    This establishes a fundamental relationship between $=$ and $\in$.
\end{axiom}

\begin{definition}[$\in$-Augmented Quantification]
    When we want to talk about all of the elements of a set $A$ that satisfy a given {\wff} $\varphi(\cdot)$,
    we say
    \[
        \pn*{\forall x \in A}\pn*{\varphi(x)} \iffdefn \forall x \pn*{x \in A \implies \varphi(x)}.
    \]
    Similarly, if we want to say that there is an element in $A$ with the property $\varphi(\cdot)$,
    we say
    \[
        \pn*{\exists x \in A}\pn*{\varphi(x)} \iffdefn \exists x \pn*{x \in A \meet \varphi(x)}.
    \]
    The parentheses around $\pn*{\forall x \in A}$ and $\pn*{\exists x \in A}$
    can be added or dropped for clarity based on context.
\end{definition}

\begin{theorem}[First-Order Set Notation]
    Let $\varphi$ be a {\wff} with at most one free variable and let $A$ be a set.
    Then, the following statements hold.
    \[
        \pn*{\forall a \in A}\pn*{\varphi(a)} \iff \pn*{\bigmeet_{a \in A}\varphi(a)}
        ~~~~~~~~~~~~~~~~
        \pn*{\exists a \in A}\pn*{\varphi(a)} \iff \pn*{\bigjoin_{a \in A}\varphi(a)}
    \]
\end{theorem}
% \begin{proof}
%     This proof is left as an exercise to the reader.
% \end{proof}
% \begin{proof}
%     Let $\varphi$ be a {\wff} with at most one free variable $x$.
%     Let $A$ be a set.
%     We know, by definition, that
%     $\pn*{\forall x \in A}\pn*{\varphi(x)} \iff \forall x \pn*{x \in A \implies \varphi(x)}$
%     and that
%     $\pn*{\exists x \in A}\pn*{\varphi(x)} \iff \exists x \pn*{x \in A \meet \varphi(x)}$.
% 
%     First, suppose that $\pn*{\forall x \in A}\pn*{\varphi(x)}$.
%     Then, letting $a$ be an arbitrary set, we have $a \in A \implies \varphi(a)$.
% \end{proof}

\begin{definition}[Set-Builder Notation]\label{not:setbuilder}
    The notation $\set*{x_1, \dots x_n}$,
    where each $x_1, \dots x_n$ is a term, 
    describes the set containing each of the $x_i$'s, and \emph{only} the $x_i$'s, as elements.
    So, we say $\set*{x_1, \dots x_n}$ denotes the set $X$ satisfying
    $
        \forall a \pn*{a \in X \iff \bigjoin_{i = 1}^{n}\pn*{a = x_i}}.
    $
    Another way of saying this is that $X \defn \set*{x_1, \dots x_n}$ is the set that satisfies
    \[
        \pn*{\bigmeet_{i = 1}^{n}x_i \in X}
             \meet \forall a \pn*{a \in X \implies \pn*{\exists x \in X \pn*{a = x}}}.
    \]
\end{definition}

\begin{definition}[Set-Comprehension Notation]
    If $\varphi$ is a {\wff} with at most one free variable,
    then when we write down $\set*{a \suchthat \varphi(a)}$,
    we mean the set consisting of all possible $a$ satisfying the formula $\varphi$
    when the every occurrence of the free variable in $\varphi$ is replaced by $a$.
    Similarly, if we have an already-existing set $A$,
    we can define the set of all elements of $A$ that satisfy $\varphi$
    with the notation $\set*{a \in A \suchthat \varphi(a)}$.
    These notations are read ``the set of all $a$ (in $A$) such that $\varphi(a)$''.
    More precisely,
    \begin{align*}
        \exists x \pn*{x = \set*{a \suchthat \varphi(a)}}
            &\iff \exists x \forall a \pn*{a \in x \iff \varphi(a)},\\
        \exists x \pn*{x = \set*{a  \in A \suchthat \varphi(a)}}
            &\iff \exists x \forall a \pn*{a \in x \iff \pn*{a \in A \meet \varphi(a)}}.
    \end{align*}

    \begin{note}
        These sets may not always exist!
        Remember that we can only speak about the objects that \emph{provably} exist,
        so when use this notation,
        we must be sure (with proof) that it is actually a set!
    \end{note}
\end{definition}

\subsection{The Pairing Axiom}
\begin{axiom}[Pairing]\label{ax:pairing}
    ~\vspace{-\baselineskip}
    \[
        \forall x \forall y \exists z \pn*{z = \set*{x, y}}
    \]
    This axiom allows us to take two existing sets and construct a set containing the pair of them.
\end{axiom}

\subsection{The Axiom of Union}
\begin{axiom}[Union]\label{ax:union}
    ~\vspace{-\baselineskip}
    \[
        \forall x \exists A \pn*{A = \set*{z \suchthat \pn*{\exists y \in x}\pn*{z \in y}}}
    \]
    The notation we use for this set $A$, which we call the \emph{union of $x$},
    is $\union x \defn \set*{z \suchthat \pn*{\exists y \in x}\pn*{z \in y}}$.
    % Notice carefully that this is \emph{not} the union of $x$ with some other set;
    % it is \emph{just} the union of $x$.
\end{axiom}

\begin{theorem}[Union of Two Sets]\label{thm:union}
    If $x$ and $y$ are sets,
    then $x \union y \defn \set*{z \suchthat \pn*{z \in x} \join \pn*{z \in y}}$ exists.
\end{theorem}
\begin{proof}
    Let $x$ and $y$ be sets.
    By \autoref{ax:pairing}, we know $A \defn \set*{x, y}$ exists.
    By \autoref{ax:union}, we know $\union A$ exists.
    Recall that $\union A = \set*{a \suchthat \pn*{\exists z \in A}{a \in z}}$.
    Now, let $b$ be an arbitrary set and observe
    \begin{alignat*}{2}
        b \in \union A &\iff \pn*{\exists z \in A}\pn*{b \in z} &&\text{by definition of $\union A$} \\
                       &\iff (b \in x) \join (b \in y) &&\text{since $A = \set*{x, y}$} \\
                       &\iff b \in \set*{a \suchthat (a \in x) \join (a \in y)}~~~~&&\text{by definition}.
    \end{alignat*}
    Therefore, we have that $\union A = \set*{a \suchthat (a \in x) \join (a \in y)}$,
    so $x \union y \defn \set*{a \suchthat (a \in x) \join (a \in y)}$ exists.
\end{proof}

\subsection{The Axiom Schema of Separation}
\begin{axiom}[Separation]\label{ax:separation}
    ~\vspace{-\baselineskip}
    \[
        \text{If $\varphi$ is a {\wff} with at most one free variable, then }
        \forall x \exists y \pn*{y = \set*{a \in x \suchthat \varphi(a)}}
    \]
    % Given an already-existing set $x$,
    % this axiom stats that we can \emph{separate} the elements that satisfy $\varphi$ from $x$.
\end{axiom}

\begin{theorem}[Intersection of Two Sets]\label{thm:intersection}
    If $x$ and $y$ are sets,
    then $x \intersect y \defn \set*{a \suchthat (a \in x) \meet (a \in y)}$ exists.
\end{theorem}
\begin{proof}
    Let $x$ and $y$ be sets. Recall that $x \union y$ exists by \autoref{thm:union}.
    Then, by \autoref{ax:separation},
    we know that $A \defn \set*{a \in x \union y \suchthat (a \in x) \meet (a \in y)}$ exists.
    Now, let $b$ be an arbitrary set and observe
    \begin{alignat*}{2}
        b \in A &\iff b \in \set*{a \in x \union y \suchthat (a \in x) \meet (a \in y)} &&\text{by the definition of $A$} \\
                &\iff (b \in x \union y) \meet \pn*{(b \in x) \meet (b \in y)} &&\text{by definition} \\
                &\iff \pn*{(b \in x) \join (b \in y)} \meet \pn*{(b \in x) \meet (b \in y)} &&\text{by definition of } x \union y \\
                &\iff \pn*{\pn*{(b \in x) \join (b \in y)} \meet (b \in x)} \meet (b \in y) &&\text{by Associativity} \\
                &\iff \pn*{\pn*{(b \in x) \meet (b \in x)} \join \pn*{(b \in y) \meet (b \in x)}} \meet (b \in y)~~~~&&\text{by Distributivity} \\
                &\iff \pn*{(b \in x) \join \pn*{(b \in y) \meet (b \in x)}} \meet (b \in y) &&\text{by Idempotency} \\
                &\iff \pn*{(b \in x) \join \pn*{(b \in x) \meet (b \in y)}} \meet (b \in y) &&\text{by Commutativity} \\
                &\iff \pn*{\pn*{(b \in x) \join (b \in x)} \meet (b \in y)} \meet (b \in y) &&\text{by Associativity} \\
                &\iff \pn*{(b \in x) \meet (b \in y)} \meet (b \in y) &&\text{by Idempotency} \\
                &\iff (b \in x) \meet \pn*{(b \in y) \meet (b \in y)} &&\text{by Associativity} \\
                &\iff (b \in x) \meet (b \in y) &&\text{by Idempotency} \\
                &\iff b \in \set*{a \suchthat (a \in x) \meet (a \in y)} &&\text{by definition}.
    \end{alignat*}
    Therefore, we have that $A = \set*{a \suchthat (a \in x) \meet (a \in y)}$,
    showing that $x \intersect y = \set*{a \suchthat (a \in x) \meet (a \in y)}$ exists.
\end{proof}

\begin{theorem}[Difference of Two Sets]
    If $x$ and $y$ are sets,
    then $x \setminus y \defn \set*{a \suchthat (a \in x) \meet (a \not \in y)}$ exists.
\end{theorem}
\begin{proof}
    Let $x$ and $y$ be sets.
    Consider $A \defn \set*{a \in x \suchthat a \not \in y}$,
    which we know exists by \autoref{ax:separation}.
    Now, let $b$ be an arbitrary set and observe
    \begin{alignat*}{2}
        b \in A &\iff b \in \set*{a \in x \suchthat a \not \in y} &&\text{by the definition of $A$} \\
                &\iff \pn*{b \in x} \meet \pn*{b \not \in y} &&\text{by definition} \\
                &\iff b \in \set*{a \suchthat \pn*{a \in x} \meet \pn*{a \not \in y}}~~~~&&\text{by definition}.
    \end{alignat*}
    Therefore, $A = \set*{a \suchthat \pn*{a \in x} \meet \pn*{a \not \in y}}$ by \autoref{ax:extensionality},
    showing that $x \setminus y = \set*{a \suchthat \pn*{a \in x} \meet \pn*{a \not \in y}}$ exists.
\end{proof}

\begin{definition}[Subsets]
    We say that $x$ is a \emph{subset} of $y$ when every element of $x$ is also an element of $y$.
    Formally,
    \[
        x \subseteq y \iffdefn \forall a \pn*{a \in x \implies a \in y}.
    \]
    If $x \subseteq y$ but $x \neq y$,
    then we say that $x$ is a \emph{proper subset} of $y$ and we write $x \subsetneq y$
    (you may also encounter the alternative notations $x \subset y$ and $x \subsetneqq y$).
\end{definition}

\subsection{The Power Set Axiom}
\begin{axiom}[Power Set]
    ~\vspace{-\baselineskip}
    \[
        \forall x \exists y \pn*{y = \set*{s \suchthat s \subseteq x}}.
    \]
    This axiom tells us that, whenever we have a set $x$ that exists,
    we are allowed to collect \emph{all} of its subsets into one set, called the \emph{power set} of $x$.
    We use the notation $\power{x}$.
\end{axiom}

\begin{definition}[Empty Set]
    We say that a set $A$ is \emph{empty} \iffbydefn $\forall x \pn*{x \not \in A}$.
    Since there can only be one empty set by \autoref{ax:extensionality},
    we refer to this set as \emph{the} empty set and use the special symbol $\emptyset$.
\end{definition}

\subsection{The Axiom of Regularity}
\begin{axiom}[Regularity]\label{ax:regularity}
    ~\vspace{-\baselineskip}
    \[
        \forall x \pn*{\pn*{x \neq \emptyset} \implies \pn*{\exists y \in x}\pn*{x \intersect y = \emptyset}}.
    \]
    This axiom establishes the \emph{regularity} (\aka \emph{well-foundedness}) of the $\in$ predicate.
    % Literally, this axiom simply says that every non-empty set has an element disjoint with it.
    % However, this seemingly strange axiom has far-reaching consequences,
    % not least of which is that it lets us prove that sets do not contain themselves.
\end{axiom}

\begin{theorem}[$\in$ Well-foundedness]
    For any set $x$, we have $x \not \in x$.
\end{theorem}
\begin{proof}
    Let $x$ be an arbitrary set.
    Towards a contradiction, assume $x \in x$.
    Let $y \defn \set*{a \in x \suchthat a = x} = \set*{x}$, which we know exists by \autoref{ax:separation}.
    Since $x \in y$, we know $y \neq \emptyset$.
    Thus, \autoref{ax:regularity} lets us know there is a $z \in y$ such that $y \intersect z = \emptyset$.
    Since $y = \set*{x}$ and $z \in y$, this implies $z = x$, yielding $y \intersect x = \emptyset$.
    However, we already knew that $x \in y$ and $x \in x$,
    so we know by definition that $x \in \set*{a \suchthat a \in y \meet a \in x} = y \intersect x$,
    and therefore $y \intersect x \neq \emptyset$. \contradiction
    Therefore, we must have that $x \not \in x$.
    Since $x$ was arbitrary, we can conclude $\forall x \pn*{x \not \in x}$, as desired.
\end{proof}
% \begin{proof}
%     Let $X$ be an arbitrary set.
%     Towards a contradiction, assume $X \in X$.
%     Let $Y \defn \set*{a \in X \suchthat a = X} = \set*{X}$, which we know exists by \autoref{ax:separation}.
%     Since $X \in Y$, we know $Y \neq \emptyset$.
%     Thus, \autoref{ax:regularity} lets us know there is a $Z \in Y$ such that $Y \intersect Z = \emptyset$.
%     Since $Y = \set*{X}$ and $Z \in Y$, this implies $Z = X$, yielding $Y \intersect X = \emptyset$.
%     However, we already knew that $X \in Y$ and $X \in X$,
%     so we know by definition that $X \in \set*{a \suchthat a \in Y \meet a \in X} = Y \intersect X$,
%     and therefore $Y \intersect X \neq \emptyset$. \contradiction
% 
%     Therefore, we must have that $X \not \in X$.
%     Since $X$ was arbitrary, we can conclude $\forall x \pn*{x \not \in x}$, as desired.
% \end{proof}

\begin{definition}[Kuratowski Ordered Pairs]
    We define the \emph{ordered pair} with left coordinate $a$ and right coordinate $b$
    by $(a, b) \defn \set*{\set*{a}, \set*{a, b}}$.
    It is left as a simple exercise to the reader to prove that $(a, b)$ always exists.
\end{definition}

% \begin{theorem}[Adequacy of Ordered Pairs]
%     We have $(a, b) = (x, y) \iff (a = x) \meet (b = y)$ for any sets $a, b, x, y$.
% \end{theorem}
% \begin{proof}
%     Let $a, b, x, y$ be sets.
%     Recall that $(a, b) = \set*{\set*{a}, \set*{a, b}}$ and $(x, y) = \set*{\set*{x}, \set*{x, y}}$.
%     There are two directions to this proof: the $\implies$ direction, and the $\impliedby$ direction.
% 
%     \begin{case}[Forward direction]
%         Suppose $(a, b) = (x, y)$ and assume, towards a contradiction, that $a \neq x$ or $b \neq y$.
%         
%     \end{case}
% 
%     \begin{case}[Backward direction]
%         Suppose $a = x$ and $b = y$.
%         \autoref{ax:extensionality} then clearly shows $\set*{a} = \set*{x}$ and $\set*{a, b} = \set*{x, y}$.
%         Therefore, $\set*{\set*{a}, \set*{a, b}} = \set*{\set*{x}, \set*{x, y}}$,
%         so $(a, b) = (x, y)$.
%     \end{case}
% 
%     Towards a 
% 
%     Observe
% \end{proof}

\begin{definition}[Cartesian Product]
    We define the \emph{Cartesian product} of two sets $A$ and $B$,
    which is the set of all possible ordered pairs with
    left coordinate coming from $A$ and right coordinate coming from $B$,
    by $A \times B \defn \set*{(a, b) \suchthat a \in A \meet b \in B}$.
\end{definition}

\begin{theorem}[Existence of Cartesian Products]
    For any sets $X$ and $Y$, the Cartesian product $X \times Y$ exists.
\end{theorem}
% \begin{proof}
%     This is left as a (tedious) exercise for the reader.
% \end{proof}

\begin{theorem}[Existence of $\emptyset$]
    $\exists x \pn*{x = \emptyset}$.
\end{theorem}
\begin{proof}
    We know, by \autoref{ax:existence}, that there exists a set $x$.
    Now, consider $E \defn \set*{y \in x \suchthat  y \neq y}$,
    which we know exists by \autoref{ax:separation}.
    We can clearly see that $E$ is empty,
    because if $x \in E$ then we would immediately run into the contradiction $x \neq x$.
    Therefore, we have $E = \emptyset$, so the empty set exists.
\end{proof}

\begin{definition}[Successor Functional]
    Given an arbitrary set $x$, we define the \emph{successor} of $x$ to be the set $x \union \set*{x}$,
    for which we introduce the notation $\successor{x} \defn x \union \set*{x}$.
    It should be clear that the successor of any set always exists.

    \begin{note}
        The successor $\successor{x}$ of a set $x$ is a set consisting of
        all of the elements of $x$ as well as $x$ itself.
    \end{note}
\end{definition}

\begin{definition}[von Neumann Ordinals]
    The \emph{natural numbers}, also known as the \emph{finite ordinal numbers},
    are defined recursively as follows:
    \begin{alignat*}{2}
        0 &\defn \emptyset &&\text{ is a natural number} \\
        n + 1 &\defn S(n)
               = n \union \set*{n}
               = \set{0, 1, 2, \dots n}~~~~&&\text{ if } n \text{ is a natural number}
    \end{alignat*}
    We say that a natural number $n$ is a \emph{successor number}
    \iffbydefn there exists a natural number $m$ such that $n = \successor{m}$.
    Every natural number except for $0$ is a successor number,
    so when we define properties or operations on the natural numbers,
    we will usually take a recursive approach that explicitly defines what happens at $0$
    and then specifies what happens when you've encountered a successor number.
    If $n$ is a successor natural number,
    then we define the \emph{predecessor} of $n$ to be $n - 1 \defn m$,
    where $m$ is a natural number such that $\successor{m} = n$.
\end{definition}

% \begin{theorem}[Uniqueness of Predecessors]
%     If $n$ is a natural number
%     such that $\exists m_1 \pn*{n = \successor{m_1}}$ and $\exists m_2 \pn*{n = \successor{m_2}}$,
%     then $m_1 = m_2$.
% \end{theorem}
% \begin{proof}
%     
% \end{proof}

\begin{definition}[Arithmetic on the Natural Numbers]
    Given a natural number $n$,
    its \emph{sum} with another natural number $m$ is recursively given by
    \begin{align*}
        n + 0 &\defn n\\
        % n + 1 &\defn \successor{n}\\
        n + \successor{m} &\defn \successor{n + m}~~~~\text{ if } m \text{ is a successor number}
    \end{align*}

    Similarly, we define the \emph{product} of two natural numbers $n$ and $m$ recursively by
    \begin{align*}
        n \cdot 0 &\defn 0\\
        % n \cdot 1 &\defn n\\
        n \cdot \successor{m} &\defn \pn*{n + m} + n~~~~\text{ if } m \text{ is a successor number}
    \end{align*}

    Finally, we can recursively define the \emph{exponentiation} of one natural number $n$ by another as
    \begin{align*}
        n^{0} &\defn 1\\
        % n^{1} &\defn n\\
        n^{\successor{m}} &\defn n^{m} \cdot m~~~~\text{ if } m \text{ is a successor number}
    \end{align*}
\end{definition}

\begin{example}[$2 + 3 = 5$]
    ~\vspace{-\baselineskip}
    \begin{alignat*}{2}
        2 + 3 &= 2 + \successor{2} &&\text{ because } 3 = \successor{2}\\
              &= \successor{2 + 2} &&\text{ by definition of } 2 + \successor{2}\\
              &= \successor{2 + \successor{1}} &&\text{ because } 2 = \successor{1}\\
              &= \successor{\successor{2 + 1}} &&\text{ by definition of } 2 + \successor{1}\\
              &= \successor{\successor{2 + \successor{0}}} &&\text{ because } 1 = \successor{0}\\
              &= \successor{\successor{\successor{2 + 0}}}~~~~&&\text{ by definition of } 2 + \successor{0}\\
              &= \successor{\successor{\successor{2}}} &&\text{ by definition of } 2 + 0\\
              &= \successor{\successor{3}} &&\text{ because } 3 = \successor{2} \text{ by definition}\\
              &= \successor{4} &&\text{ because } 4 = \successor{3} \text{ by definition}\\
              &= 5 &&\text{ because } 5 = \successor{4} \text{ by definition}.
    \end{alignat*}
    Thus, we have a proof that $2 + 3 = 5$.
\end{example}
\begin{example}[$2 \cdot 3 = 6$]
    ~\vspace{-\baselineskip}
    \begin{alignat*}{2}
        2 \cdot 3 &= 2 \cdot \successor{2} &&\text{ because } 3 = \successor{2}\\
              &= 2 \cdot 2 + 2 &&\text{ by definition of } 2 \cdot \successor{2}\\
              &= 2 \cdot \successor{1} + 2 &&\text{ because } 2 = \successor{1}\\
              &= \pn*{2 \cdot 1 + 2} + 2 &&\text{ by definition of } 2 \cdot \successor{1}\\
              &= \pn*{2 \cdot \successor{0} + 2} + 2 &&\text{ because } 1 = \successor{0}\\
              &= \pn*{\pn*{2 \cdot 0 + 2} + 2} + 2~~~~&&\text{ by definition of } 2 \cdot \successor{0}\\
              &= \pn*{\pn*{0 + 2} + 2} + 2 &&\text{ because } 2 \cdot 0 = 0 \text{ by definition}\\
              &= \pn*{2 + 2} + 2 &&\text{ because } 0 + 2 = 2 \text{ \emph{(this requires its own proof)}}\\
              &= 4 + 2 &&\text{ because } 2 + 2 = 4 \text{ \emph{(this requires its own proof)}}\\
              &= 6 &&\text{ because } 2 + 4 = 6 \text{ \emph{(this requires its own proof)}}
    \end{alignat*}
    Since those last three steps are not a part of any definition or theorem so far,
    we would need to \emph{separately} prove $0 + 2 = 2$ and $2 + 2 = 4$ and $2 + 4 = 6$
    and then apply those results to this proof of $2 \cdot 3 = 6$.
\end{example}

\begin{definition}[Order on the Natural Numbers]
    Given two natural numbers $n$ and $m$,
    say that $n$ is \emph{less than or equal to} $m$ \iffbydefn $n \subseteq m$.
    The syntax we use to express this relationship is $n \leq m$.

    If $n \leq m$ and $n \neq m$, then we say $n < m$,
    which means $n$ is \emph{strictly less than} $m$.
\end{definition}

\begin{definition}[Iterated Sums \& Products]
    If $n$ and $x_0, \dots x_{n - 1}$ are natural numbers,
    then we define the \emph{iterated sum} of the $x_i$ recursively by
    \begin{align*}
        \sum_{i = 0}^{0}x_i &= x_0\\
        \sum_{i = 0}^{k + 1}x_i &= \pn*{\sum_{i = 0}^{k}x_i} + \pn*{x_{k + 1}}~~~~\text{ if } 0 < k + 1 \leq n.
    \end{align*}
    Similarly, we define the \emph{iterated product} of the $x_i$ recursively as
    \begin{align*}
        \prod_{i = 0}^{0}x_i &= x_0\\
        \prod_{i = 0}^{k + 1}x_i &= \pn*{\prod_{i = 0}^{k}x_i} \cdot \pn*{x_{k + 1}}~~~~\text{ if } 0 < k + 1 \leq n.
    \end{align*}
\end{definition}

\begin{definition}[Parity]
    We say that a natural number $n$ is \emph{even} \iffbydefn
    $\pn*{\exists m}\pn*{\pn*{m \text{ is a natural number}} \meet \pn*{n = 2 \cdot m}}$.
    Conversely, we say that $n$ is \emph{odd} \iffbydefn
    $\pn*{\exists m}\pn*{\pn*{m \text{ is a natural number}} \meet \pn*{n = 2 \cdot m + 1}}$.
    The fact that every natural number is either even or odd requires a new kind of proof technique.
\end{definition}

\subsection{The Axiom of Infinity}
\begin{axiom}[Infinity]
    ~\vspace{-\baselineskip}
    \[
        \exists N \pn*{N = \set*{n \suchthat n = \emptyset \join \exists\pn*{m \in N}\pn*{n = \successor{m}}}}
    \]
    We call this set the \emph{set of natural numbers} and use the symbol $\N$ to denote it.
    For convenience,
    we also define the \emph{positive naturals} by $\N_+ \defn \set*{n \in \N \suchthat n > 0}$.
\end{axiom}

% APPENDIX
\appendix
\chapter{Construction of the Number Sets}
There are a few different ways of constructing of each of the important number sets.
We will use \emph{quotient constructions}---a fundamental technique in modern algebra---%
and present the standard construction for each set.

\section{The Integers}
Consider the equivalence relation $\sim$ on $\pn*{\N \times \N} \times \pn*{\N \times \N}$
given by $(a, b) \sim (c, d)$ \iffbydefn $a + d = c + b$.
Notice that this relation captures the idea that ordered pairs $(a, b), (c, d)$ are related
if $a - b = c - d$, so the ordered pairs $(a, b)$ represent ``differences'' between natural numbers.
However, since the difference of two natural numbers is not always defined,
we can still express that relationship using only addition,
and we consider them \emph{equivalent} if the differences they represent are the same.
We then define the \emph{equivalence class} of an ordered pair $(a, b) \in \N \times \N$ under the $\sim$
relation as $[(a, b)]_{\sim} = \set*{(c, d) \in \N \times \N \suchthat (a, b) \sim (c, d)}$.\\
We then say $\Z \defn \sfrac{\N \times \N}{\sim}$ is the set of integers,
which is $\set*{[(a, b)]_{\sim} \suchthat (a, b) \in \N \times \N}$.
So every integer, underneath, is an equivalence class of ordered pairs of natural numbers.
We define the integer $z \in \Z$ to be the equivalence class $[(a, b)]_{\sim}$ such that $z + b = a$,
meaning that $z$ represents the difference $a - b$.
% This simulates the idea of \emph{closing} $\N$ under the subtraction operation
% with a \emph{quotient construction} (a fundamental technique of modern algebra).\\
The arithmetic operations on $\N$ extend naturally to $\Z$
by operating on the representatives of the equivalence classes.
% define integer arithmetic

\section{The Rationals}
We take a similar approach to the rational numbers as we did the integers.
Consider the equivalence relation $\sim$ on $\Z \times \N_+$
given by $(p, q) \sim (r, s)$ \iffbydefn $p \cdot s = r \cdot q$.
This captures the essence of $\sfrac{p}{q} = \sfrac{r}{s}$.\\
We then say $\Q \defn \sfrac{\Z \times \N_+}{\sim}$ is the set of rational numbers,
which is $\set*{[(p, q)]_{\sim} \suchthat (p, q) \in \Z \times \N_+}$.
So, similarly to the integers,
we represent a rational number $\sfrac{p}{q}$
as an equivalence class of ordered pairs of integers with positive naturals
so that the equivalence class of $(p, q)$ represents all of those fractions equal to $\sfrac{p}{q}$.
The arithmetic operations on $\Z$ and $\N$ extend naturally to $\Q$
by operating on the equivalence class representatives.
% define rational arithmetic

\section{The Reals}
The approach we must take to define the reals is similar in some ways but very different in other ways.
We would still like a quotient construction,
but we can not define our relation on $\N$, $\Z$, or $\Q$ directly.
Instead, the idea here will then be to take quotients of these things called
\emph{Cauchy sequences} of rational numbers.

\begin{definition}[Cauchy Sequence]\label{def:cauchy}
    We say that a sequence $q: \N \to \Q$
    of rational numbers $\langle q_i \rangle_{i \in \N} = q_0, q_1, \dots $
    is \emph{Cauchy convergent} {\ifandonlyif}
    \[
        \pn*{\forall \varepsilon > 0}\pn*{\exists m \in \N}\pn*{\forall n_1, n_2 \in \N}
        \pn*{\pn*{n_1 > m~\meet~n_2 > m} \implies \abs{q_{n_1} - q_{n_2}} < \varepsilon}.
    \]
    We call such sequences \emph{Cauchy sequences}, with the adjective \emph{Cauchy} for short.
\end{definition}

We then define $\sim$ on the set of all rational Cauchy sequences $\mathcal{Q}$
$\langle p_i \rangle_{i \in \N} \sim \langle q_i \rangle_{i \in \N}$ \iffbydefn
$\langle p_i - q_i \rangle_{i \in \N} = \langle 0 \rangle_{i \in \N}$.
Then $\R \defn \sfrac{\mathcal{Q}}{\sim}$ is the set of real numbers
(\ie each real number is a class of \emph{equivalent} Cauchy sequences over $\Q$).

% \section{The Complex Numbers}


\chapter{Basic Algebraic Properties}
\section{The Naturals}
\begin{theorem}[$\N$ is \emph{the smallest} Ordered Semiring]
    \hl{TODO}
\end{theorem}

\section{The Integers}
\begin{theorem}[$\Z$ is \emph{the smallest} Ordered Ring]
    \hl{TODO}
\end{theorem}

\section{The Rationals}
\begin{theorem}[$\Q$ is \emph{the smallest} Ordered Field]
    \hl{TODO}
\end{theorem}

\section{The Reals}
\begin{theorem}[$\R$ is \emph{the smallest} Complete Ordered Field]
    \hl{TODO}
\end{theorem}

% \section{The Complex Numbers}


\end{document}
