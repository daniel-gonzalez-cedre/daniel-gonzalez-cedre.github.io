% define new conditionals: \if<conditional>
\newif\ifbook  % whether or not to use book styles
\newif\ifdate  % whether or not to include the current date header
\newif\ifalgorithms  % whether or not to define the algorithm environment
\newif\ifdaggerfootnotes  % whether or not to have fnsymbol footnotes or numerical ones

% set the conditionals: \<conditional>true / \<conditional>false
\bookfalse
\datefalse
\algorithmsfalse
\daggerfootnotestrue

\ifbook
    \documentclass[letterpaper]{book}
    \usepackage{arydshln, chngcntr}
\else
    \documentclass[letterpaper]{article}
\fi

% math
\usepackage{amsmath, amsfonts, amssymb, amstext, amscd, amsthm, mathrsfs, mathtools, xfrac}
% fonts
\usepackage{bbm, CJKutf8, caption, dsfont, marvosym, stmaryrd}
% tables
\usepackage{booktabs, colortbl, makecell}
% colors
\usepackage{color, soul, xcolor}
% references
\usepackage{xr-hyper, hyperref, url}
% figures
\usepackage{graphicx, float, subcaption, tikz}
% headers and footers
\usepackage{fancyhdr, lastpage}
% miscellaneous
\usepackage{enumerate, ifthen, lipsum, listings, makeidx, parskip, ulem, verbatim, xargs}
\usepackage[nodayofweek]{datetime}

\usepackage[left=2cm,top=2cm,right=2cm,bottom=2cm,bindingoffset=0cm]{geometry}
\usepackage[group-separator={,},group-minimum-digits={3}]{siunitx}
\usepackage[shortlabels]{enumitem}
\setlist[enumerate]{topsep=0ex,itemsep=0ex,partopsep=1ex,parsep=1ex}
\setlist[itemize]{topsep=0ex,itemsep=0ex,partopsep=1ex,parsep=1ex}

\usepackage[math]{cellspace}
\cellspacetoplimit 1pt
\cellspacebottomlimit 1pt

\definecolor{gruvred}{HTML}{CC214D}
\definecolor{gruvorange}{HTML}{D65D0E}
\definecolor{gruvaqua}{HTML}{689D6A}
\definecolor{gruvpurple}{HTML}{B16286}
\definecolor{colorblack}{HTML}{252422}
\definecolor{colorgrey}{HTML}{f4efef}
\definecolor{colorblue}{HTML}{045275}
\definecolor{colorteal}{HTML}{089099}
\definecolor{colorgreen}{HTML}{7ccba2}
\definecolor{coloryellow}{HTML}{ffc61e}  % fcde9c % ffc61e  % b8860b
\definecolor{colororange}{HTML}{f0746e}
\definecolor{colorred}{HTML}{dc3977}
\definecolor{colorpurple}{HTML}{7c1d6f}

\hypersetup{
    colorlinks=true,
    linkcolor=gruvorange,
    citecolor=gruvaqua,
    urlcolor=gruvpurple
}

\allowdisplaybreaks
\newdateformat{verbosedate}{\ordinal{DAY} of \monthname[\THEMONTH], \THEYEAR}
\verbosedate

\pagestyle{fancy}
% \fancyfoot[C]{--~\thepage~--}
\fancyfoot[C]{\tiny \thepage\ / \pageref*{LastPage}}
\ifbook
    \fancypagestyle{plain}{%
        \fancyhead[L]{}
        \ifdate
            \fancyhead[R]{\textsc{\today}}
        \else
            \fancyhead[R]{}
        \fi
        \renewcommand{\headrulewidth}{0pt}
    }

    \let\cleardoublepage=\clearpage
\else
    \fancypagestyle{plain}{}
    \renewcommand{\headrulewidth}{0pt}
\fi

\ifdaggerfootnotes
    \renewcommand{\thefootnote}{\fnsymbol{footnote}}
\else
\fi

\delimitershortfall=-1pt
\normalem

\newlist{detail}{itemize}{2}
\setlist[detail]{label={\boldmath$\cdot$},topsep=0pt,leftmargin=*,noitemsep}

\ifalgorithms
    \newcounter{nalg}[chapter]
    \renewcommand{\thenalg}{\thechapter.\arabic{nalg}}
    \DeclareCaptionLabelFormat{algocaption}{\it Algorithm \thenalg}

    \lstnewenvironment{algorithm}[1][]
    {
        \refstepcounter{nalg}
        \captionsetup{labelformat=algocaption,labelsep=colon}
        \lstset{
            mathescape=true,
            frame=tB,
            numbers=left,
            numberstyle=\tiny,
            basicstyle=\scriptsize,
            keywordstyle=\color{black}\bfseries\em,
            keywords={,input, output, return, datatype, function, in, if, else, elif, for, foreach, while, not, begin, end, true, false, null, break, continue, let, and, or, }
            numbers=left,
            xleftmargin=.04\textwidth,
            #1
        }
    }
    {}
\else
\fi

\ifindentproofs
    % begin new proof environment
    \expandafter\let\expandafter\oldproof\csname\string\proof\endcsname
    \let\oldendproof\endproof

    \renewenvironment{proof}[1][\proofname]{%
        \ifindenttheorems
            \vspace{-\abovedisplayskip}
        \else
        \fi
        \oldproof[#1]\quote~\vspace{-\parskip}

    }{%
        %\endquote\oldendproof
        \endquote\vspace{-\parskip}\qed
    }
    % end new proof environment
\else
\fi

\ifindenttheorems
    \newtheorem{pretheorem}{Theorem}
    \newtheorem{prelemma}{Lemma}
    \newtheorem{preproposition}{Proposition}
    \newtheorem{precorollary}{Corollary}
    \newtheorem{preclaim}{Claim}
    \newtheorem{preconjecture}{Conjecture}
    \newtheorem{prejustification}{Justification}

    \newtheorem{preaxiom}{Axiom}
    \newtheorem{predefinition}{Definition}
    \newtheorem{prenotation}{Notation}
    \newtheorem{preexercise}{Exercise}
    \newtheorem{preexample}{Example}
    \newtheorem{precounterexample}{Counterexample}

    \newtheorem{preidea}{Idea}
    \newtheorem*{preremark}{Remark}
    \newtheorem*{prenote}{Note}

    % theorem
    \NewDocumentEnvironment{theorem}{O{} O{}}
        {\begin{pretheorem}[#1]~#2\quote\vspace{-0.75\parskip}}
        {\endquote\end{pretheorem}}
    % lemma
    \NewDocumentEnvironment{lemma}{O{} O{}}
        {\begin{prelemma}[#1]~#2\quote\vspace{-0.75\parskip}}
        {\endquote\end{prelemma}}
    % proposition
    \NewDocumentEnvironment{proposition}{O{} O{}}
        {\begin{preproposition}[#1]~#2\quote\vspace{-0.75\parskip}}
        {\endquote\end{preproposition}}
    % corollary
    \NewDocumentEnvironment{corollary}{O{} O{}}
        {\begin{precorollary}[#1]~#2\quote\vspace{-0.75\parskip}}
        {\endquote\end{precorollary}}
    % claim
    \NewDocumentEnvironment{claim}{O{} O{}}
        {\begin{preclaim}[#1]~#2\quote\vspace{-0.75\parskip}}
        {\endquote\end{preclaim}}
    % conjecture
    \NewDocumentEnvironment{conjecture}{O{} O{}}
        {\begin{preconjecture}[#1]~#2\quote\vspace{-0.75\parskip}}
        {\endquote\end{preconjecture}}
    % justification
    \NewDocumentEnvironment{justification}{O{} O{}}
        {\begin{prejustification}[#1]~#2\quote\vspace{-0.75\parskip}}
        {\endquote\end{prejustification}}

    % axiom
    \NewDocumentEnvironment{axiom}{O{} O{}}
        {\begin{preaxiom}[#1]~#2\quote\normalfont\vspace{-0.75\parskip}}
        {\endquote\end{preaxiom}}
    % definition
    \NewDocumentEnvironment{definition}{O{} O{}}
        {\begin{predefinition}[#1]~#2\quote\normalfont\vspace{-0.75\parskip}}
        {\endquote\end{predefinition}}
    % notation
    \NewDocumentEnvironment{notation}{O{} O{}}
        {\begin{prenotation}[#1]~#2\quote\normalfont\vspace{-0.75\parskip}}
        {\endquote\end{prenotation}}
    % exercise
    \NewDocumentEnvironment{exercise}{O{} O{}}
        {\begin{preexercise}[#1]~#2\quote\normalfont\vspace{-0.75\parskip}}
        {\endquote\end{preexercise}}
    % example
    \NewDocumentEnvironment{example}{O{} O{}}
        {\begin{preexample}[#1]~#2\quote\normalfont\vspace{-0.75\parskip}}
        {\endquote\end{preexample}}
    % counterexample
    \NewDocumentEnvironment{counterexample}{O{} O{}}
        {\begin{precounterexample}[#1]~#2\quote\normalfont\vspace{-0.75\parskip}}
        {\endquote\end{precounterexample}}

    % idea
    \NewDocumentEnvironment{idea}{O{} O{}}
        {\begin{preidea}[#1]~#2\normalfont}
        {\end{preidea}}
    % remark
    \NewDocumentEnvironment{remark}{O{} O{}}
        {\begin{preremark}[#1]~#2\normalfont}
        {\end{preremark}}
    % note
    \NewDocumentEnvironment{note}{O{} O{}}
        {\begin{prenote}[#1]~#2\normalfont}
        {\end{prenote}}
\else
    \theoremstyle{thm}% style for theorems
    \newtheorem{theorem}{Theorem}
    \newtheorem{lemma}{Lemma}
    \newtheorem{proposition}{Proposition}
    \newtheorem{corollary}{Corollary}
    \newtheorem{claim}{Claim}
    \newtheorem{conjecture}{Conjecture}
    \newtheorem{justification}{Justification}

    \theoremstyle{dfn}% style for definitions
    \newtheorem{axiom}{Axiom}
    \newtheorem{definition}{Definition}
    \newtheorem{notation}{Notation}
    \newtheorem{exercise}{Exercise}
    \newtheorem{example}{Example}
    \newtheorem{counterexample}{Counterexample}

    \theoremstyle{rmk}% style for remarks
    \newtheorem{idea}{Idea}
    \newtheorem*{remark}{Remark}
    \newtheorem*{note}{Note}
\fi

\newenvironment{case}[1][Case]
    {\textbf{#1:}\quote\vspace{-0.75\parskip}}
    {\endquote}

\def\lstlistingautorefname{Algorithm}
\def\itemautorefname{Section}
\renewcommand{\chapterautorefname}{Chapter}
\renewcommand{\sectionautorefname}{Section}
\newcommand{\pretheoremautorefname}{Theorem}
\newcommand{\preaxiomautorefname}{Axiom}
\newcommand{\prelemmaautorefname}{Lemma}
\newcommand{\prepropositionautorefname}{Proposition}
\newcommand{\precorollaryautorefname}{Corollary}
\newcommand{\preclaimautorefname}{Claim}
\newcommand{\preconjectureautorefname}{Conjecture}
\newcommand{\prejustificationautorefname}{Justification}
\newcommand{\predefinitionautorefname}{Definition}
\newcommand{\prenotationautorefname}{Notation}
\newcommand{\preexampleautorefname}{Example}
\newcommand{\precounterexampleautorefname}{Counterexample}
\newcommand{\preideaautorefname}{Idea}
\newcommand{\axiomautorefname}{Axiom}
\newcommand{\lemmaautorefname}{Lemma}
\newcommand{\propositionautorefname}{Proposition}
\newcommand{\corollaryautorefname}{Corollary}
\newcommand{\claimautorefname}{Claim}
\newcommand{\conjectureautorefname}{Conjecture}
\newcommand{\justificationautorefname}{Justification}
\newcommand{\definitionautorefname}{Definition}
\newcommand{\notationautorefname}{Notation}
\newcommand{\exampleautorefname}{Example}
\newcommand{\counterexampleautorefname}{Counterexample}
\newcommand{\ideaautorefname}{Idea}

\ifbook
    \renewcommand{\theequation}{\thechapter.\arabic{equation}}
    \renewcommand{\thepretheorem}{\thechapter.\arabic{pretheorem}}
    \renewcommand{\theprelemma}{\thechapter.\arabic{prelemma}}
    \renewcommand{\thepreproposition}{\thechapter.\arabic{preproposition}}
    \renewcommand{\theprecorollary}{\thechapter.\arabic{precorollary}}
    \renewcommand{\thepreclaim}{\thechapter.\arabic{preclaim}}
    \renewcommand{\thepreconjecture}{\thechapter.\arabic{preconjecture}}
    \renewcommand{\theprejustification}{\thechapter.\arabic{prejustification}}
    \renewcommand{\thepredefinition}{\thechapter.\arabic{predefinition}}
    \renewcommand{\theprenotation}{\thechapter.\arabic{prenotation}}
    \renewcommand{\thepreexample}{\thechapter.\arabic{preexample}}
    \renewcommand{\theprecounterexample}{\thechapter.\arabic{precounterexample}}
    \counterwithin*{equation}{chapter}
    \counterwithin*{pretheorem}{chapter}
    \counterwithin*{prelemma}{chapter}
    \counterwithin*{preproposition}{chapter}
    \counterwithin*{precorollary}{chapter}
    \counterwithin*{preclaim}{chapter}
    \counterwithin*{preconjecture}{chapter}
    \counterwithin*{prejustification}{chapter}
    \counterwithin*{predefinition}{chapter}
    \counterwithin*{prenotation}{chapter}
    \counterwithin*{preexercise}{chapter}
    \counterwithin*{preexample}{chapter}
    \counterwithin*{precounterexample}{chapter}
\else
\fi

\newcommand*{\xline}[1][3em]{\rule[0.5ex]{#1}{0.55pt}}

\newcommand{\isomorphic}{\cong}
\newcommand{\iffdefn}{~\mathrel{\vcentcolon\Leftrightarrow}~}
\newcommand{\iffbydefn}{\(\mathrel{\vcentcolon\Leftrightarrow}\)\ }
\newcommand{\niff}{\mathrel{{\ooalign{\hidewidth$\not\phantom{"}$\hidewidth\cr$\iff$}}}}
\renewcommand{\implies}{\Rightarrow}
\renewcommand{\iff}{\Leftrightarrow}
\newcommand{\proves}{\vdash}
\newcommand{\satisfies}{\models}
\renewcommand{\qedsymbol}{\sc q.e.d.}

\renewcommand{\restriction}[1]{\downharpoonright_{#1}}
\renewcommand{\leq}{\leqslant}
\renewcommand{\geq}{\geqslant}

\newcommand{\meet}{\wedge}
\newcommand{\join}{\vee}
\newcommand{\conjunct}{\wedge}
\newcommand{\disjunct}{\vee}
\newcommand{\bigmeet}{\bigwedge}
\newcommand{\bigjoin}{\bigvee}
\newcommand{\bigconjunct}{\bigwedge}
\newcommand{\bigdisjunct}{\bigvee}
\newcommand{\defn}{\coloneqq}
\newcommand{\xor}{\oplus}
\newcommand{\nand}{\uparrow}
\newcommand{\nor}{\downarrow}

\newcommand{\compose}{\circ}
\newcommand{\divides}{~|~}
\newcommand{\notdivides}{\not|~}
\newcommand{\given}{~\middle|~}
\newcommand{\suchthat}{~\middle|~}
\newcommand{\contradiction}{~\smash{\text{\raisebox{-0.6ex}{\Large \Lightning}}}~}

\newcommand{\conjugate}[1]{\overline{#1}}
\newcommand{\mean}[1]{\overline{#1}}

\newcommand*\diff{\mathop{}\!\mathrm{d}}
\newcommand{\integral}[1]{\smashoperator{\int_{#1}}}
\newcommand{\E}[1]{\mathbb{E}\sq*{#1}}
\newcommand{\Esub}[2]{\mathbb{E}_{#1}\sq*{#2}}
\newcommand{\var}[1]{\mathrm{Var}\pn*{#1}}
\newcommand{\cov}[2]{\mathrm{Cov}\pn*{#1, #2}}
\newcommand{\der}[2]{\frac{\diff{#1}}{\diff{#2}}}
\newcommand{\dern}[3]{\frac{\diff^{#3}{#1}}{\diff{#2}^{#3}}}
\newcommand{\derm}[3]{\frac{\diff^{#3}{#1}}{\diff{#2}}}
\newcommand{\prt}[2]{\frac{\partial{#1}}{\partial{#2}}}
\newcommand{\prtn}[3]{\frac{\partial^{#3}{#1}}{\partial{#2}^{#3}}}
\newcommand{\prtm}[3]{\frac{\partial^{#3}{#1}}{\partial{#2}}}
\newcommand{\modulo}[1]{~\pn{\mathrm{mod}~#1}}

\newcommand{\inj}{\hookrightarrow}
\newcommand{\injection}{\hookrightarrow}

\newcommand{\surj}{\twoheadrightarrow}
\newcommand{\surjection}{\twoheadrightarrow}

\newcommand{\bij}{\lhook\joinrel\twoheadrightarrow}
\newcommand{\bijection}{\lhook\joinrel\twoheadrightarrow}

\newcommand{\monic}{\hookrightarrow}
\newcommand{\monomorphism}{\hookrightarrow}

\newcommand{\epic}{\twoheadrightarrow}
\newcommand{\epimorphism}{\twoheadrightarrow}

\newcommand{\iso}{\lhook\joinrel\twoheadrightarrow}
\newcommand{\isomorphism}{\lhook\joinrel\twoheadrightarrow}
\newcommand{\immersion}{\looprightarrow}

\renewcommand{\O}[1]{\mathcal{O}\pn*{#1}}
\renewcommand{\P}[1]{\mathbb{P}\pn*{#1}}
\newcommand{\power}[1]{\mathcal{P}\pn*{#1}}
\newcommand{\successor}[1]{\mathcal{S}\pn*{#1}}
\newcommand{\C}{\mathbb{C}}
\newcommand{\N}{\mathbb{N}}
\newcommand{\Q}{\mathbb{Q}}
\newcommand{\R}{\mathbb{R}}
\newcommand{\Z}{\mathbb{Z}}

% these don't need {} after them since they should be followed by text
\newcommand{\cf}{\textit{c.f.},\ }
\newcommand{\eg}{\textit{e.g.},\ }
\newcommand{\ie}{\textit{i.e.},\ }
\newcommand{\aka}{\textit{a.k.a.}\ }
\newcommand{\viz}{\textit{viz.}\ }
\newcommand{\vide}{\textit{v.}\ }
\newcommand{\ifandonlyif}{\textit{iff}\ }

% these need {} after them
\newcommand{\etal}{\textit{et al.}}
\newcommand{\wff}{\textit{wff}}

\DeclareMathOperator{\lcm}{lcm}
\DeclareMathOperator*{\argmin}{arg\!\min}
\DeclareMathOperator*{\argmax}{arg\!\max}

\let\originalleft\left
\let\originalright\right
\renewcommand{\left}{\mathopen{}\mathclose\bgroup\originalleft}
\renewcommand{\right}{\aftergroup\egroup\originalright}

\newcommand{\zh}[1]{\begin{CJK}{UTF8}{gbsn}#1\end{CJK}}
\newcommand{\jp}[1]{\begin{CJK}{UTF8}{gbsn}#1\end{CJK}}

\DeclarePairedDelimiterX \inner[2]{\langle}{\rangle}{#1,#2}
\DeclarePairedDelimiter \bra{\langle}{\rvert}
\DeclarePairedDelimiter \ket{\lvert}{\rangle}
\DeclarePairedDelimiter \abs{\lvert}{\rvert}
\DeclarePairedDelimiter \cardinality{\lvert}{\rvert}
\DeclarePairedDelimiter \norm{\lVert}{\rVert}
\DeclarePairedDelimiter \set{\lbrace}{\rbrace}
\DeclarePairedDelimiter \seq{\langle}{\rangle}
\DeclarePairedDelimiter \pn{(}{)}
\DeclarePairedDelimiter \sq{[}{]}
\DeclarePairedDelimiter \curly{\lbrace}{\rbrace}
\DeclarePairedDelimiter \bracket{\langle}{\rangle}

\let\oldemptyset\emptyset
\let\emptyset\varnothing
\let\union\cup
\let\intersection\cap
\let\intersect\cap


\externaldocument{../set-theory/set-theory}

\begin{document}

\title{Discrete Mathematics}
\author{Daniel Gonzalez Cedre}
\date{University of Notre Dame \\ Spring of 2023}
\maketitle

\setcounter{chapter}{3}
\chapter{Mathematical Induction}

\section{Weak Induction}

\begin{definition}[Well-Order]
    Let $X$ be a set with a relation $\leq$ defined on it.
    We say that $X$ is \emph{well-ordered} by this relation \ifandonlyif
    every non-empty subset of $X$ has a minimal element with respect to $\leq$.
    In other words, we say that $\leq$ is a \emph{well-order} on $X$
    \iffbydefn
    \[
        \pn*{\forall A \in \power{X}}\pn*{\exists a \in A}\pn*{\forall b \in A}{a \leq b}.
    \]
\end{definition}

\begin{theorem}[$\N$ is Well-Ordered]\label{thm:lep}
    \vspace{-\parskip}
    \[
        \pn*{\forall A \subseteq \N}\pn*{A \neq \emptyset \implies \pn*{\exists a \in A}\pn*{\forall b \in A}\pn*{a \leq b}}
    \]
    This says that every nonempty subset of $\N$ has a least element
    (according to the $\leq$ order defined on $\N$).
\end{theorem}
\begin{proof}
    This proof is left as an exercise.
\end{proof}

\begin{theorem}[Weak Induction]\label{thm:weakinduction}
    If $\varphi(\cdot)$ is a {\wff}, then
    \[
        \pn*{\forall n \in \N}\pn*{\varphi(n)}
        \iff \varphi(0) \meet \pn*{\forall k \in \N}\pn*{\varphi(k) \implies \varphi(k + 1)}.
    \]
\end{theorem}
\begin{proof}
    There are two fragments to this proof:
    the forward ($\implies$) direction and the backward ($\impliedby$) direction.
    \begin{case}[Fragment 1 ($\implies$)]
        Suppose that $\pn*{\forall n \in \N}\pn*{\varphi(n)}$.
        Since $0 \in \N$, we then obviously have $\varphi(0)$.
        Now, let $k \in \N$ and assume $\varphi(k)$.
        Since $k + 1 \in \N$, we know from our initial assumption that $\varphi(k + 1)$.
        Thus, we have $\pn*{\forall k \in \N}\pn*{\varphi(k) \implies \varphi(k + 1)}$,
        and we have reached both of our desired conclusions.
    \end{case}
    \begin{case}[Fragment 2 ($\impliedby$)]
        Assume $\varphi(0)$ and $\pn*{\forall k \in \N}\pn*{\varphi(k) \implies \varphi(k + 1)}$.
        Towards a contradiction,
        suppose that there is some $n \in \N$ such that $\neg \varphi(a)$.
        Consider $A \defn \set*{x \in \N \suchthat \neg \varphi(x)}$,
        which we clearly know exists by \autoref{ax:separation}.
        We know that $n \in A$ because we assumed that $\neg \varphi(n)$,
        which implies that $A \neq \emptyset$.
        Then, we can use \autoref{thm:lep} to conclude that there is a minimal element $a$ in $A$.

        Since we know that $\varphi(0)$, it follows that $a \neq 0$, so $a$ must be a successor number.
        This means there is a $b \in \N$ such that $b + 1 = a$.

        If $b \in A$, then that would mean that $a \leq b$ since $a$ is minimal in $A$.
        However, we know that $b < b + 1 = a$, so we would then have $a \leq b < a$. \contradiction
        Therefore, $b \not \in A$.
    \end{case}
    With these two directions proven, we finally have
    $\pn*{\forall n \in \N}\pn*{\varphi(n)}
    \iff \varphi(0) \meet \pn*{\forall k \in \N}\pn*{\varphi(k) \implies \varphi(k + 1)}$.
\end{proof}

\begin{note}
The above theorem actually generalizes beyond just $\N$.
In fact, we can generalize the proof of \autoref{thm:weakinduction}
to \emph{any} well-ordered set $X$ by replacing $k + 1$
with the least element of the non-empty subset $X \setminus \set*{\ell \in X \suchthat \ell \leq k}$.
\end{note}

\newpage

Let's practice induction by proving the following few theorems.

\begin{lemma}[Gaussian Summation Formula]\label{thm:gaussiansum}
    $\displaystyle \pn*{\forall n \in \N}\pn*{\sum_{i = 0}^{n}i = \frac{n(n + 1)}{2}}$.
\end{lemma}
\begin{proof}
    We will prove the claim by induction on $n \in \N$.
    \begin{case}[Base Case]
        Observe that $\displaystyle \sum_{i = 0}^{0}i = 0 = \frac{0*(0 + 1)}{2}$.
        Therefore, the statement is satisfied at $0$.
    \end{case}
    \begin{case}[Inductive Step]
        Let $k \in \N$ and assume $\displaystyle \sum_{i = 0}^{k}i = \frac{k(k + 1)}{2}$
        (this is our \emph{inductive hypothesis}).
        Now, observe
        \begin{alignat*}{2}
            \sum_{i = 0}^{k + 1}i &= \pn*{\sum_{i = 0}^{k}i} + (k + 1) &&\text{ by definition}\\
                                  &= \pn*{\frac{k(k + 1)}{2}} + (k + 1)~~~~&&\text{ by the \emph{inductive hypothesis}}\\
                                  &= (k + 1)\pn*{\frac{k}{2} + 1} &&\text{ by the distributive property of multiplication}\footnotemark[1]\footnotemark[2]\\
                                  &= \frac{(k + 1)(k + 2)}{2} &&\text{ because } \frac{k}{2} + 1 = \frac{k}{2} + \frac{2}{2} = \frac{k + 2}{2}.\footnotemark[2]\footnotemark[3]
        \end{alignat*}
        Thus, we have that $\displaystyle \sum_{i = 0}^{k + 1}i = \frac{(k + 1)(k + 2)}{2}$, as desired.
    \end{case}
    Therefore, we can conclude that $\pn*{\forall n \in \N}\pn*{\sum_{i = 0}^{n}i = \frac{n(n + 1)}{2}}$.
\end{proof}

\begin{lemma}
    $3n \leq 3^n$ for all $n \in \N_+$.
\end{lemma}
\begin{proof}
    We will prove the claim by induction on $n \in \N_+$.
    \begin{case}[Base Case]
        Observe that $3 \cdot 0 = 0 \leq 1 = 3^0$.
    \end{case}
    \begin{case}[Inductive Step]
        Let $k \in \N_+$ and assume $3 \cdot k \leq 3^k$
        (this is our \emph{inductive hypothesis}).
        Now, observe
        \begin{alignat*}{2}
            3 \cdot (k + 1) &= 3 \cdot k + 3 &&\text{ by the distributive property of multiplication} \\
                            &\leq 3^k + 3 &&\text{ by the \emph{inductive hypothesis}} \\
                            &\leq 3^k + 3^k &&\text{ because $1 \leq k \implies a^1 \leq a^k$ if $a < 0$}\footnotemark[1]\footnotemark[2] \\
                            &\leq 3^k + 3^k + 3^k~~~~&&\text{ since $a^k$ is always positive if $0 < a$}\footnotemark[1]\footnotemark[2] \\
                            &= 3 \cdot 3^k &&\text{ because $\textstyle \sum_{i = 1}^{m}a = m \cdot a$ for any $a$}\footnotemark[1]\footnotemark[2] \\
                            &= 3^{k + 1} &&\text{ because $a^ba^c = a^{b + c}$ for any $a$}.\footnotemark[1]\footnotemark[2]
        \end{alignat*}
        So, $3 \cdot (k + 1) \leq 3^{k + 1}$, as desired.
    \end{case}
    Therefore, $\pn*{\forall n \in \N}\pn*{3n \leq 3^n}$.
\end{proof}

\footnotetext[1]{These ``basic grade-school'' algebraic properties will now be assumed without special mention.}
\footnotetext[2]{This is true in any \emph{ordered semiring} where exponentiation is defined in terms of multiplication.}
\footnotetext[3]{This is true in any \emph{field}.}

\begin{lemma}
    $\displaystyle \sum_{i = 0}^{n}2^i = 2^{n + 1} - 1$ for all $n \in \N$.
\end{lemma}
\begin{proof}
    We will prove the claim by induction on $n \in \N_+$.
    \begin{case}[Base Case]
        Observe that $\displaystyle \sum_{i = 0}^{0}2^i = 2^0 = 1 = 2 - 1 = 2^{1 + 1} - 1$
    \end{case}
    \begin{case}[Inductive Step]
        Let $k \in \N_+$ and assume that $\displaystyle \sum_{i = 0}^{k}2^i = 2^{k + 1} - 1$
        (this is our \emph{inductive hypothesis}).
        Observe that
        \begin{alignat*}{2}
            \sum_{i = 0}^{k + 1}2^i &= \pn*{\sum_{i = 0}^{k}2^i} + 2^{k + 1} &&\text{ by definition} \\
                                    &= \pn*{2^{k + 1} - 1} + 2^{k + 1}~~~~&&\text{ by the \emph{inductive hypothesis}} \\
                                    &= 2 \cdot 2^{k + 1} - 1 &&\text{ using some basic algebra} \\
                                    &= 2^{k + 2} - 1 &&\text{ using some basic algebra}.
        \end{alignat*}
        So, we get $\displaystyle \sum_{i = 0}^{k + 1}2^i = 2^{k + 1} - 1$, as desired.
    \end{case}
    Therefore, we can conclude $\pn*{\forall n \in \N}\pn*{\sum_{i = 0}^{n}2^i = 2^{n + 1} - 1}$.
\end{proof}

\section{Strong Induction}
\begin{theorem}[Strong Induction]\label{thm:stronginduction}
    If $\varphi(\cdot)$ is a {\wff} and $r \in \N$, then
    \[
        \pn*{\forall n \in \N}\pn*{\varphi(n)}
        \iff \pn*{\bigmeet_{i = 0}^{r}\varphi(r)}
        \meet \pn*{\forall k \in \N}\pn*{\pn*{\forall \ell \leq k}\pn*{\varphi(\ell)} \implies \varphi(k + 1)}.
    \]
\end{theorem}

\begin{lemma}
    $\pn*{\forall n \in \N}\pn*{\exists r \in \N}\pn*{\exists c_0, \dots c_r \in \set*{0, 1}}\pn*{n = \sum_{i = 0}^{r}c_i 2^i}$;
    \ie every natural number admits a binary representation.
\end{lemma}
\begin{proof}
    We will prove the claim by using \autoref{thm:stronginduction}.
\end{proof}

\end{document}
