% define new conditionals: \if<conditional>
\newif\ifbook  % whether or not to use book styles
\newif\ifdate  % whether or not to include the current date header
\newif\ifalgorithms  % whether or not to define the algorithm environment
\newif\ifdaggerfootnotes  % whether or not to have fnsymbol footnotes or numerical ones

% set the conditionals: \<conditional>true / \<conditional>false
\bookfalse
\datefalse
\algorithmsfalse
\daggerfootnotestrue

\ifbook
    \documentclass[letterpaper]{book}
    \usepackage{arydshln, chngcntr}
\else
    \documentclass[letterpaper]{article}
\fi

% math
\usepackage{amsmath, amsfonts, amssymb, amstext, amscd, amsthm, mathrsfs, mathtools, xfrac}
% fonts
\usepackage{bbm, CJKutf8, caption, dsfont, marvosym, stmaryrd}
% tables
\usepackage{booktabs, colortbl, makecell}
% colors
\usepackage{color, soul, xcolor}
% references
\usepackage{xr-hyper, hyperref, url}
% figures
\usepackage{graphicx, float, subcaption, tikz}
% headers and footers
\usepackage{fancyhdr, lastpage}
% miscellaneous
\usepackage{enumerate, ifthen, lipsum, listings, makeidx, parskip, ulem, verbatim, xargs}
\usepackage[nodayofweek]{datetime}

\usepackage[left=2cm,top=2cm,right=2cm,bottom=2cm,bindingoffset=0cm]{geometry}
\usepackage[group-separator={,},group-minimum-digits={3}]{siunitx}
\usepackage[shortlabels]{enumitem}
\setlist[enumerate]{topsep=0ex,itemsep=0ex,partopsep=1ex,parsep=1ex}
\setlist[itemize]{topsep=0ex,itemsep=0ex,partopsep=1ex,parsep=1ex}

\usepackage[math]{cellspace}
\cellspacetoplimit 1pt
\cellspacebottomlimit 1pt

\definecolor{gruvred}{HTML}{CC214D}
\definecolor{gruvorange}{HTML}{D65D0E}
\definecolor{gruvaqua}{HTML}{689D6A}
\definecolor{gruvpurple}{HTML}{B16286}
\definecolor{colorblack}{HTML}{252422}
\definecolor{colorgrey}{HTML}{f4efef}
\definecolor{colorblue}{HTML}{045275}
\definecolor{colorteal}{HTML}{089099}
\definecolor{colorgreen}{HTML}{7ccba2}
\definecolor{coloryellow}{HTML}{ffc61e}  % fcde9c % ffc61e  % b8860b
\definecolor{colororange}{HTML}{f0746e}
\definecolor{colorred}{HTML}{dc3977}
\definecolor{colorpurple}{HTML}{7c1d6f}

\hypersetup{
    colorlinks=true,
    linkcolor=gruvorange,
    citecolor=gruvaqua,
    urlcolor=gruvpurple
}

\allowdisplaybreaks
\newdateformat{verbosedate}{\ordinal{DAY} of \monthname[\THEMONTH], \THEYEAR}
\verbosedate

\pagestyle{fancy}
% \fancyfoot[C]{--~\thepage~--}
\fancyfoot[C]{\tiny \thepage\ / \pageref*{LastPage}}
\ifbook
    \fancypagestyle{plain}{%
        \fancyhead[L]{}
        \ifdate
            \fancyhead[R]{\textsc{\today}}
        \else
            \fancyhead[R]{}
        \fi
        \renewcommand{\headrulewidth}{0pt}
    }

    \let\cleardoublepage=\clearpage
\else
    \fancypagestyle{plain}{}
    \renewcommand{\headrulewidth}{0pt}
\fi

\ifdaggerfootnotes
    \renewcommand{\thefootnote}{\fnsymbol{footnote}}
\else
\fi

\delimitershortfall=-1pt
\normalem

\newlist{detail}{itemize}{2}
\setlist[detail]{label={\boldmath$\cdot$},topsep=0pt,leftmargin=*,noitemsep}

\ifalgorithms
    \newcounter{nalg}[chapter]
    \renewcommand{\thenalg}{\thechapter.\arabic{nalg}}
    \DeclareCaptionLabelFormat{algocaption}{\it Algorithm \thenalg}

    \lstnewenvironment{algorithm}[1][]
    {
        \refstepcounter{nalg}
        \captionsetup{labelformat=algocaption,labelsep=colon}
        \lstset{
            mathescape=true,
            frame=tB,
            numbers=left,
            numberstyle=\tiny,
            basicstyle=\scriptsize,
            keywordstyle=\color{black}\bfseries\em,
            keywords={,input, output, return, datatype, function, in, if, else, elif, for, foreach, while, not, begin, end, true, false, null, break, continue, let, and, or, }
            numbers=left,
            xleftmargin=.04\textwidth,
            #1
        }
    }
    {}
\else
\fi

\ifindentproofs
    % begin new proof environment
    \expandafter\let\expandafter\oldproof\csname\string\proof\endcsname
    \let\oldendproof\endproof

    \renewenvironment{proof}[1][\proofname]{%
        \ifindenttheorems
            \vspace{-\abovedisplayskip}
        \else
        \fi
        \oldproof[#1]\quote~\vspace{-\parskip}

    }{%
        %\endquote\oldendproof
        \endquote\vspace{-\parskip}\qed
    }
    % end new proof environment
\else
\fi

\ifindenttheorems
    \newtheorem{pretheorem}{Theorem}
    \newtheorem{prelemma}{Lemma}
    \newtheorem{preproposition}{Proposition}
    \newtheorem{precorollary}{Corollary}
    \newtheorem{preclaim}{Claim}
    \newtheorem{preconjecture}{Conjecture}
    \newtheorem{prejustification}{Justification}

    \newtheorem{preaxiom}{Axiom}
    \newtheorem{predefinition}{Definition}
    \newtheorem{prenotation}{Notation}
    \newtheorem{preexercise}{Exercise}
    \newtheorem{preexample}{Example}
    \newtheorem{precounterexample}{Counterexample}

    \newtheorem{preidea}{Idea}
    \newtheorem*{preremark}{Remark}
    \newtheorem*{prenote}{Note}

    % theorem
    \NewDocumentEnvironment{theorem}{O{} O{}}
        {\begin{pretheorem}[#1]~#2\quote\vspace{-0.75\parskip}}
        {\endquote\end{pretheorem}}
    % lemma
    \NewDocumentEnvironment{lemma}{O{} O{}}
        {\begin{prelemma}[#1]~#2\quote\vspace{-0.75\parskip}}
        {\endquote\end{prelemma}}
    % proposition
    \NewDocumentEnvironment{proposition}{O{} O{}}
        {\begin{preproposition}[#1]~#2\quote\vspace{-0.75\parskip}}
        {\endquote\end{preproposition}}
    % corollary
    \NewDocumentEnvironment{corollary}{O{} O{}}
        {\begin{precorollary}[#1]~#2\quote\vspace{-0.75\parskip}}
        {\endquote\end{precorollary}}
    % claim
    \NewDocumentEnvironment{claim}{O{} O{}}
        {\begin{preclaim}[#1]~#2\quote\vspace{-0.75\parskip}}
        {\endquote\end{preclaim}}
    % conjecture
    \NewDocumentEnvironment{conjecture}{O{} O{}}
        {\begin{preconjecture}[#1]~#2\quote\vspace{-0.75\parskip}}
        {\endquote\end{preconjecture}}
    % justification
    \NewDocumentEnvironment{justification}{O{} O{}}
        {\begin{prejustification}[#1]~#2\quote\vspace{-0.75\parskip}}
        {\endquote\end{prejustification}}

    % axiom
    \NewDocumentEnvironment{axiom}{O{} O{}}
        {\begin{preaxiom}[#1]~#2\quote\normalfont\vspace{-0.75\parskip}}
        {\endquote\end{preaxiom}}
    % definition
    \NewDocumentEnvironment{definition}{O{} O{}}
        {\begin{predefinition}[#1]~#2\quote\normalfont\vspace{-0.75\parskip}}
        {\endquote\end{predefinition}}
    % notation
    \NewDocumentEnvironment{notation}{O{} O{}}
        {\begin{prenotation}[#1]~#2\quote\normalfont\vspace{-0.75\parskip}}
        {\endquote\end{prenotation}}
    % exercise
    \NewDocumentEnvironment{exercise}{O{} O{}}
        {\begin{preexercise}[#1]~#2\quote\normalfont\vspace{-0.75\parskip}}
        {\endquote\end{preexercise}}
    % example
    \NewDocumentEnvironment{example}{O{} O{}}
        {\begin{preexample}[#1]~#2\quote\normalfont\vspace{-0.75\parskip}}
        {\endquote\end{preexample}}
    % counterexample
    \NewDocumentEnvironment{counterexample}{O{} O{}}
        {\begin{precounterexample}[#1]~#2\quote\normalfont\vspace{-0.75\parskip}}
        {\endquote\end{precounterexample}}

    % idea
    \NewDocumentEnvironment{idea}{O{} O{}}
        {\begin{preidea}[#1]~#2\normalfont}
        {\end{preidea}}
    % remark
    \NewDocumentEnvironment{remark}{O{} O{}}
        {\begin{preremark}[#1]~#2\normalfont}
        {\end{preremark}}
    % note
    \NewDocumentEnvironment{note}{O{} O{}}
        {\begin{prenote}[#1]~#2\normalfont}
        {\end{prenote}}
\else
    \theoremstyle{thm}% style for theorems
    \newtheorem{theorem}{Theorem}
    \newtheorem{lemma}{Lemma}
    \newtheorem{proposition}{Proposition}
    \newtheorem{corollary}{Corollary}
    \newtheorem{claim}{Claim}
    \newtheorem{conjecture}{Conjecture}
    \newtheorem{justification}{Justification}

    \theoremstyle{dfn}% style for definitions
    \newtheorem{axiom}{Axiom}
    \newtheorem{definition}{Definition}
    \newtheorem{notation}{Notation}
    \newtheorem{exercise}{Exercise}
    \newtheorem{example}{Example}
    \newtheorem{counterexample}{Counterexample}

    \theoremstyle{rmk}% style for remarks
    \newtheorem{idea}{Idea}
    \newtheorem*{remark}{Remark}
    \newtheorem*{note}{Note}
\fi

\newenvironment{case}[1][Case]
    {\textbf{#1:}\quote\vspace{-0.75\parskip}}
    {\endquote}

\def\lstlistingautorefname{Algorithm}
\def\itemautorefname{Section}
\renewcommand{\chapterautorefname}{Chapter}
\renewcommand{\sectionautorefname}{Section}
\newcommand{\pretheoremautorefname}{Theorem}
\newcommand{\preaxiomautorefname}{Axiom}
\newcommand{\prelemmaautorefname}{Lemma}
\newcommand{\prepropositionautorefname}{Proposition}
\newcommand{\precorollaryautorefname}{Corollary}
\newcommand{\preclaimautorefname}{Claim}
\newcommand{\preconjectureautorefname}{Conjecture}
\newcommand{\prejustificationautorefname}{Justification}
\newcommand{\predefinitionautorefname}{Definition}
\newcommand{\prenotationautorefname}{Notation}
\newcommand{\preexampleautorefname}{Example}
\newcommand{\precounterexampleautorefname}{Counterexample}
\newcommand{\preideaautorefname}{Idea}
\newcommand{\axiomautorefname}{Axiom}
\newcommand{\lemmaautorefname}{Lemma}
\newcommand{\propositionautorefname}{Proposition}
\newcommand{\corollaryautorefname}{Corollary}
\newcommand{\claimautorefname}{Claim}
\newcommand{\conjectureautorefname}{Conjecture}
\newcommand{\justificationautorefname}{Justification}
\newcommand{\definitionautorefname}{Definition}
\newcommand{\notationautorefname}{Notation}
\newcommand{\exampleautorefname}{Example}
\newcommand{\counterexampleautorefname}{Counterexample}
\newcommand{\ideaautorefname}{Idea}

\ifbook
    \renewcommand{\theequation}{\thechapter.\arabic{equation}}
    \renewcommand{\thepretheorem}{\thechapter.\arabic{pretheorem}}
    \renewcommand{\theprelemma}{\thechapter.\arabic{prelemma}}
    \renewcommand{\thepreproposition}{\thechapter.\arabic{preproposition}}
    \renewcommand{\theprecorollary}{\thechapter.\arabic{precorollary}}
    \renewcommand{\thepreclaim}{\thechapter.\arabic{preclaim}}
    \renewcommand{\thepreconjecture}{\thechapter.\arabic{preconjecture}}
    \renewcommand{\theprejustification}{\thechapter.\arabic{prejustification}}
    \renewcommand{\thepredefinition}{\thechapter.\arabic{predefinition}}
    \renewcommand{\theprenotation}{\thechapter.\arabic{prenotation}}
    \renewcommand{\thepreexample}{\thechapter.\arabic{preexample}}
    \renewcommand{\theprecounterexample}{\thechapter.\arabic{precounterexample}}
    \counterwithin*{equation}{chapter}
    \counterwithin*{pretheorem}{chapter}
    \counterwithin*{prelemma}{chapter}
    \counterwithin*{preproposition}{chapter}
    \counterwithin*{precorollary}{chapter}
    \counterwithin*{preclaim}{chapter}
    \counterwithin*{preconjecture}{chapter}
    \counterwithin*{prejustification}{chapter}
    \counterwithin*{predefinition}{chapter}
    \counterwithin*{prenotation}{chapter}
    \counterwithin*{preexercise}{chapter}
    \counterwithin*{preexample}{chapter}
    \counterwithin*{precounterexample}{chapter}
\else
\fi

\newcommand*{\xline}[1][3em]{\rule[0.5ex]{#1}{0.55pt}}

\newcommand{\isomorphic}{\cong}
\newcommand{\iffdefn}{~\mathrel{\vcentcolon\Leftrightarrow}~}
\newcommand{\iffbydefn}{\(\mathrel{\vcentcolon\Leftrightarrow}\)\ }
\newcommand{\niff}{\mathrel{{\ooalign{\hidewidth$\not\phantom{"}$\hidewidth\cr$\iff$}}}}
\renewcommand{\implies}{\Rightarrow}
\renewcommand{\iff}{\Leftrightarrow}
\newcommand{\proves}{\vdash}
\newcommand{\satisfies}{\models}
\renewcommand{\qedsymbol}{\sc q.e.d.}

\renewcommand{\restriction}[1]{\downharpoonright_{#1}}
\renewcommand{\leq}{\leqslant}
\renewcommand{\geq}{\geqslant}

\newcommand{\meet}{\wedge}
\newcommand{\join}{\vee}
\newcommand{\conjunct}{\wedge}
\newcommand{\disjunct}{\vee}
\newcommand{\bigmeet}{\bigwedge}
\newcommand{\bigjoin}{\bigvee}
\newcommand{\bigconjunct}{\bigwedge}
\newcommand{\bigdisjunct}{\bigvee}
\newcommand{\defn}{\coloneqq}
\newcommand{\xor}{\oplus}
\newcommand{\nand}{\uparrow}
\newcommand{\nor}{\downarrow}

\newcommand{\compose}{\circ}
\newcommand{\divides}{~|~}
\newcommand{\notdivides}{\not|~}
\newcommand{\given}{~\middle|~}
\newcommand{\suchthat}{~\middle|~}
\newcommand{\contradiction}{~\smash{\text{\raisebox{-0.6ex}{\Large \Lightning}}}~}

\newcommand{\conjugate}[1]{\overline{#1}}
\newcommand{\mean}[1]{\overline{#1}}

\newcommand*\diff{\mathop{}\!\mathrm{d}}
\newcommand{\integral}[1]{\smashoperator{\int_{#1}}}
\newcommand{\E}[1]{\mathbb{E}\sq*{#1}}
\newcommand{\Esub}[2]{\mathbb{E}_{#1}\sq*{#2}}
\newcommand{\var}[1]{\mathrm{Var}\pn*{#1}}
\newcommand{\cov}[2]{\mathrm{Cov}\pn*{#1, #2}}
\newcommand{\der}[2]{\frac{\diff{#1}}{\diff{#2}}}
\newcommand{\dern}[3]{\frac{\diff^{#3}{#1}}{\diff{#2}^{#3}}}
\newcommand{\derm}[3]{\frac{\diff^{#3}{#1}}{\diff{#2}}}
\newcommand{\prt}[2]{\frac{\partial{#1}}{\partial{#2}}}
\newcommand{\prtn}[3]{\frac{\partial^{#3}{#1}}{\partial{#2}^{#3}}}
\newcommand{\prtm}[3]{\frac{\partial^{#3}{#1}}{\partial{#2}}}
\newcommand{\modulo}[1]{~\pn{\mathrm{mod}~#1}}

\newcommand{\inj}{\hookrightarrow}
\newcommand{\injection}{\hookrightarrow}

\newcommand{\surj}{\twoheadrightarrow}
\newcommand{\surjection}{\twoheadrightarrow}

\newcommand{\bij}{\lhook\joinrel\twoheadrightarrow}
\newcommand{\bijection}{\lhook\joinrel\twoheadrightarrow}

\newcommand{\monic}{\hookrightarrow}
\newcommand{\monomorphism}{\hookrightarrow}

\newcommand{\epic}{\twoheadrightarrow}
\newcommand{\epimorphism}{\twoheadrightarrow}

\newcommand{\iso}{\lhook\joinrel\twoheadrightarrow}
\newcommand{\isomorphism}{\lhook\joinrel\twoheadrightarrow}
\newcommand{\immersion}{\looprightarrow}

\renewcommand{\O}[1]{\mathcal{O}\pn*{#1}}
\renewcommand{\P}[1]{\mathbb{P}\pn*{#1}}
\newcommand{\power}[1]{\mathcal{P}\pn*{#1}}
\newcommand{\successor}[1]{\mathcal{S}\pn*{#1}}
\newcommand{\C}{\mathbb{C}}
\newcommand{\N}{\mathbb{N}}
\newcommand{\Q}{\mathbb{Q}}
\newcommand{\R}{\mathbb{R}}
\newcommand{\Z}{\mathbb{Z}}

% these don't need {} after them since they should be followed by text
\newcommand{\cf}{\textit{c.f.},\ }
\newcommand{\eg}{\textit{e.g.},\ }
\newcommand{\ie}{\textit{i.e.},\ }
\newcommand{\aka}{\textit{a.k.a.}\ }
\newcommand{\viz}{\textit{viz.}\ }
\newcommand{\vide}{\textit{v.}\ }
\newcommand{\ifandonlyif}{\textit{iff}\ }

% these need {} after them
\newcommand{\etal}{\textit{et al.}}
\newcommand{\wff}{\textit{wff}}

\DeclareMathOperator{\lcm}{lcm}
\DeclareMathOperator*{\argmin}{arg\!\min}
\DeclareMathOperator*{\argmax}{arg\!\max}

\let\originalleft\left
\let\originalright\right
\renewcommand{\left}{\mathopen{}\mathclose\bgroup\originalleft}
\renewcommand{\right}{\aftergroup\egroup\originalright}

\newcommand{\zh}[1]{\begin{CJK}{UTF8}{gbsn}#1\end{CJK}}
\newcommand{\jp}[1]{\begin{CJK}{UTF8}{gbsn}#1\end{CJK}}

\DeclarePairedDelimiterX \inner[2]{\langle}{\rangle}{#1,#2}
\DeclarePairedDelimiter \bra{\langle}{\rvert}
\DeclarePairedDelimiter \ket{\lvert}{\rangle}
\DeclarePairedDelimiter \abs{\lvert}{\rvert}
\DeclarePairedDelimiter \cardinality{\lvert}{\rvert}
\DeclarePairedDelimiter \norm{\lVert}{\rVert}
\DeclarePairedDelimiter \set{\lbrace}{\rbrace}
\DeclarePairedDelimiter \seq{\langle}{\rangle}
\DeclarePairedDelimiter \pn{(}{)}
\DeclarePairedDelimiter \sq{[}{]}
\DeclarePairedDelimiter \curly{\lbrace}{\rbrace}
\DeclarePairedDelimiter \bracket{\langle}{\rangle}

\let\oldemptyset\emptyset
\let\emptyset\varnothing
\let\union\cup
\let\intersection\cap
\let\intersect\cap

\usepackage{tikz}
\usetikzlibrary{math}
\tikzmath{%
    \pegone = 0.0;
    \pegtwo = 10.0;
    \pegthree = 20.0;
}
\definecolor{plotblue}{HTML}{045275}
\definecolor{plotteal}{HTML}{089099}
\definecolor{plotgreen}{HTML}{7ccba2}
\definecolor{plotyellow}{HTML}{ffc61e}  % fcde9c % ffc61e  % b8860b
\definecolor{plotorange}{HTML}{f0746e}
\definecolor{plotred}{HTML}{dc3977}
\definecolor{plotpurple}{HTML}{7c1d6f}

\begin{document}
\begin{center}
    \textsc{\huge Problem Set 6}\\
    \textsc{Discrete Mathematics}\\
    \color{gruvred}{Due: $29$\textsuperscript{th} of March, $2023$}
\end{center}

\begin{enumerate}
    \item
        Suppose that we are given a sorted list $L$ of length $n \in \N$
        and we are asked to determine
        whether or not $\pn*{\exists i \in n}\pn*{L(i) = x}$.
        The \emph{binary search} algorithm solves this problem by comparing the middle element $L\pn*{\sfrac{n}{2}}$ of the list to $x$ and either returning immediately (if they are equal)
        or recursively searching the appropriate sublist (if they are unequal).
        \begin{enumerate}
            \item
                Provide a \emph{recursive} implementation in \texttt{Python} of \emph{binary search}.
                Name your function \texttt{ps06pr1a}.
                \begin{itemize}
                    \item[$\cdot$]
                        \textbf{Input:}
                        a list $L: n \to \Z$ and an integer $x \in \Z$.
                    \item[$\cdot$]
                        \textbf{Output:}
                        \texttt{True} if $\pn*{\exists i}\pn*{L(i) = x}$; \texttt{False} otherwise.
                    \item[$\cdot$]
                        \textbf{Constraints:}
                        if \texttt{len($L$) == 0}, your function should make $0$ comparisons;
                        if \texttt{len($L$) == 1}, your function should make $1$ comparison;
                        otherwise, your function should make $2$ comparisons.
                \end{itemize}
            \item
                Find a recurrence relation for the number of \emph{comparisons} your function makes.
            \item
                Prove that your recurrence relation has the closed form $T(n) = 2\log_2(n) + 1$.
        \end{enumerate}
    \item
        In this problem,
        we want an efficient way of recursively \emph{merging} two sorted lists into one sorted list.
        \begin{enumerate}
            \item
                Provide a \emph{recursive} implementation in \texttt{Python} of \emph{merge}.
                Name your function \texttt{ps06pr2a}.
                \begin{itemize}
                    \item[$\cdot$]
                        \textbf{Input:}
                        a sorted list $L_1: n_1 \to \Z$ and a sorted list $L_2: n_2 \to \Z$.
                    \item[$\cdot$]
                        \textbf{Output:}
                        a sorted list $L: (n_1 + n_2) \to \Z$ containing all of the elements of $L_1$ and $L_2$.
                    \item[$\cdot$]
                        \textbf{Constraints:}
                        if the length of either input list is $0$, your function can return immediately;
                        otherwise, your function should make $1$ comparison.
                \end{itemize}
            \item
                Find a recurrence relation for the number of \emph{comparisons} your function makes.\\
                \emph{Hint:} your recurrence should be a function of \emph{one} variable,
                which is the \emph{size of the problem}.
            \item
                Prove that your recurrence relation has the closed form $T(n) = n$.
        \end{enumerate}
    \item
        The towers of Hanoi are an arrangement of three wooden pegs,
        labelled \emph{start}, \emph{middle}, and \emph{end},
        along with a collection of $n$ rings of distinct sizes.
        The rings are all stacked on the \emph{start} peg in ascending order based on their sizes.
        The goal is to move all of the rings from the \emph{start} peg to the \emph{end} peg
        without violating the following constraints:
        \begin{itemize}
            \item[$\cdot$]
                A \emph{move} consists of moving the top-most ring from one peg
                to the top of the stack on another peg.
            \item[$\cdot$]
                A larger ring can never be stacked on top of a smaller ring.
            \item[$\cdot$]
                Rings can only be moved one-at-a-time.
        \end{itemize}
        The question is: what is the minimum number of moves required to win the game with $n \in \N_+$ rings?
        \begin{figure}[H]
            \centering
            \begin{subfigure}{0.25\linewidth}
                \centering
                \begin{tikzpicture}[scale=0.1]
                    \draw[draw=plotyellow,fill=plotyellow] (0.0 + \pegone,0.0) rectangle ++(9.0,1.0);
                    \draw[draw=plotyellow,fill=plotyellow] (4.0 + \pegone,0.0) rectangle ++(1.0,8.0);
                    \draw[draw=plotpurple,fill=plotpurple] (1 + \pegone,1.5) rectangle ++(7.0,1.0);
                    \draw[draw=plotblue,fill=plotblue] (2 + \pegone,3) rectangle ++(5.0,1.0);
                    \draw[draw=plotteal,fill=plotteal] (3 + \pegone,4.5) rectangle ++(3.0,1.0);
                    \draw[draw=plotyellow,fill=plotyellow] (0.0 + \pegtwo,0.0) rectangle ++(9.0,1.0);
                    \draw[draw=plotyellow,fill=plotyellow] (4.0 + \pegtwo,0.0) rectangle ++(1.0,8.0);
                    \draw[draw=plotyellow,fill=plotyellow] (0.0 + \pegthree,0.0) rectangle ++(9.0,1.0);
                    \draw[draw=plotyellow,fill=plotyellow] (4.0 + \pegthree,0.0) rectangle ++(1.0,8.0);
                \end{tikzpicture}
                \caption{Initial configuration.}
            \end{subfigure}%
            \begin{subfigure}{0.25\linewidth}
                \centering
                \begin{tikzpicture}[scale=0.1]
                    \draw[draw=plotyellow,fill=plotyellow] (0.0 + \pegone,0.0) rectangle ++(9.0,1.0);
                    \draw[draw=plotyellow,fill=plotyellow] (4.0 + \pegone,0.0) rectangle ++(1.0,8.0);
                    \draw[draw=plotpurple,fill=plotpurple] (1 + \pegone,1.5) rectangle ++(7.0,1.0);
                    \draw[draw=plotyellow,fill=plotyellow] (0.0 + \pegtwo,0.0) rectangle ++(9.0,1.0);
                    \draw[draw=plotyellow,fill=plotyellow] (4.0 + \pegtwo,0.0) rectangle ++(1.0,8.0);
                    \draw[draw=plotblue,fill=plotblue] (2 + \pegtwo,1.5) rectangle ++(5.0,1.0);
                    \draw[draw=plotyellow,fill=plotyellow] (0.0 + \pegthree,0.0) rectangle ++(9.0,1.0);
                    \draw[draw=plotyellow,fill=plotyellow] (4.0 + \pegthree,0.0) rectangle ++(1.0,8.0);
                    \draw[draw=plotteal,fill=plotteal] (3 + \pegthree,1.5) rectangle ++(3.0,1.0);
                \end{tikzpicture}
                \caption{After two moves.}
            \end{subfigure}%
            \begin{subfigure}{0.25\linewidth}
                \centering
                \begin{tikzpicture}[scale=0.1]
                    \draw[draw=plotyellow,fill=plotyellow] (0.0 + \pegone,0.0) rectangle ++(9.0,1.0);
                    \draw[draw=plotyellow,fill=plotyellow] (4.0 + \pegone,0.0) rectangle ++(1.0,8.0);
                    \draw[draw=plotyellow,fill=plotyellow] (0.0 + \pegtwo,0.0) rectangle ++(9.0,1.0);
                    \draw[draw=plotyellow,fill=plotyellow] (4.0 + \pegtwo,0.0) rectangle ++(1.0,8.0);
                    \draw[draw=plotyellow,fill=plotyellow] (0.0 + \pegthree,0.0) rectangle ++(9.0,1.0);
                    \draw[draw=plotyellow,fill=plotyellow] (4.0 + \pegthree,0.0) rectangle ++(1.0,8.0);
                    \draw[draw=plotpurple,fill=plotpurple] (1 + \pegthree,1.5) rectangle ++(7.0,1.0);
                    \draw[draw=plotblue,fill=plotblue] (2 + \pegthree,3) rectangle ++(5.0,1.0);
                    \draw[draw=plotteal,fill=plotteal] (3 + \pegthree,4.5) rectangle ++(3.0,1.0);
                \end{tikzpicture}
                \caption{Final configuration.}
            \end{subfigure}
        \end{figure}
        \begin{enumerate}
            \item
                Provide a \emph{recursive} implementation in \texttt{Python} of \emph{towers of Hanoi}.
                Name your function \texttt{ps06pr3a}.
                \begin{itemize}
                    \item[$\cdot$]
                        \textbf{Input:}
                        a positive natural number $n \in \N_+$ representing the number of rings.
                    \item[$\cdot$]
                        \textbf{Output:}
                        a number $n \in \N_+$ denoting the minimum number of moves required to win the game.
                    \item[$\cdot$]
                        \textbf{Constraints:}
                        none.
                \end{itemize}
            \item
                Find a recurrence relation for the
                \emph{minimum number of moves required to win the game} with $n$ rings.
            \item
                Prove that your recurrence relation has the closed form $T(n) = 2^n - 1$.\\
                \emph{Hint:} recall that $\sum_{i = 0}^{n}2^i = 2^{n + 1} - 1$ for all $n \in \N$.
        \end{enumerate}
\end{enumerate}

\begin{remark}
    Formally, a \emph{list} of length $n \in \N$ with elements from $A$ is just a function $L: n \to A$.
    The $k\textsuperscript{th}$ term in the list is given by $L(k)$ for $k \in \set*{0, \dots n - 1}$,
    so this corresponds to $0$-indexed arrays when programming.
\end{remark}

\textbf{Code Submission Instructions:}
\begin{quote}
    \vspace{-\parskip}
    Several of the problems in this problem set have a programming component.
    The \texttt{Python} functions you define
    must be named as the instructions for each problem indicate,
    and they \emph{must be recursive}.
    You are not permitted to use any internal or external libraries
    (\ie no \texttt{import <...>} statements).
    Your functions should all be implemented in one file,
    with the filename \texttt{ps06-<lastname>-<firstname>.py};
    for example, a possible file name would be \texttt{ps06-gonzalez-cedre-daniel.py}.

    If you are submitting the rest of your solutions to this problem set electronically,
    then attach your \texttt{Python} file \emph{in the same email}
    as the rest of your solutions.

    If you are submitting your proofs in-person on paper, then email your code separately.
\end{quote}
\end{document}
