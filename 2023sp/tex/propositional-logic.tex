% define new conditionals: \if<conditional>
\newif\ifbook  % whether or not to use book styles
\newif\ifdate  % whether or not to include the current date header
\newif\ifalgorithms  % whether or not to define the algorithm environment
\newif\ifdaggerfootnotes  % whether or not to have fnsymbol footnotes or numerical ones

% set the conditionals: \<conditional>true / \<conditional>false
\bookfalse
\datefalse
\algorithmsfalse
\daggerfootnotestrue

\ifbook
    \documentclass[letterpaper]{book}
    \usepackage{arydshln, chngcntr}
\else
    \documentclass[letterpaper]{article}
\fi

% math
\usepackage{amsmath, amsfonts, amssymb, amstext, amscd, amsthm, mathrsfs, mathtools, xfrac}
% fonts
\usepackage{bbm, CJKutf8, caption, dsfont, marvosym, stmaryrd}
% tables
\usepackage{booktabs, colortbl, makecell}
% colors
\usepackage{color, soul, xcolor}
% references
\usepackage{xr-hyper, hyperref, url}
% figures
\usepackage{graphicx, float, subcaption, tikz}
% headers and footers
\usepackage{fancyhdr, lastpage}
% miscellaneous
\usepackage{enumerate, ifthen, lipsum, listings, makeidx, parskip, ulem, verbatim, xargs}
\usepackage[nodayofweek]{datetime}

\usepackage[left=2cm,top=2cm,right=2cm,bottom=2cm,bindingoffset=0cm]{geometry}
\usepackage[group-separator={,},group-minimum-digits={3}]{siunitx}
\usepackage[shortlabels]{enumitem}
\setlist[enumerate]{topsep=0ex,itemsep=0ex,partopsep=1ex,parsep=1ex}
\setlist[itemize]{topsep=0ex,itemsep=0ex,partopsep=1ex,parsep=1ex}

\usepackage[math]{cellspace}
\cellspacetoplimit 1pt
\cellspacebottomlimit 1pt

\definecolor{gruvred}{HTML}{CC214D}
\definecolor{gruvorange}{HTML}{D65D0E}
\definecolor{gruvaqua}{HTML}{689D6A}
\definecolor{gruvpurple}{HTML}{B16286}
\definecolor{colorblack}{HTML}{252422}
\definecolor{colorgrey}{HTML}{f4efef}
\definecolor{colorblue}{HTML}{045275}
\definecolor{colorteal}{HTML}{089099}
\definecolor{colorgreen}{HTML}{7ccba2}
\definecolor{coloryellow}{HTML}{ffc61e}  % fcde9c % ffc61e  % b8860b
\definecolor{colororange}{HTML}{f0746e}
\definecolor{colorred}{HTML}{dc3977}
\definecolor{colorpurple}{HTML}{7c1d6f}

\hypersetup{
    colorlinks=true,
    linkcolor=gruvorange,
    citecolor=gruvaqua,
    urlcolor=gruvpurple
}

\allowdisplaybreaks
\newdateformat{verbosedate}{\ordinal{DAY} of \monthname[\THEMONTH], \THEYEAR}
\verbosedate

\pagestyle{fancy}
% \fancyfoot[C]{--~\thepage~--}
\fancyfoot[C]{\tiny \thepage\ / \pageref*{LastPage}}
\ifbook
    \fancypagestyle{plain}{%
        \fancyhead[L]{}
        \ifdate
            \fancyhead[R]{\textsc{\today}}
        \else
            \fancyhead[R]{}
        \fi
        \renewcommand{\headrulewidth}{0pt}
    }

    \let\cleardoublepage=\clearpage
\else
    \fancypagestyle{plain}{}
    \renewcommand{\headrulewidth}{0pt}
\fi

\ifdaggerfootnotes
    \renewcommand{\thefootnote}{\fnsymbol{footnote}}
\else
\fi

\delimitershortfall=-1pt
\normalem

\newlist{detail}{itemize}{2}
\setlist[detail]{label={\boldmath$\cdot$},topsep=0pt,leftmargin=*,noitemsep}

\ifalgorithms
    \newcounter{nalg}[chapter]
    \renewcommand{\thenalg}{\thechapter.\arabic{nalg}}
    \DeclareCaptionLabelFormat{algocaption}{\it Algorithm \thenalg}

    \lstnewenvironment{algorithm}[1][]
    {
        \refstepcounter{nalg}
        \captionsetup{labelformat=algocaption,labelsep=colon}
        \lstset{
            mathescape=true,
            frame=tB,
            numbers=left,
            numberstyle=\tiny,
            basicstyle=\scriptsize,
            keywordstyle=\color{black}\bfseries\em,
            keywords={,input, output, return, datatype, function, in, if, else, elif, for, foreach, while, not, begin, end, true, false, null, break, continue, let, and, or, }
            numbers=left,
            xleftmargin=.04\textwidth,
            #1
        }
    }
    {}
\else
\fi

\ifindentproofs
    % begin new proof environment
    \expandafter\let\expandafter\oldproof\csname\string\proof\endcsname
    \let\oldendproof\endproof

    \renewenvironment{proof}[1][\proofname]{%
        \ifindenttheorems
            \vspace{-\abovedisplayskip}
        \else
        \fi
        \oldproof[#1]\quote~\vspace{-\parskip}

    }{%
        %\endquote\oldendproof
        \endquote\vspace{-\parskip}\qed
    }
    % end new proof environment
\else
\fi

\ifindenttheorems
    \newtheorem{pretheorem}{Theorem}
    \newtheorem{prelemma}{Lemma}
    \newtheorem{preproposition}{Proposition}
    \newtheorem{precorollary}{Corollary}
    \newtheorem{preclaim}{Claim}
    \newtheorem{preconjecture}{Conjecture}
    \newtheorem{prejustification}{Justification}

    \newtheorem{preaxiom}{Axiom}
    \newtheorem{predefinition}{Definition}
    \newtheorem{prenotation}{Notation}
    \newtheorem{preexercise}{Exercise}
    \newtheorem{preexample}{Example}
    \newtheorem{precounterexample}{Counterexample}

    \newtheorem{preidea}{Idea}
    \newtheorem*{preremark}{Remark}
    \newtheorem*{prenote}{Note}

    % theorem
    \NewDocumentEnvironment{theorem}{O{} O{}}
        {\begin{pretheorem}[#1]~#2\quote\vspace{-0.75\parskip}}
        {\endquote\end{pretheorem}}
    % lemma
    \NewDocumentEnvironment{lemma}{O{} O{}}
        {\begin{prelemma}[#1]~#2\quote\vspace{-0.75\parskip}}
        {\endquote\end{prelemma}}
    % proposition
    \NewDocumentEnvironment{proposition}{O{} O{}}
        {\begin{preproposition}[#1]~#2\quote\vspace{-0.75\parskip}}
        {\endquote\end{preproposition}}
    % corollary
    \NewDocumentEnvironment{corollary}{O{} O{}}
        {\begin{precorollary}[#1]~#2\quote\vspace{-0.75\parskip}}
        {\endquote\end{precorollary}}
    % claim
    \NewDocumentEnvironment{claim}{O{} O{}}
        {\begin{preclaim}[#1]~#2\quote\vspace{-0.75\parskip}}
        {\endquote\end{preclaim}}
    % conjecture
    \NewDocumentEnvironment{conjecture}{O{} O{}}
        {\begin{preconjecture}[#1]~#2\quote\vspace{-0.75\parskip}}
        {\endquote\end{preconjecture}}
    % justification
    \NewDocumentEnvironment{justification}{O{} O{}}
        {\begin{prejustification}[#1]~#2\quote\vspace{-0.75\parskip}}
        {\endquote\end{prejustification}}

    % axiom
    \NewDocumentEnvironment{axiom}{O{} O{}}
        {\begin{preaxiom}[#1]~#2\quote\normalfont\vspace{-0.75\parskip}}
        {\endquote\end{preaxiom}}
    % definition
    \NewDocumentEnvironment{definition}{O{} O{}}
        {\begin{predefinition}[#1]~#2\quote\normalfont\vspace{-0.75\parskip}}
        {\endquote\end{predefinition}}
    % notation
    \NewDocumentEnvironment{notation}{O{} O{}}
        {\begin{prenotation}[#1]~#2\quote\normalfont\vspace{-0.75\parskip}}
        {\endquote\end{prenotation}}
    % exercise
    \NewDocumentEnvironment{exercise}{O{} O{}}
        {\begin{preexercise}[#1]~#2\quote\normalfont\vspace{-0.75\parskip}}
        {\endquote\end{preexercise}}
    % example
    \NewDocumentEnvironment{example}{O{} O{}}
        {\begin{preexample}[#1]~#2\quote\normalfont\vspace{-0.75\parskip}}
        {\endquote\end{preexample}}
    % counterexample
    \NewDocumentEnvironment{counterexample}{O{} O{}}
        {\begin{precounterexample}[#1]~#2\quote\normalfont\vspace{-0.75\parskip}}
        {\endquote\end{precounterexample}}

    % idea
    \NewDocumentEnvironment{idea}{O{} O{}}
        {\begin{preidea}[#1]~#2\normalfont}
        {\end{preidea}}
    % remark
    \NewDocumentEnvironment{remark}{O{} O{}}
        {\begin{preremark}[#1]~#2\normalfont}
        {\end{preremark}}
    % note
    \NewDocumentEnvironment{note}{O{} O{}}
        {\begin{prenote}[#1]~#2\normalfont}
        {\end{prenote}}
\else
    \theoremstyle{thm}% style for theorems
    \newtheorem{theorem}{Theorem}
    \newtheorem{lemma}{Lemma}
    \newtheorem{proposition}{Proposition}
    \newtheorem{corollary}{Corollary}
    \newtheorem{claim}{Claim}
    \newtheorem{conjecture}{Conjecture}
    \newtheorem{justification}{Justification}

    \theoremstyle{dfn}% style for definitions
    \newtheorem{axiom}{Axiom}
    \newtheorem{definition}{Definition}
    \newtheorem{notation}{Notation}
    \newtheorem{exercise}{Exercise}
    \newtheorem{example}{Example}
    \newtheorem{counterexample}{Counterexample}

    \theoremstyle{rmk}% style for remarks
    \newtheorem{idea}{Idea}
    \newtheorem*{remark}{Remark}
    \newtheorem*{note}{Note}
\fi

\newenvironment{case}[1][Case]
    {\textbf{#1:}\quote\vspace{-0.75\parskip}}
    {\endquote}

\def\lstlistingautorefname{Algorithm}
\def\itemautorefname{Section}
\renewcommand{\chapterautorefname}{Chapter}
\renewcommand{\sectionautorefname}{Section}
\newcommand{\pretheoremautorefname}{Theorem}
\newcommand{\preaxiomautorefname}{Axiom}
\newcommand{\prelemmaautorefname}{Lemma}
\newcommand{\prepropositionautorefname}{Proposition}
\newcommand{\precorollaryautorefname}{Corollary}
\newcommand{\preclaimautorefname}{Claim}
\newcommand{\preconjectureautorefname}{Conjecture}
\newcommand{\prejustificationautorefname}{Justification}
\newcommand{\predefinitionautorefname}{Definition}
\newcommand{\prenotationautorefname}{Notation}
\newcommand{\preexampleautorefname}{Example}
\newcommand{\precounterexampleautorefname}{Counterexample}
\newcommand{\preideaautorefname}{Idea}
\newcommand{\axiomautorefname}{Axiom}
\newcommand{\lemmaautorefname}{Lemma}
\newcommand{\propositionautorefname}{Proposition}
\newcommand{\corollaryautorefname}{Corollary}
\newcommand{\claimautorefname}{Claim}
\newcommand{\conjectureautorefname}{Conjecture}
\newcommand{\justificationautorefname}{Justification}
\newcommand{\definitionautorefname}{Definition}
\newcommand{\notationautorefname}{Notation}
\newcommand{\exampleautorefname}{Example}
\newcommand{\counterexampleautorefname}{Counterexample}
\newcommand{\ideaautorefname}{Idea}

\ifbook
    \renewcommand{\theequation}{\thechapter.\arabic{equation}}
    \renewcommand{\thepretheorem}{\thechapter.\arabic{pretheorem}}
    \renewcommand{\theprelemma}{\thechapter.\arabic{prelemma}}
    \renewcommand{\thepreproposition}{\thechapter.\arabic{preproposition}}
    \renewcommand{\theprecorollary}{\thechapter.\arabic{precorollary}}
    \renewcommand{\thepreclaim}{\thechapter.\arabic{preclaim}}
    \renewcommand{\thepreconjecture}{\thechapter.\arabic{preconjecture}}
    \renewcommand{\theprejustification}{\thechapter.\arabic{prejustification}}
    \renewcommand{\thepredefinition}{\thechapter.\arabic{predefinition}}
    \renewcommand{\theprenotation}{\thechapter.\arabic{prenotation}}
    \renewcommand{\thepreexample}{\thechapter.\arabic{preexample}}
    \renewcommand{\theprecounterexample}{\thechapter.\arabic{precounterexample}}
    \counterwithin*{equation}{chapter}
    \counterwithin*{pretheorem}{chapter}
    \counterwithin*{prelemma}{chapter}
    \counterwithin*{preproposition}{chapter}
    \counterwithin*{precorollary}{chapter}
    \counterwithin*{preclaim}{chapter}
    \counterwithin*{preconjecture}{chapter}
    \counterwithin*{prejustification}{chapter}
    \counterwithin*{predefinition}{chapter}
    \counterwithin*{prenotation}{chapter}
    \counterwithin*{preexercise}{chapter}
    \counterwithin*{preexample}{chapter}
    \counterwithin*{precounterexample}{chapter}
\else
\fi

\newcommand*{\xline}[1][3em]{\rule[0.5ex]{#1}{0.55pt}}

\newcommand{\isomorphic}{\cong}
\newcommand{\iffdefn}{~\mathrel{\vcentcolon\Leftrightarrow}~}
\newcommand{\iffbydefn}{\(\mathrel{\vcentcolon\Leftrightarrow}\)\ }
\newcommand{\niff}{\mathrel{{\ooalign{\hidewidth$\not\phantom{"}$\hidewidth\cr$\iff$}}}}
\renewcommand{\implies}{\Rightarrow}
\renewcommand{\iff}{\Leftrightarrow}
\newcommand{\proves}{\vdash}
\newcommand{\satisfies}{\models}
\renewcommand{\qedsymbol}{\sc q.e.d.}

\renewcommand{\restriction}[1]{\downharpoonright_{#1}}
\renewcommand{\leq}{\leqslant}
\renewcommand{\geq}{\geqslant}

\newcommand{\meet}{\wedge}
\newcommand{\join}{\vee}
\newcommand{\conjunct}{\wedge}
\newcommand{\disjunct}{\vee}
\newcommand{\bigmeet}{\bigwedge}
\newcommand{\bigjoin}{\bigvee}
\newcommand{\bigconjunct}{\bigwedge}
\newcommand{\bigdisjunct}{\bigvee}
\newcommand{\defn}{\coloneqq}
\newcommand{\xor}{\oplus}
\newcommand{\nand}{\uparrow}
\newcommand{\nor}{\downarrow}

\newcommand{\compose}{\circ}
\newcommand{\divides}{~|~}
\newcommand{\notdivides}{\not|~}
\newcommand{\given}{~\middle|~}
\newcommand{\suchthat}{~\middle|~}
\newcommand{\contradiction}{~\smash{\text{\raisebox{-0.6ex}{\Large \Lightning}}}~}

\newcommand{\conjugate}[1]{\overline{#1}}
\newcommand{\mean}[1]{\overline{#1}}

\newcommand*\diff{\mathop{}\!\mathrm{d}}
\newcommand{\integral}[1]{\smashoperator{\int_{#1}}}
\newcommand{\E}[1]{\mathbb{E}\sq*{#1}}
\newcommand{\Esub}[2]{\mathbb{E}_{#1}\sq*{#2}}
\newcommand{\var}[1]{\mathrm{Var}\pn*{#1}}
\newcommand{\cov}[2]{\mathrm{Cov}\pn*{#1, #2}}
\newcommand{\der}[2]{\frac{\diff{#1}}{\diff{#2}}}
\newcommand{\dern}[3]{\frac{\diff^{#3}{#1}}{\diff{#2}^{#3}}}
\newcommand{\derm}[3]{\frac{\diff^{#3}{#1}}{\diff{#2}}}
\newcommand{\prt}[2]{\frac{\partial{#1}}{\partial{#2}}}
\newcommand{\prtn}[3]{\frac{\partial^{#3}{#1}}{\partial{#2}^{#3}}}
\newcommand{\prtm}[3]{\frac{\partial^{#3}{#1}}{\partial{#2}}}
\newcommand{\modulo}[1]{~\pn{\mathrm{mod}~#1}}

\newcommand{\inj}{\hookrightarrow}
\newcommand{\injection}{\hookrightarrow}

\newcommand{\surj}{\twoheadrightarrow}
\newcommand{\surjection}{\twoheadrightarrow}

\newcommand{\bij}{\lhook\joinrel\twoheadrightarrow}
\newcommand{\bijection}{\lhook\joinrel\twoheadrightarrow}

\newcommand{\monic}{\hookrightarrow}
\newcommand{\monomorphism}{\hookrightarrow}

\newcommand{\epic}{\twoheadrightarrow}
\newcommand{\epimorphism}{\twoheadrightarrow}

\newcommand{\iso}{\lhook\joinrel\twoheadrightarrow}
\newcommand{\isomorphism}{\lhook\joinrel\twoheadrightarrow}
\newcommand{\immersion}{\looprightarrow}

\renewcommand{\O}[1]{\mathcal{O}\pn*{#1}}
\renewcommand{\P}[1]{\mathbb{P}\pn*{#1}}
\newcommand{\power}[1]{\mathcal{P}\pn*{#1}}
\newcommand{\successor}[1]{\mathcal{S}\pn*{#1}}
\newcommand{\C}{\mathbb{C}}
\newcommand{\N}{\mathbb{N}}
\newcommand{\Q}{\mathbb{Q}}
\newcommand{\R}{\mathbb{R}}
\newcommand{\Z}{\mathbb{Z}}

% these don't need {} after them since they should be followed by text
\newcommand{\cf}{\textit{c.f.},\ }
\newcommand{\eg}{\textit{e.g.},\ }
\newcommand{\ie}{\textit{i.e.},\ }
\newcommand{\aka}{\textit{a.k.a.}\ }
\newcommand{\viz}{\textit{viz.}\ }
\newcommand{\vide}{\textit{v.}\ }
\newcommand{\ifandonlyif}{\textit{iff}\ }

% these need {} after them
\newcommand{\etal}{\textit{et al.}}
\newcommand{\wff}{\textit{wff}}

\DeclareMathOperator{\lcm}{lcm}
\DeclareMathOperator*{\argmin}{arg\!\min}
\DeclareMathOperator*{\argmax}{arg\!\max}

\let\originalleft\left
\let\originalright\right
\renewcommand{\left}{\mathopen{}\mathclose\bgroup\originalleft}
\renewcommand{\right}{\aftergroup\egroup\originalright}

\newcommand{\zh}[1]{\begin{CJK}{UTF8}{gbsn}#1\end{CJK}}
\newcommand{\jp}[1]{\begin{CJK}{UTF8}{gbsn}#1\end{CJK}}

\DeclarePairedDelimiterX \inner[2]{\langle}{\rangle}{#1,#2}
\DeclarePairedDelimiter \bra{\langle}{\rvert}
\DeclarePairedDelimiter \ket{\lvert}{\rangle}
\DeclarePairedDelimiter \abs{\lvert}{\rvert}
\DeclarePairedDelimiter \cardinality{\lvert}{\rvert}
\DeclarePairedDelimiter \norm{\lVert}{\rVert}
\DeclarePairedDelimiter \set{\lbrace}{\rbrace}
\DeclarePairedDelimiter \seq{\langle}{\rangle}
\DeclarePairedDelimiter \pn{(}{)}
\DeclarePairedDelimiter \sq{[}{]}
\DeclarePairedDelimiter \curly{\lbrace}{\rbrace}
\DeclarePairedDelimiter \bracket{\langle}{\rangle}

\let\oldemptyset\emptyset
\let\emptyset\varnothing
\let\union\cup
\let\intersection\cap
\let\intersect\cap


\begin{document}

\title{Discrete Mathematics}
\author{Daniel Gonzalez Cedre}
\date{University of Notre Dame \\ Spring of 2023}
\maketitle

\datetrue

\setcounter{chapter}{0}
\chapter{Propositional Logic}

\section{Propositions \& Connectives}

\begin{definition}[Proposition]
    A \emph{proposition} is a sentence (in our language)
    that has one (and only one) definite, consistent truth value.
\end{definition}

\begin{definition}[Negation]
    \begin{center}
        \begin{minipage}[t]{.55\linewidth}
            Given a proposition $p$, the \emph{negation} of $p$
            is denoted $\neg p$ and is defined by the following truth table:
            \begin{table}[H]
                \centering
                \label{tab:not}
                \begin{tabular}{|Cc||Cc|}
                    \hline
                    \thead{$p$} & \thead{$\neg p$} \\ \hline
                    \thead{$\top$} & \cellcolor{red!30!white}{\thead{$\bot$}} \\
                    \thead{$\bot$} & \cellcolor{green!30!white}{\thead{$\top$}} \\ \hline
                \end{tabular}
            \end{table}
        \end{minipage}%
        \begin{minipage}[t]{.05\linewidth}
            ~
        \end{minipage}%
        \begin{minipage}[t]{.4\linewidth}
            Some possible readings of $\neg p$:\\
            \begin{itemize}
                \item[$\cdot$]
                    Not $p$.
                \item[$\cdot$]
                    $p$ does not hold.
                \item[$\cdot$]
                    It is not the case that $p$.
                \item[$\cdot$]
                    We do not have that $p$.
            \end{itemize}
        \end{minipage}
    \end{center}
\end{definition}

\begin{definition}[Conjunction]
    \begin{center}
        \begin{minipage}[t]{.55\linewidth}
            Given two propositions $p$ and $q$, the \emph{conjunction} of $p$ with $q$
            is denoted $p \meet q$ and is defined by the following truth table:
            \begin{table}[H]
                \centering
                \label{tab:and}
                \begin{tabular}{|CcCc||Cc|}
                    \hline
                    \thead{$p$} & \thead{$q$} & \thead{$p \meet q$} \\ \hline
                    \thead{$\top$} & \thead{$\top$} & \cellcolor{green!30!white}{\thead{$\top$}} \\
                    \thead{$\top$} & \thead{$\bot$} & \cellcolor{red!30!white}{\thead{$\bot$}} \\
                    \thead{$\bot$} & \thead{$\top$} & \cellcolor{red!30!white}{\thead{$\bot$}} \\
                    \thead{$\bot$} & \thead{$\bot$} & \cellcolor{red!30!white}{\thead{$\bot$}} \\ \hline
                \end{tabular}
            \end{table}
        \end{minipage}%
        \begin{minipage}[t]{.05\linewidth}
            ~
        \end{minipage}%
        \begin{minipage}[t]{.4\linewidth}
            Some possible readings of $p \meet q$:\\
            \begin{itemize}
                \item[$\cdot$]
                    $p$, and $q$.
                \item[$\cdot$]
                    $p$, but $q$.
                \item[$\cdot$]
                    $p$; also, $q$.
                \item[$\cdot$]
                    $p$; further, $q$.
                \item[$\cdot$]
                    In addition to $p$, we also have $q$.
            \end{itemize}
        \end{minipage}
    \end{center}
\end{definition}


\begin{definition}[Disjunction]
    \begin{center}
        \begin{minipage}[t]{.55\linewidth}
            Given two propositions $p$ and $q$, the \emph{disjunction} of $p$ with $q$
            is denoted $p \join q$ and is defined by the following truth table:
            \begin{table}[H]
                \centering
                \label{tab:or}
                \begin{tabular}{|CcCc||Cc|}
                    \hline
                    \thead{$p$} & \thead{$q$} & \thead{$p \join q$} \\ \hline
                    \thead{$\top$} & \thead{$\top$} & \cellcolor{green!30!white}{\thead{$\top$}} \\
                    \thead{$\top$} & \thead{$\bot$} & \cellcolor{green!30!white}{\thead{$\top$}} \\
                    \thead{$\bot$} & \thead{$\top$} & \cellcolor{green!30!white}{\thead{$\top$}} \\
                    \thead{$\bot$} & \thead{$\bot$} & \cellcolor{red!30!white}{\thead{$\bot$}} \\ \hline
                \end{tabular}
            \end{table}
        \end{minipage}%
        \begin{minipage}[t]{.05\linewidth}
            ~
        \end{minipage}%
        \begin{minipage}[t]{.4\linewidth}
            Some possible readings of $p \join q$:\\
            \begin{itemize}
                \item[$\cdot$]
                    $p$, or $q$.
                \item[$\cdot$]
                    Either $p$, or $q$.
            \end{itemize}
        \end{minipage}
    \end{center}
\end{definition}

\begin{definition}[Material Implication]
    \begin{center}
        \begin{minipage}[t]{.55\linewidth}
            Given two propositions $p$ and $q$, the \emph{conditional} formed by assuming $p$ and concluding $q$
            is denoted $p \rightarrow q$ and is defined by the following truth table:
            \begin{table}[H]
                \centering
                \label{tab:implies}
                \begin{tabular}{|CcCc||Cc|}
                    \hline
                    \thead{$p$} & \thead{$q$} & \thead{$p \rightarrow q$} \\ \hline
                    \thead{$\top$} & \thead{$\top$} & \cellcolor{green!30!white}{\thead{$\top$}} \\
                    \thead{$\top$} & \thead{$\bot$} & \cellcolor{red!30!white}{\thead{$\bot$}} \\
                    \thead{$\bot$} & \thead{$\top$} & \cellcolor{green!30!white}{\thead{$\top$}} \\
                    \thead{$\bot$} & \thead{$\bot$} & \cellcolor{green!30!white}{\thead{$\top$}} \\ \hline
                \end{tabular}
            \end{table}
        \end{minipage}%
        \begin{minipage}[t]{.05\linewidth}
            ~
        \end{minipage}%
        \begin{minipage}[t]{.4\linewidth}
            Some possible readings of $p \rightarrow q$:\\
            \begin{itemize}
                \item[$\cdot$]
                    If $p$, then $q$.
                \item[$\cdot$]
                    $p$ implies $q$.
                \item[$\cdot$]
                    $q$ is conditioned on $p$.
                \item[$\cdot$]
                    $q$ only if $p$.
                \item[$\cdot$]
                    $p$ is sufficient for $q$.
                \item[$\cdot$]
                    $q$ is necessary for $p$.
                \item[$\cdot$]
                    $q$ unless not $p$.
                \item[$\cdot$]
                    $q$ or not $p$.
            \end{itemize}
        \end{minipage}
    \end{center}
\end{definition}

\begin{definition}[Biconditional]
    \begin{center}
        \begin{minipage}[t]{.55\linewidth}
            Given two propositions $p$ and $q$, the \emph{biconditional} formed by $p$ and $q$
            is denoted $p \leftrightarrow q$ and is defined by the following truth table:
            \begin{table}[H]
                \centering
                \label{tab:iff}
                \begin{tabular}{|CcCc||Cc|}
                    \hline
                    \thead{$p$} & \thead{$q$} & \thead{$p \leftrightarrow q$} \\ \hline
                    \thead{$\top$} & \thead{$\top$} & \cellcolor{green!30!white}{\thead{$\top$}} \\
                    \thead{$\top$} & \thead{$\bot$} & \cellcolor{red!30!white}{\thead{$\bot$}} \\
                    \thead{$\bot$} & \thead{$\top$} & \cellcolor{red!30!white}{\thead{$\bot$}} \\
                    \thead{$\bot$} & \thead{$\bot$} & \cellcolor{green!30!white}{\thead{$\top$}} \\ \hline
                \end{tabular}
            \end{table}
        \end{minipage}%
        \begin{minipage}[t]{.05\linewidth}
            ~
        \end{minipage}%
        \begin{minipage}[t]{.4\linewidth}
            Some possible readings of $p \leftrightarrow q$:\\
            \begin{itemize}
                \item[$\cdot$]
                    $p$ if and only if $q$.
                \item[$\cdot$]
                    $p$ is necessary and sufficient for $q$.
                \item[$\cdot$]
                    $q$ is necessary and sufficient for $p$.
            \end{itemize}
        \end{minipage}
    \end{center}
\end{definition}

% \begin{definition}[Well-Formed Formula]
%     We say that $\varphi$ is a \emph{proposition} \iffbydefn any one of the following conditions are true:
%     \begin{alignat}{2}
%         \varphi &= \top\\
%         \varphi &= \bot\\
%         \varphi &= \neg (p), &&\text{ where $p$ is a proposition}\\
%         \varphi &= (p \meet q), &&\text{ where $p$ and $q$ are propositions}\\
%         \varphi &= (p \join q), &&\text{ where $p$ and $q$ are propositions}\\
%         \varphi &= (p \rightarrow q), &&\text{ where $p$ and $q$ are propositions}\\
%         \varphi &= (p \leftrightarrow q), &&\text{ where $p$ and $q$ are propositions}
%     \end{alignat}
%     As we can see, this is a \emph{recursive} definition.
%     We start with two base cases that claim that $\top$ and $\bot$ are propositions.
%     We then take existing propositions (starting with $\top$ and $\bot$)
%     and build more complicated expressions out of simpler ones.
% \end{definition}

% \begin{definition}[Equivalence]
%     Let $n$ and $m$ be natural numbers (whole numbers $\geq 0$).
%     Let $\varphi_n$ and $\psi_m$ be propositional expressions composed of $n$ and $m$ propositional variables respectively.
%     With the notation $\varphi_n \iff \psi_n$,
%     we say that ``$\varphi_n$ is logically equivalent to $\psi_n$''
%     \iffbydefn every assignment of truth values (either $\top$ or $\bot$)
%     to the propositional variables of $\varphi_n$ and $\psi_m$ results in the two expressions having the same truth value.
% \end{definition}

\begin{definition}[Equivalence]
    If we have two expressions $\varphi$ and $\psi$ in our formal language,
    consisting of some number of (possibly shared) propositional variables, connected together by logical connectives,
    then with the notation $\varphi \iff \psi$ we say that $\varphi$ is equivalent to $\psi$
    \iffbydefn every assignment of truth values to the propositional variables of $\varphi$ and $\psi$ results
    in the same truth value for the two expressions.
\end{definition}
\begin{example}
    Consider the two expressions $\varphi \defn p \rightarrow q$ and $\psi \defn \neg p \join q$.
    We can see that $\varphi$ has two propositional variables: $p$ and $q$.
    $\psi$ also has two propositional variables: the same $p$ and the same $q$.

    If we construct the truth table for these two expressions,
    we will see that every assignment of truth values to $p$ and $q$ will result in $p \rightarrow q$ and $\neg p \join q$
    having the same truth value.

    \begin{table}[H]
        \centering
        \begin{tabular}{|CcCc||CcCc|}
            \hline
            \thead{$p$} & \thead{$q$} & \thead{$p \rightarrow q$} & \thead{$\neg p \join q$}\\ \hline
            \thead{$\top$} & \thead{$\top$} & \cellcolor{green!30!white}{\thead{$\top$}} & \cellcolor{green!30!white}{$\top$}\\
            \thead{$\top$} & \thead{$\bot$} & \cellcolor{red!30!white}{\thead{$\bot$}} & \cellcolor{red!30!white}{\thead{$\bot$}}\\
            \thead{$\bot$} & \thead{$\top$} & \cellcolor{green!30!white}{\thead{$\top$}} & \cellcolor{green!30!white}{\thead{$\top$}} \\
            \thead{$\bot$} & \thead{$\bot$} & \cellcolor{green!30!white}{\thead{$\top$}} & \cellcolor{green!30!white}{\thead{$\top$}} \\ \hline
        \end{tabular}
    \end{table}
\end{example}

\begin{definition}[Tautology]
    A propositional expression $\varphi$ consisting the propositional variables $p_1, \dots p_n$ is a \emph{tautology}
    \iffbydefn every assignment of truth values to $p_1, \dots p_n$ results in $\varphi$ being equivalent to $\top$.
\end{definition}
\begin{definition}[Contradiction]
    A propositional expression $\varphi$ consisting the propositional variables $p_1, \dots p_n$ is a \emph{contradiction}
    \iffbydefn every assignment of truth values to $p_1, \dots p_n$ results in $\varphi$ being equivalent to $\bot$.
\end{definition}

\section{Boolean Algebras}

\begin{definition}[Boolean Algebra]
    A \emph{Boolean algebra}
    is a collection of \emph{terms} $B$ with two distinguished (and distinct) terms called $\top$ and $\bot$,
    along with a unary operation called $\neg$ and two binary operations called $\meet$ and $\join$,
    such that the following statements are true for any terms $p, q, r$ in $B$:

    \begin{table}[H]
        \centering
        \label{tab:boole}
        \begin{tabular}{|Cc|Cc|}
            \hline
            \multicolumn{2}{|c|}{\thead{Axioms of a Boolean Algebra}} \\\hline
            \thead{Identity} & \thead{$p \meet \top \iff p$\\$p \join \bot \iff p$} \\ \hline
            \thead{Complement\\(\aka Negation)} & \thead{$p \meet \neg p \iff \bot$ \\$p \join \neg p \iff \top$} \\ \hline
            \thead{Commutativity} & \thead{$p \meet q \iff q \meet p$\\$p \join q \iff q \join p$} \\ \hline
            \thead{Associativity} & \thead{$p \meet (q \meet r) \iff (p \meet q) \meet r$\\$p \join (q \join r) \iff (p \join q) \join r$} \\ \hline
            \thead{Distributive Laws} & \thead{$p \conjunct (q \disjunct r) \iff (p \conjunct q) \disjunct (p \conjunct r)$\\$p \disjunct (q \conjunct r) \iff (p \disjunct q) \conjunct (p \disjunct r)$} \\ \hline
        \end{tabular}
    \end{table}

    This kind of structure is also referred to as a \emph{complemented, distributive lattice}.
    Since we are establishing the algebra of \emph{propositions},
    our terms consist only of $\top$ and $\bot$.

    However, since none of these axioms tell us how to use the (very useful) symbols
    $\rightarrow$ and $\leftrightarrow$, we need two additional axioms that will turn our Boolean algebra
    into an example of a \emph{Heyting algebra}:

    \begin{table}[H]
        \centering
        \label{tab:heyting}
        \begin{tabular}{|Cc|Cc|}
            \hline
            \multicolumn{2}{|c|}{\thead{Heyting Axioms}} \\\hline
            \thead{Conditional Disintegration}   & \thead{$p \rightarrow q \iff \neg p \join q$} \\ \hline
            \thead{Biconditional Disintegration} & \thead{$p \leftrightarrow q \iff (p \rightarrow q) \meet (q \rightarrow p)$} \\ \hline
        \end{tabular}
    \end{table}

    By referring to the truth tables, it should be easy to see that these axioms are \emph{truth preserving} transformations,
    meaning that taking an expression like $p \meet (q \rightarrow r)$ and applying an axiom like Identity to it
    does not change the truth value of the resulting expression $(p \meet \top) \meet (q \rightarrow r)$.
    For this reason, these are sometimes referred to as \emph{equivalence laws} and many treatments of this subject
    \emph{prove} these laws by referring to the truth tables.

    For our purposes, we don't need to refer to the truth tables at all.
    The truth tables were a nice, intuitive, and compact way of defining the logical connectives,
    but we could just as easily have defined them by assuming that all of the axioms are true,
    without ever writing down a truth table.
    This provides a more \emph{algebraic} approach to the study of logic,
    which is more in-line with the way logic is used to actually prove theorems in mathematics.

    As such, while the while the truth tables provided a nice way of defining the logical connectives,
    we will be taking the algebraic approach by \emph{assuming the axioms} of a Boolean algebra
    are true about our propositional logic
    and using them to prove theorems about our logical system.
\end{definition}

\begin{theorem}[Uniqueness of Complements]\label{thm:unique}
    Let $p$ be a term in a Boolean algebra $(B, \neg, \meet, \join)$.
    Suppose there were two terms $x$ and $y$ in the Boolean algebra such that
    \begin{center}
        \begin{minipage}{.2\linewidth}
            \begin{align*}
                p \meet x &\iff \bot,\\
                p \join x &\iff \top,
            \end{align*}
        \end{minipage}%
        \begin{minipage}{.2\linewidth}
            \begin{align*}
                p \meet y &\iff \bot,\\
                p \join y &\iff \top,
            \end{align*}
        \end{minipage}
    \end{center}
    meaning that $x$ and $y$ act like negations for $p$.
    Then, we have $x \iff y$.
\end{theorem}
\begin{proof}
    Let $(B, \neg, \meet, \join)$ be a Boolean algebra
    and consider an arbitrary term $p$ in the algebra.
    Suppose we have $x$ and $y$ satisfying the conditions given in the statement of the theorem above.
    Then, we can observe
    \begin{alignat*}{2}
        x &\iff \top \meet x ~~~~~~~~~~~~~~~~&&\text{by Identity}\\
          &\iff (p \join y) \meet x &&\text{by the assumption $p \join y \iff \top$}\\
          &\iff (p \meet x) \join (y \meet x) &&\text{by Distributivity}\\
          &\iff \bot \join (y \meet x) &&\text{by the assumption $p \meet x \iff \bot$}\\
          &\iff y \meet x &&\text{by Identity}.
    \end{alignat*}
    Similarly, we can see that
    \begin{alignat*}{2}
        y &\iff \top \meet y ~~~~~~~~~~~~~~~~&&\text{by Identity}\\
          &\iff (p \join x) \meet y &&\text{by the assumption $p \join x \iff \top$}\\
          &\iff (p \meet y) \join (x \meet y) &&\text{by Distributivity}\\
          &\iff \bot \join (x \meet y) &&\text{by the assumption $p \meet y \iff \bot$}\\
          &\iff x \meet y &&\text{by Identity}.
    \end{alignat*}
    So, from Commutativity, we can conclude that $x \iff (y \meet x) \iff (x \meet y) \iff y$.
\end{proof}

\begin{theorem}[Double Negation]
    For any Boolean algebra $(B, \neg, \meet, \join)$ and any term $p$ in the algebra,
    we have $\neg \neg p \iff p$.
\end{theorem}
\begin{proof}
    Let $(B, \neg, \meet, \join)$ be a Boolean algebra
    and consider an arbitrary term $p$ in the algebra.
    We want to show that $\neg \neg p \iff p$.

    By the Complement axiom, we know $p \meet \neg p \iff \bot$ and $p \join \neg p \iff \top$,
    meaning that $p$ is the complement of $\neg p$.
    Similarly, we can see that $\neg p \meet \neg (\neg p) \iff \bot$ and $\neg p \join \neg (\neg p) \iff \top$,
    showing us that $\neg \neg p$ is the complement of $\neg p$.

    Since complements are unique, as seen in \autoref{thm:unique},
    and both $p$ and $\neg \neg p$ are complements of $\neg p$, we must have that $p \iff \neg \neg p$.
\end{proof}

\begin{theorem}[Idempotency]
    For every Boolean algebra $(B, \neg, \meet, \join)$ and every term $p$ in the algebra,
    we have
    \begin{align*}
        p \meet p &\iff p\\
        p \join p &\iff p.
    \end{align*}
\end{theorem}
\begin{proof}
    Let $p$ be a term in an arbitrary Boolean algebra $(B, \neg, \meet, \join)$.
    Observe
    \begin{center}
        \begin{minipage}{.45\linewidth}
            \begin{alignat*}{2}
                p \meet p &\iff (p \meet p) \join \bot~~~~~~~~~~~~&&\text{by Identity}\\
                          &\iff (p \meet p) \join (p \meet \neg p) &&\text{by Complement}\\
                          &\iff p \meet (p \join \neg p) &&\text{by Distributivity}\\
                          &\iff p \meet \top &&\text{by Complement}\\
                          &\iff p &&\text{by Identity},
            \end{alignat*}
        \end{minipage}%
        \begin{minipage}{.1\linewidth}
            ~
        \end{minipage}%
        \begin{minipage}{.45\linewidth}
            \begin{alignat*}{2}
                p \join p &\iff (p \join p) \meet \top~~~~~~~~~~~~&&\text{by Identity}\\
                          &\iff (p \join p) \meet (p \join \neg p) &&\text{by Complement}\\
                          &\iff p \join (p \meet \neg p) &&\text{by Distributivity}\\
                          &\iff p \join \bot &&\text{by Complement}\\
                          &\iff p &&\text{by Identity}.
            \end{alignat*}
        \end{minipage}
    \end{center}
    Therefore, $p \meet p \iff p$ and $p \join p \iff p$, as desired.
\end{proof}

% TODO
\begin{theorem}[Domination]
    For every Boolean algebra $(B, \neg, \meet, \join)$ and every term $p$ in the algebra,
    we have
    \begin{align*}
        p \meet \bot &\iff \bot\\
        p \join \top &\iff \top.
    \end{align*}
\end{theorem}
\begin{proof}
    Let $p$ be a term in an arbitrary Boolean algebra $(B, \neg, \meet, \join)$.
    Observe
    \begin{center}
        \begin{minipage}{.45\linewidth}
            \begin{alignat*}{2}
                p \meet \bot &\iff p \meet (p \meet \neg p)~~~~~~~~~~~~&&\text{by Complement}\\
                             &\iff (p \meet p) \meet \neg p &&\text{by Associativity}\\
                             &\iff p \meet \neg p &&\text{by Idempotency}\\
                             &\iff \bot &&\text{by Complement},
            \end{alignat*}
        \end{minipage}%
        \begin{minipage}{.1\linewidth}
            ~
        \end{minipage}%
        \begin{minipage}{.45\linewidth}
            \begin{alignat*}{2}
                p \join \top &\iff p \join (p \join \neg p)~~~~~~~~~~~~&&\text{by Complement}\\
                             &\iff (p \join p) \join \neg p &&\text{by Associativity}\\
                             &\iff p \join \neg p &&\text{by Idempotency}\\
                             &\iff \top &&\text{by Complement}.
            \end{alignat*}
        \end{minipage}
    \end{center}
    Therefore, $p \meet \bot \iff \bot$ and $p \join \top \iff \top$.
\end{proof}

\begin{theorem}[Absorption]
    For every Boolean algebra $(B, \neg, \meet, \join)$ and any two terms $p$ and $q$ in the algebra,
    we have
    \begin{align*}
        p \meet (p \join q) &\iff p\\
        p \join (p \meet q) &\iff q.
    \end{align*}
\end{theorem}
\begin{proof}
    Let $p$ and $q$ be terms in an arbitrary Boolean algebra $(B, \neg, \meet, \join)$.
    Observe
    \begin{center}
        \begin{minipage}{.45\linewidth}
            \begin{alignat*}{2}
                p \meet (p \join q) &\iff (p \join \bot) \meet (p \join q)~~&&\text{by Identity}\\
                                    &\iff p \join (\bot \meet q) &&\text{by Distributivity}\\
                                    &\iff p \join \bot &&\text{by Domination}\\
                                    &\iff p &&\text{by Identity},
            \end{alignat*}
        \end{minipage}%
        \begin{minipage}{.1\linewidth}
            ~
        \end{minipage}%
        \begin{minipage}{.45\linewidth}
            \begin{alignat*}{2}
                p \join (p \meet q) &\iff (p \meet \top) \join (p \meet q)~~&&\text{by Identity}\\
                                    &\iff p \meet (\top \join q) &&\text{by Distributivity}\\
                                    &\iff p \meet \top &&\text{by Domination}\\
                                    &\iff p &&\text{by Identity}.
            \end{alignat*}
        \end{minipage}
    \end{center}
    Therefore, $p \meet (p \join q) \iff p$ and $p \join (p \meet q) \iff q$.
\end{proof}

\begin{theorem}[De Morgan's Laws]
    For every Boolean algebra $(B, \neg, \meet, \join)$ and any two terms $p$ and $q$ in the algebra,
    we have
    \begin{align*}
        \neg (p \meet q) &\iff \neg p \join \neg q\\
        \neg (p \join q) &\iff \neg p \meet \neg q.
    \end{align*}
\end{theorem}
\begin{proof}
    Let $p$ and $q$ be terms in an arbitrary Boolean algebra $(B, \neg, \meet, \join)$.
    For the first statement, we need to show that $p \meet q$ is a complement for $\neg p \join \neg q$,
    meaning $(p \meet q) \meet (\neg p \join \neg q) \iff \bot$
    and $(p \meet q) \join (\neg p \join \neg q) \iff \top$.

    Observe that
    \begin{alignat*}{2}
        (p \meet q) \meet (\neg p \join \neg q) &\iff (q \meet p) \meet (\neg p \join \neg q)~~~~~~~~~~~~&&\text{by Commutativity}\\
                                                &\iff q \meet (p \meet (\neg p \join \neg q)) &&\text{by Associativity}\\
                                                &\iff q \meet ((p \meet \neg p) \join (p \meet \neg q)) &&\text{by Distributivity}\\
                                                &\iff q \meet (\bot \join (p \meet \neg q)) &&\text{by Complement}\\
                                                &\iff q \meet (p \meet \neg q) &&\text{by Identity}\\
                                                &\iff q \meet (\neg q \meet p) &&\text{by Commutativity}\\
                                                &\iff (q \meet \neg q) \meet p &&\text{by Associativity}\\
                                                &\iff \bot \meet p &&\text{by Complement}\\
                                                &\iff \bot &&\text{by Domination}.
    \end{alignat*}
    Further, we have
    \begin{alignat*}{2}
        (p \meet q) \join (\neg p \join \neg q) &\iff (p \meet q) \meet (\neg q \join \neg p)~~~~~~~~~~~~&&\text{by Commutativity}\\
                                                &\iff ((p \meet q) \join \neg q) \join \neg p &&\text{by Associativity}\\
                                                &\iff ((p \join \neg q) \meet (q \join \neg q)) \join \neg p &&\text{by Distributivity}\\
                                                &\iff ((p \join \neg q) \meet \top) \join \neg p &&\text{by Complement}\\
                                                &\iff (p \join \neg q) \join \neg p &&\text{by Identity}\\
                                                &\iff (\neg q \join p) \join \neg p &&\text{by Commutativity}\\
                                                &\iff \neg q \join (p \join \neg p) &&\text{by Associativity}\\
                                                &\iff \neg q \join \top &&\text{by Complement}\\
                                                &\iff \top &&\text{by Domination}.
    \end{alignat*}
    So, we can see that $p \meet q$ is a complement for $\neg p \join \neg q$.
    Since complements are unique by \autoref{thm:unique},
    we can conclude that $\neg (p \meet q) \iff \neg p \join \neg q$.

    Showing $\neg (p \join q) \iff \neg p \meet \neg q$ is left as an exercise to the reader.
\end{proof}
\end{document}
