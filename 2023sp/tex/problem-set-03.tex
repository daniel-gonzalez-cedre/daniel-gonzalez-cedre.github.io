% define new conditionals: \if<conditional>
\newif\ifbook  % whether or not to use book styles
\newif\ifdate  % whether or not to include the current date header
\newif\ifalgorithms  % whether or not to define the algorithm environment
\newif\ifindentproofs  % whether or not to hang-indent proof environments
\newif\ifindenttheorems  % whether or not to hang-indent theorem environments

% set the conditionals: \<conditional>true / \<conditional>false
\bookfalse
\datefalse
\algorithmsfalse
\indentproofstrue
\indenttheoremstrue

\ifbook
    \documentclass[letterpaper]{book}
    \usepackage{arydshln, chngcntr}
\else
    \documentclass[letterpaper]{article}
\fi

% math
\usepackage{amsmath, amsfonts, amssymb, amstext, amscd, amsthm, mathrsfs, mathtools, xfrac}
% fonts
\usepackage{bbm, CJKutf8, caption, dsfont, marvosym, stmaryrd}
% tables
\usepackage{booktabs, colortbl, makecell}
% colors
\usepackage{color, soul, xcolor}
% references
\usepackage{hyperref, xr-hyper, url}
% figures
\usepackage{graphicx, float, tikz}
% headers and footers
\usepackage{fancyhdr, lastpage}
% miscellaneous
\usepackage{enumerate, ifthen, lipsum, listings, makeidx, parskip, verbatim, xargs}
\usepackage[nodayofweek]{datetime}

\usepackage[left=2cm,top=2cm,right=2cm,bottom=2cm,bindingoffset=0cm]{geometry}
\usepackage[group-separator={,},group-minimum-digits={3}]{siunitx}
\usepackage[shortlabels]{enumitem}
\usepackage[math]{cellspace}
\cellspacetoplimit 1pt
\cellspacebottomlimit 1pt

\definecolor{gruvred}{HTML}{CC214D}
\definecolor{gruvorange}{HTML}{D65D0E}
\definecolor{gruvaqua}{HTML}{689D6A}
\definecolor{gruvpurple}{HTML}{B16286}

\hypersetup{
    colorlinks=true,
    linkcolor=gruvorange,
    citecolor=gruvaqua,
    urlcolor=gruvpurple
}

\allowdisplaybreaks
\newdateformat{verbosedate}{\ordinal{DAY} of \monthname[\THEMONTH], \THEYEAR}
\verbosedate

\pagestyle{fancy}
% \fancyfoot[C]{--~\thepage~--}
\fancyfoot[C]{\tiny \thepage\ / \pageref*{LastPage}}
\ifbook
    \fancypagestyle{plain}{%
        \fancyhead[L]{}
        \ifdate
            \fancyhead[R]{\textsc{\today}}
        \else
            \fancyhead[R]{}
        \fi
        \renewcommand{\headrulewidth}{0pt}
    }

    \let\cleardoublepage=\clearpage
\else
    \fancypagestyle{plain}{}
    \renewcommand{\headrulewidth}{0pt}
\fi

\delimitershortfall=-1pt

\newlist{detail}{itemize}{2}
\setlist[detail]{label={\boldmath$\cdot$},topsep=0pt,leftmargin=*,noitemsep}

\ifalgorithms
    \newcounter{nalg}[chapter]
    \renewcommand{\thenalg}{\thechapter.\arabic{nalg}}
    \DeclareCaptionLabelFormat{algocaption}{\it Algorithm \thenalg}

    \lstnewenvironment{algorithm}[1][]
    {
        \refstepcounter{nalg}
        \captionsetup{labelformat=algocaption,labelsep=colon}
        \lstset{
            mathescape=true,
            frame=tB,
            numbers=left,
            numberstyle=\tiny,
            basicstyle=\scriptsize,
            keywordstyle=\color{black}\bfseries\em,
            keywords={,input, output, return, datatype, function, in, if, else, elif, for, foreach, while, not, begin, end, true, false, null, break, continue, let, and, or, }
            numbers=left,
            xleftmargin=.04\textwidth,
            #1
        }
    }
    {}
\else
\fi

% new proof environment
\expandafter\let\expandafter\oldproof\csname\string\proof\endcsname
\let\oldendproof\endproof
\renewenvironment{proof}[1][\proofname]{%
    \vspace{-0.5\parskip}%
    \oldproof[#1]
}{%
    ~\\\qed
}

\newtheoremstyle{cedretheorem} % name
    {1ex}  % space above
    {-1ex}  % space below
    {\itshape} % body font
    {}  % indent amount
    {\bfseries}  % theorem head font
    {.\\}  % punctuation after theorem head
    {.5em}  % space after theorem head
    {}  % theorem head spec (can be left empty, meaning ‘normal’)

\newtheoremstyle{cedredefinition} % name
    {1ex}  % space above
    {-1ex}  % space below
    {} % body font
    {}  % indent amount
    {\bfseries}  % theorem head font
    {.\\}  % punctuation after theorem head
    {.5em}  % space after theorem head
    {}  % theorem head spec (can be left empty, meaning ‘normal’)

\newtheoremstyle{cedrenote} % name
    {}  % space above
    {}  % space below
    {} % body font
    {}  % indent amount
    {\bfseries}  % theorem head font
    {.}  % punctuation after theorem head
    {.5em}  % space after theorem head
    {}  % theorem head spec (can be left empty, meaning ‘normal’)

% style for theorems
\theoremstyle{cedretheorem}
\newtheorem{theorem}{Theorem}
\newtheorem{lemma}{Lemma}[theorem]
\newtheorem{proposition}{Proposition}[theorem]
\newtheorem{corollary}{Corollary}[theorem]
\newtheorem{conjecture}{Conjecture}[theorem]
\newtheorem*{claim}{Claim}
\newtheorem*{justification}{Justification}

% style for definitions
\theoremstyle{cedredefinition}
\newtheorem{axiom}{Axiom}
\newtheorem{definition}{Definition}
\newtheorem{notation}{Notation}[definition]
\newtheorem{exercise}{Exercise}[definition]
\newtheorem{example}{Example}[definition]
\newtheorem*{counterexample}{Counterexample}

% style for notes
\theoremstyle{cedrenote}
\newtheorem{idea}{Idea}[definition]
\newtheorem*{remark}{Remark}
\newtheorem*{note}{Note}

\newenvironment{case}[1][Case]
    {\quote\textbf{#1:}~\\}
    {\endquote}

\def\lstlistingautorefname{Algorithm}
\def\itemautorefname{Section}
\renewcommand{\chapterautorefname}{Chapter}
\renewcommand{\sectionautorefname}{Section}
\renewcommand{\theoremautorefname}{Theorem}
\newcommand{\axiomautorefname}{Axiom}
\newcommand{\lemmaautorefname}{Lemma}
\newcommand{\propositionautorefname}{Proposition}
\newcommand{\corollaryautorefname}{Corollary}
\newcommand{\claimautorefname}{Claim}
\newcommand{\conjectureautorefname}{Conjecture}
\newcommand{\justificationautorefname}{Justification}
\newcommand{\definitionautorefname}{Definition}
\newcommand{\notationautorefname}{Notation}
\newcommand{\exampleautorefname}{Example}
\newcommand{\counterexampleautorefname}{Counterexample}
\newcommand{\ideaautorefname}{Idea}

\ifbook
    \renewcommand{\theequation}{\thechapter.\arabic{equation}}
    \renewcommand{\thetheorem}{\thechapter.\arabic{theorem}}
    \renewcommand{\thelemma}{\thechapter.\arabic{lemma}}
    \renewcommand{\theproposition}{\thechapter.\arabic{proposition}}
    \renewcommand{\thecorollary}{\thechapter.\arabic{corollary}}
    \renewcommand{\theconjecture}{\thechapter.\arabic{conjecture}}
    % \renewcommand{\theclaim}{\thechapter.\arabic{claim}}
    % \renewcommand{\thejustification}{\thechapter.\arabic{justification}}
    \renewcommand{\thedefinition}{\thechapter.\arabic{definition}}
    \renewcommand{\thenotation}{\thechapter.\arabic{notation}}
    \renewcommand{\theexample}{\thechapter.\arabic{example}}
    % \renewcommand{\thecounterexample}{\thechapter.\arabic{counterexample}}
    \counterwithin*{equation}{chapter}
    \counterwithin*{theorem}{chapter}
    \counterwithin*{lemma}{chapter}
    \counterwithin*{proposition}{chapter}
    \counterwithin*{corollary}{chapter}
    \counterwithin*{conjecture}{chapter}
    % \counterwithin*{claim}{chapter}
    % \counterwithin*{justification}{chapter}
    \counterwithin*{definition}{chapter}
    \counterwithin*{notation}{chapter}
    \counterwithin*{exercise}{chapter}
    \counterwithin*{example}{chapter}
    % \counterwithin*{counterexample}{chapter}
\else
\fi

\newcommand*{\xline}[1][3em]{\rule[0.5ex]{#1}{0.55pt}}

\newcommand{\isomorphic}{\cong}
\newcommand{\iffdefn}{\mathrel{\vcentcolon\Leftrightarrow}}
\newcommand{\iffbydefn}{$\iffdefn{}$}
\newcommand{\niff}{\mathrel{{\ooalign{\hidewidth$\not\phantom{"}$\hidewidth\cr$\iff$}}}}
\renewcommand{\implies}{~\Rightarrow~}
\renewcommand{\iff}{~\Leftrightarrow~}
\renewcommand{\restriction}[1]{\downharpoonright_{#1}}
\renewcommand{\qedsymbol}{\sc q.e.d.}
\renewcommand{\leq}{\leqslant}
\renewcommand{\geq}{\geqslant}

\newcommand{\meet}{\wedge}
\newcommand{\join}{\vee}
\newcommand{\conjunct}{\wedge}
\newcommand{\disjunct}{\vee}
\newcommand{\defn}{\coloneqq}
\newcommand{\xor}{\oplus}
\newcommand{\nand}{\uparrow}
\newcommand{\nor}{\downarrow}

\newcommand{\compose}{\circ}
\newcommand{\divides}{~|~}
\newcommand{\notdivides}{\not|~}
\newcommand{\given}{~\middle|~}
\newcommand{\suchthat}{~\middle|~}
\newcommand{\contradiction}{~\smash{\text{\Large \Lightning}}~}

\newcommand{\conjugate}[1]{\overline{#1}}
\newcommand{\mean}[1]{\overline{#1}}

\newcommand*\diff{\mathop{}\!\mathrm{d}}
\newcommand{\integral}[1]{\smashoperator{\int_{#1}}}
\newcommand{\E}[1]{\mathbb{E}\crochets*{#1}}
\newcommand{\Esub}[2]{\mathbb{E}_{#1}\crochets*{#2}}
\newcommand{\var}[1]{\mathrm{Var}\parens*{#1}}
\newcommand{\cov}[2]{\mathrm{Cov}\parens*{#1, #2}}
\newcommand{\der}[2]{\frac{\diff{#1}}{\diff{#2}}}
\newcommand{\dern}[3]{\frac{\diff^{#3}{#1}}{\diff{#2}^{#3}}}
\newcommand{\derm}[3]{\frac{\diff^{#3}{#1}}{\diff{#2}}}
\newcommand{\prt}[2]{\frac{\partial{#1}}{\partial{#2}}}
\newcommand{\prtn}[3]{\frac{\partial^{#3}{#1}}{\partial{#2}^{#3}}}
\newcommand{\prtm}[3]{\frac{\partial^{#3}{#1}}{\partial{#2}}}
\newcommand{\modulo}[1]{~\parens{\mathrm{mod}~#1}}

\newcommand{\inj}{\hookrightarrow}
\newcommand{\injection}{\hookrightarrow}

\newcommand{\surj}{\twoheadrightarrow}
\newcommand{\surjection}{\twoheadrightarrow}

\newcommand{\bij}{\lhook\joinrel\twoheadrightarrow}
\newcommand{\bijection}{\lhook\joinrel\twoheadrightarrow}

\newcommand{\monic}{\hookrightarrow}
\newcommand{\monomorphism}{\hookrightarrow}

\newcommand{\epic}{\twoheadrightarrow}
\newcommand{\epimorphism}{\twoheadrightarrow}

\newcommand{\iso}{\lhook\joinrel\twoheadrightarrow}
\newcommand{\isomorphism}{\lhook\joinrel\twoheadrightarrow}
\newcommand{\immersion}{\looprightarrow}

\renewcommand{\O}[1]{\mathcal{O}\parens*{#1}}
\renewcommand{\P}[1]{\mathcal{P}\parens*{#1}}
\newcommand{\C}{\mathbb{C}}
\newcommand{\N}{\mathbb{N}}
\newcommand{\Q}{\mathbb{Q}}
\newcommand{\R}{\mathbb{R}}
\newcommand{\Z}{\mathbb{Z}}

\newcommand{\century}{c.\ }
\newcommand{\ca}{\textit{ca.}\ }
\newcommand{\cf}{\textit{c.f.},\ }
\newcommand{\eg}{\textit{e.g.},\ }
\newcommand{\ie}{\textit{i.e.},\ }
\newcommand{\aka}{\textit{a.k.a.}\ }
\newcommand{\viz}{\textit{viz.}\ }
\newcommand{\vide}{\textit{v.}\ }
\newcommand{\etal}{\textit{et al.}\ }

\DeclareMathOperator{\lcm}{lcm}
\DeclareMathOperator*{\argmin}{arg\!\min}
\DeclareMathOperator*{\argmax}{arg\!\max}

\let\originalleft\left
\let\originalright\right
\renewcommand{\left}{\mathopen{}\mathclose\bgroup\originalleft}
\renewcommand{\right}{\aftergroup\egroup\originalright}

\newcommand{\zh}[1]{\begin{CJK}{UTF8}{gbsn}#1\end{CJK}}
\newcommand{\jp}[1]{\begin{CJK}{UTF8}{gbsn}#1\end{CJK}}

\DeclarePairedDelimiterX \inner[2]{\langle}{\rangle}{#1,#2}
\DeclarePairedDelimiter \bra{\langle}{\rvert}
\DeclarePairedDelimiter \ket{\lvert}{\rangle}
\DeclarePairedDelimiter \abs{\lvert}{\rvert}
\DeclarePairedDelimiter \cardinality{\lvert}{\rvert}
\DeclarePairedDelimiter \norm{\lVert}{\rVert}
\DeclarePairedDelimiter \set{\lbrace}{\rbrace}
\DeclarePairedDelimiter \seq{\langle}{\rangle}
\DeclarePairedDelimiter \parens{(}{)}
\DeclarePairedDelimiter \crochets{[}{]}
\DeclarePairedDelimiter \brackets{\langle}{\rangle}

\let\oldemptyset\emptyset
\let\emptyset\varnothing
\let\union\cup
\let\intersection\cap
\let\intersect\cap


\begin{document}
\begin{center}
    \textsc{\huge Problem Set 3}\\
    \textsc{Discrete Mathematics}\\
    {\color{gruvred}Due: $10$\textsuperscript{th} of February, $2023$}
\end{center}

\begin{enumerate}
    \item
        Show that the following arguments are valid.
        \begin{remark}
            Every application of a rule of inference must be explicitly referenced by name for these subproblems.
        \end{remark}
        \begin{enumerate}
            \item
                \(
                    \begin{array}{l}
                        \varphi \rightarrow \psi \\
                        \cline{1-1}
                        \varphi \implies q
                    \end{array}
                \)
                \hfill known as \textbf{Conditional Elimination}
            \item
                \(
                    \begin{array}{l}
                        \varphi \rightarrow \psi \\
                        \psi \rightarrow \chi \\
                        \cline{1-1}
                        \varphi \rightarrow \chi
                    \end{array}
                \)
                \hfill known as the \textbf{Hypothetical Syllogism}
            \item
                \(
                    \begin{array}{l}
                        \varphi \\
                        \psi \\
                        \cline{1-1}
                        \varphi \meet \psi
                    \end{array}
                \)
                \hfill known as \textbf{Adjunction}, \aka \textbf{Conjunction Introduction}
            \item
                \(
                    \begin{array}{l}
                        \varphi \meet \psi \\
                        \cline{1-1}
                        \varphi
                    \end{array}
                \)
                \hfill known as \textbf{Simplification}, \aka \textbf{Conjunction Elimination}
            \item
                \(
                    \begin{array}{l}
                        \varphi \\
                        \cline{1-1}
                        \varphi \join \psi
                    \end{array}
                \)
                \hfill known as \textbf{Addition}, \aka \textbf{Disjunction Introduction}
            \item
                \(
                    \begin{array}{l}
                        \varphi \rightarrow \chi \\
                        \psi \rightarrow \chi \\
                        \varphi \join \psi \\
                        \cline{1-1}
                        \chi
                    \end{array}
                \)
                \hfill known as \textbf{Proof by Cases}, \aka \textbf{Disjunction Elimination}
            \item
                \(
                    \begin{array}{l}
                        \varphi \join \psi \\
                        \neg \varphi \\
                        \cline{1-1}
                        \psi
                    \end{array}
                \)
                \hfill known as the \textbf{Disjunctive Syllogism}
            \item
                \(
                    \begin{array}{l}
                        \varphi \rightarrow \chi \\
                        \psi \rightarrow \xi \\
                        \varphi \join \psi \\
                        \cline{1-1}
                        \chi \join \xi
                    \end{array}
                \)
                \hfill known as the \textbf{Constructive Dilemma}
            \item
                \(
                    \begin{array}{l}
                        \varphi \\
                        \neg \varphi \\
                        \cline{1-1}
                        \psi
                    \end{array}
                \)
                \hfill known as the \textbf{Ex Falso Quodlibet}, \aka the \textbf{Principle of Explosion}
            \item
                \(
                    \begin{array}{l}
                        \varphi \leftrightarrow \psi \\
                        \varphi \\
                        \cline{1-1}
                        \psi
                    \end{array}
                \)
                \hfill known as the \textbf{Ex Falso Quodlibet}, \aka the \textbf{Principle of Explosion}
        \end{enumerate}
    \item
        Imagine a universe of discourse consisting of the collection of sentient humanoid beings
        (\eg people, humans, androids) in the year \(2029\) in Japan.
        Now, consider the following facts:

        \begin{center}
            \begin{minipage}{0.45\linewidth}
                \begin{enumerate}
                    \item[I.]
                        Every ghost in a shell is an android.
                    \item[II.]
                        Some androids are ghosts in shells.
                    \item[III.]
                        All humans are people.
                \end{enumerate}
            \end{minipage}%
            \begin{minipage}{0.45\linewidth}
                \begin{enumerate}
                    \item[IV.]
                        No androids are people.
                    \item[V.]
                        Major Kusanagi (\kanji{草薙素子}) is an android.
                    \item[VI.]
                        Togusa (\kana{トグサ}) is a human.
                \end{enumerate}
            \end{minipage}
        \end{center}

        \begin{enumerate}
            \item\label{prob:2a}
                Translate these six statements into the first-order logic by defining appropriate predicates.
            \item
                Translate each of the following English sentences into the first-order logic
                using the definitions made in \hyperref[prob:2a]{subproblem~2(a)}
                and then determine whether or not they follow from the facts given.
                If you think they do follow, then prove the argument is valid.
                Otherwise, provide an informal argument of invalidity.
                \begin{center}
                    \begin{minipage}{0.5\linewidth}
                        \begin{enumerate}
                            \item
                                Major Kusanagi is a person.
                            \item
                                Someone is a ghost in a shell.
                            \item
                                Togusa is an android if he is a ghost in a shell.
                        \end{enumerate}
                    \end{minipage}%
                    \begin{minipage}{0.5\linewidth}
                        \begin{enumerate}
                            \item
                                No ghost in a shell is a person.
                            \item
                                There is a human who is not an android.
                        \end{enumerate}
                    \end{minipage}
                \end{center}
        \end{enumerate}
\end{enumerate}

% ALGORITHM
% \begin{algorithm}[caption={placeholder}, label={alg:placeholder}]
%     input:  int list $\brackets*{x_1, \dots x_n}$ with $n \geq 2$
%     output: int list
%     begin
%         for $x$ in $\brackets*{x_1, \dots x_n}$:
%             do something
%         return result
%     end
% \end{algorithm}
\end{document}
