% define new conditionals: \if<conditional>
\newif\ifbook  % whether or not to use book styles
\newif\ifdate  % whether or not to include the current date header
\newif\ifalgorithms  % whether or not to define the algorithm environment
\newif\ifdaggerfootnotes  % whether or not to have fnsymbol footnotes or numerical ones

% set the conditionals: \<conditional>true / \<conditional>false
\bookfalse
\datefalse
\algorithmsfalse
\daggerfootnotestrue

\ifbook
    \documentclass[letterpaper]{book}
    \usepackage{arydshln, chngcntr}
\else
    \documentclass[letterpaper]{article}
\fi

% math
\usepackage{amsmath, amsfonts, amssymb, amstext, amscd, amsthm, mathrsfs, mathtools, xfrac}
% fonts
\usepackage{bbm, CJKutf8, caption, dsfont, marvosym, stmaryrd}
% tables
\usepackage{booktabs, colortbl, makecell}
% colors
\usepackage{color, soul, xcolor}
% references
\usepackage{xr-hyper, hyperref, url}
% figures
\usepackage{graphicx, float, subcaption, tikz}
% headers and footers
\usepackage{fancyhdr, lastpage}
% miscellaneous
\usepackage{enumerate, ifthen, lipsum, listings, makeidx, parskip, ulem, verbatim, xargs}
\usepackage[nodayofweek]{datetime}

\usepackage[left=2cm,top=2cm,right=2cm,bottom=2cm,bindingoffset=0cm]{geometry}
\usepackage[group-separator={,},group-minimum-digits={3}]{siunitx}
\usepackage[shortlabels]{enumitem}
\setlist[enumerate]{topsep=0ex,itemsep=0ex,partopsep=1ex,parsep=1ex}
\setlist[itemize]{topsep=0ex,itemsep=0ex,partopsep=1ex,parsep=1ex}

\usepackage[math]{cellspace}
\cellspacetoplimit 1pt
\cellspacebottomlimit 1pt

\definecolor{gruvred}{HTML}{CC214D}
\definecolor{gruvorange}{HTML}{D65D0E}
\definecolor{gruvaqua}{HTML}{689D6A}
\definecolor{gruvpurple}{HTML}{B16286}
\definecolor{colorblack}{HTML}{252422}
\definecolor{colorgrey}{HTML}{f4efef}
\definecolor{colorblue}{HTML}{045275}
\definecolor{colorteal}{HTML}{089099}
\definecolor{colorgreen}{HTML}{7ccba2}
\definecolor{coloryellow}{HTML}{ffc61e}  % fcde9c % ffc61e  % b8860b
\definecolor{colororange}{HTML}{f0746e}
\definecolor{colorred}{HTML}{dc3977}
\definecolor{colorpurple}{HTML}{7c1d6f}

\hypersetup{
    colorlinks=true,
    linkcolor=gruvorange,
    citecolor=gruvaqua,
    urlcolor=gruvpurple
}

\allowdisplaybreaks
\newdateformat{verbosedate}{\ordinal{DAY} of \monthname[\THEMONTH], \THEYEAR}
\verbosedate

\pagestyle{fancy}
% \fancyfoot[C]{--~\thepage~--}
\fancyfoot[C]{\tiny \thepage\ / \pageref*{LastPage}}
\ifbook
    \fancypagestyle{plain}{%
        \fancyhead[L]{}
        \ifdate
            \fancyhead[R]{\textsc{\today}}
        \else
            \fancyhead[R]{}
        \fi
        \renewcommand{\headrulewidth}{0pt}
    }

    \let\cleardoublepage=\clearpage
\else
    \fancypagestyle{plain}{}
    \renewcommand{\headrulewidth}{0pt}
\fi

\ifdaggerfootnotes
    \renewcommand{\thefootnote}{\fnsymbol{footnote}}
\else
\fi

\delimitershortfall=-1pt
\normalem

\newlist{detail}{itemize}{2}
\setlist[detail]{label={\boldmath$\cdot$},topsep=0pt,leftmargin=*,noitemsep}

\ifalgorithms
    \newcounter{nalg}[chapter]
    \renewcommand{\thenalg}{\thechapter.\arabic{nalg}}
    \DeclareCaptionLabelFormat{algocaption}{\it Algorithm \thenalg}

    \lstnewenvironment{algorithm}[1][]
    {
        \refstepcounter{nalg}
        \captionsetup{labelformat=algocaption,labelsep=colon}
        \lstset{
            mathescape=true,
            frame=tB,
            numbers=left,
            numberstyle=\tiny,
            basicstyle=\scriptsize,
            keywordstyle=\color{black}\bfseries\em,
            keywords={,input, output, return, datatype, function, in, if, else, elif, for, foreach, while, not, begin, end, true, false, null, break, continue, let, and, or, }
            numbers=left,
            xleftmargin=.04\textwidth,
            #1
        }
    }
    {}
\else
\fi

\ifindentproofs
    % begin new proof environment
    \expandafter\let\expandafter\oldproof\csname\string\proof\endcsname
    \let\oldendproof\endproof

    \renewenvironment{proof}[1][\proofname]{%
        \ifindenttheorems
            \vspace{-\abovedisplayskip}
        \else
        \fi
        \oldproof[#1]\quote~\vspace{-\parskip}

    }{%
        %\endquote\oldendproof
        \endquote\vspace{-\parskip}\qed
    }
    % end new proof environment
\else
\fi

\ifindenttheorems
    \newtheorem{pretheorem}{Theorem}
    \newtheorem{prelemma}{Lemma}
    \newtheorem{preproposition}{Proposition}
    \newtheorem{precorollary}{Corollary}
    \newtheorem{preclaim}{Claim}
    \newtheorem{preconjecture}{Conjecture}
    \newtheorem{prejustification}{Justification}

    \newtheorem{preaxiom}{Axiom}
    \newtheorem{predefinition}{Definition}
    \newtheorem{prenotation}{Notation}
    \newtheorem{preexercise}{Exercise}
    \newtheorem{preexample}{Example}
    \newtheorem{precounterexample}{Counterexample}

    \newtheorem{preidea}{Idea}
    \newtheorem*{preremark}{Remark}
    \newtheorem*{prenote}{Note}

    % theorem
    \NewDocumentEnvironment{theorem}{O{} O{}}
        {\begin{pretheorem}[#1]~#2\quote\vspace{-0.75\parskip}}
        {\endquote\end{pretheorem}}
    % lemma
    \NewDocumentEnvironment{lemma}{O{} O{}}
        {\begin{prelemma}[#1]~#2\quote\vspace{-0.75\parskip}}
        {\endquote\end{prelemma}}
    % proposition
    \NewDocumentEnvironment{proposition}{O{} O{}}
        {\begin{preproposition}[#1]~#2\quote\vspace{-0.75\parskip}}
        {\endquote\end{preproposition}}
    % corollary
    \NewDocumentEnvironment{corollary}{O{} O{}}
        {\begin{precorollary}[#1]~#2\quote\vspace{-0.75\parskip}}
        {\endquote\end{precorollary}}
    % claim
    \NewDocumentEnvironment{claim}{O{} O{}}
        {\begin{preclaim}[#1]~#2\quote\vspace{-0.75\parskip}}
        {\endquote\end{preclaim}}
    % conjecture
    \NewDocumentEnvironment{conjecture}{O{} O{}}
        {\begin{preconjecture}[#1]~#2\quote\vspace{-0.75\parskip}}
        {\endquote\end{preconjecture}}
    % justification
    \NewDocumentEnvironment{justification}{O{} O{}}
        {\begin{prejustification}[#1]~#2\quote\vspace{-0.75\parskip}}
        {\endquote\end{prejustification}}

    % axiom
    \NewDocumentEnvironment{axiom}{O{} O{}}
        {\begin{preaxiom}[#1]~#2\quote\normalfont\vspace{-0.75\parskip}}
        {\endquote\end{preaxiom}}
    % definition
    \NewDocumentEnvironment{definition}{O{} O{}}
        {\begin{predefinition}[#1]~#2\quote\normalfont\vspace{-0.75\parskip}}
        {\endquote\end{predefinition}}
    % notation
    \NewDocumentEnvironment{notation}{O{} O{}}
        {\begin{prenotation}[#1]~#2\quote\normalfont\vspace{-0.75\parskip}}
        {\endquote\end{prenotation}}
    % exercise
    \NewDocumentEnvironment{exercise}{O{} O{}}
        {\begin{preexercise}[#1]~#2\quote\normalfont\vspace{-0.75\parskip}}
        {\endquote\end{preexercise}}
    % example
    \NewDocumentEnvironment{example}{O{} O{}}
        {\begin{preexample}[#1]~#2\quote\normalfont\vspace{-0.75\parskip}}
        {\endquote\end{preexample}}
    % counterexample
    \NewDocumentEnvironment{counterexample}{O{} O{}}
        {\begin{precounterexample}[#1]~#2\quote\normalfont\vspace{-0.75\parskip}}
        {\endquote\end{precounterexample}}

    % idea
    \NewDocumentEnvironment{idea}{O{} O{}}
        {\begin{preidea}[#1]~#2\normalfont}
        {\end{preidea}}
    % remark
    \NewDocumentEnvironment{remark}{O{} O{}}
        {\begin{preremark}[#1]~#2\normalfont}
        {\end{preremark}}
    % note
    \NewDocumentEnvironment{note}{O{} O{}}
        {\begin{prenote}[#1]~#2\normalfont}
        {\end{prenote}}
\else
    \theoremstyle{thm}% style for theorems
    \newtheorem{theorem}{Theorem}
    \newtheorem{lemma}{Lemma}
    \newtheorem{proposition}{Proposition}
    \newtheorem{corollary}{Corollary}
    \newtheorem{claim}{Claim}
    \newtheorem{conjecture}{Conjecture}
    \newtheorem{justification}{Justification}

    \theoremstyle{dfn}% style for definitions
    \newtheorem{axiom}{Axiom}
    \newtheorem{definition}{Definition}
    \newtheorem{notation}{Notation}
    \newtheorem{exercise}{Exercise}
    \newtheorem{example}{Example}
    \newtheorem{counterexample}{Counterexample}

    \theoremstyle{rmk}% style for remarks
    \newtheorem{idea}{Idea}
    \newtheorem*{remark}{Remark}
    \newtheorem*{note}{Note}
\fi

\newenvironment{case}[1][Case]
    {\textbf{#1:}\quote\vspace{-0.75\parskip}}
    {\endquote}

\def\lstlistingautorefname{Algorithm}
\def\itemautorefname{Section}
\renewcommand{\chapterautorefname}{Chapter}
\renewcommand{\sectionautorefname}{Section}
\newcommand{\pretheoremautorefname}{Theorem}
\newcommand{\preaxiomautorefname}{Axiom}
\newcommand{\prelemmaautorefname}{Lemma}
\newcommand{\prepropositionautorefname}{Proposition}
\newcommand{\precorollaryautorefname}{Corollary}
\newcommand{\preclaimautorefname}{Claim}
\newcommand{\preconjectureautorefname}{Conjecture}
\newcommand{\prejustificationautorefname}{Justification}
\newcommand{\predefinitionautorefname}{Definition}
\newcommand{\prenotationautorefname}{Notation}
\newcommand{\preexampleautorefname}{Example}
\newcommand{\precounterexampleautorefname}{Counterexample}
\newcommand{\preideaautorefname}{Idea}
\newcommand{\axiomautorefname}{Axiom}
\newcommand{\lemmaautorefname}{Lemma}
\newcommand{\propositionautorefname}{Proposition}
\newcommand{\corollaryautorefname}{Corollary}
\newcommand{\claimautorefname}{Claim}
\newcommand{\conjectureautorefname}{Conjecture}
\newcommand{\justificationautorefname}{Justification}
\newcommand{\definitionautorefname}{Definition}
\newcommand{\notationautorefname}{Notation}
\newcommand{\exampleautorefname}{Example}
\newcommand{\counterexampleautorefname}{Counterexample}
\newcommand{\ideaautorefname}{Idea}

\ifbook
    \renewcommand{\theequation}{\thechapter.\arabic{equation}}
    \renewcommand{\thepretheorem}{\thechapter.\arabic{pretheorem}}
    \renewcommand{\theprelemma}{\thechapter.\arabic{prelemma}}
    \renewcommand{\thepreproposition}{\thechapter.\arabic{preproposition}}
    \renewcommand{\theprecorollary}{\thechapter.\arabic{precorollary}}
    \renewcommand{\thepreclaim}{\thechapter.\arabic{preclaim}}
    \renewcommand{\thepreconjecture}{\thechapter.\arabic{preconjecture}}
    \renewcommand{\theprejustification}{\thechapter.\arabic{prejustification}}
    \renewcommand{\thepredefinition}{\thechapter.\arabic{predefinition}}
    \renewcommand{\theprenotation}{\thechapter.\arabic{prenotation}}
    \renewcommand{\thepreexample}{\thechapter.\arabic{preexample}}
    \renewcommand{\theprecounterexample}{\thechapter.\arabic{precounterexample}}
    \counterwithin*{equation}{chapter}
    \counterwithin*{pretheorem}{chapter}
    \counterwithin*{prelemma}{chapter}
    \counterwithin*{preproposition}{chapter}
    \counterwithin*{precorollary}{chapter}
    \counterwithin*{preclaim}{chapter}
    \counterwithin*{preconjecture}{chapter}
    \counterwithin*{prejustification}{chapter}
    \counterwithin*{predefinition}{chapter}
    \counterwithin*{prenotation}{chapter}
    \counterwithin*{preexercise}{chapter}
    \counterwithin*{preexample}{chapter}
    \counterwithin*{precounterexample}{chapter}
\else
\fi

\newcommand*{\xline}[1][3em]{\rule[0.5ex]{#1}{0.55pt}}

\newcommand{\isomorphic}{\cong}
\newcommand{\iffdefn}{~\mathrel{\vcentcolon\Leftrightarrow}~}
\newcommand{\iffbydefn}{\(\mathrel{\vcentcolon\Leftrightarrow}\)\ }
\newcommand{\niff}{\mathrel{{\ooalign{\hidewidth$\not\phantom{"}$\hidewidth\cr$\iff$}}}}
\renewcommand{\implies}{\Rightarrow}
\renewcommand{\iff}{\Leftrightarrow}
\newcommand{\proves}{\vdash}
\newcommand{\satisfies}{\models}
\renewcommand{\qedsymbol}{\sc q.e.d.}

\renewcommand{\restriction}[1]{\downharpoonright_{#1}}
\renewcommand{\leq}{\leqslant}
\renewcommand{\geq}{\geqslant}

\newcommand{\meet}{\wedge}
\newcommand{\join}{\vee}
\newcommand{\conjunct}{\wedge}
\newcommand{\disjunct}{\vee}
\newcommand{\bigmeet}{\bigwedge}
\newcommand{\bigjoin}{\bigvee}
\newcommand{\bigconjunct}{\bigwedge}
\newcommand{\bigdisjunct}{\bigvee}
\newcommand{\defn}{\coloneqq}
\newcommand{\xor}{\oplus}
\newcommand{\nand}{\uparrow}
\newcommand{\nor}{\downarrow}

\newcommand{\compose}{\circ}
\newcommand{\divides}{~|~}
\newcommand{\notdivides}{\not|~}
\newcommand{\given}{~\middle|~}
\newcommand{\suchthat}{~\middle|~}
\newcommand{\contradiction}{~\smash{\text{\raisebox{-0.6ex}{\Large \Lightning}}}~}

\newcommand{\conjugate}[1]{\overline{#1}}
\newcommand{\mean}[1]{\overline{#1}}

\newcommand*\diff{\mathop{}\!\mathrm{d}}
\newcommand{\integral}[1]{\smashoperator{\int_{#1}}}
\newcommand{\E}[1]{\mathbb{E}\sq*{#1}}
\newcommand{\Esub}[2]{\mathbb{E}_{#1}\sq*{#2}}
\newcommand{\var}[1]{\mathrm{Var}\pn*{#1}}
\newcommand{\cov}[2]{\mathrm{Cov}\pn*{#1, #2}}
\newcommand{\der}[2]{\frac{\diff{#1}}{\diff{#2}}}
\newcommand{\dern}[3]{\frac{\diff^{#3}{#1}}{\diff{#2}^{#3}}}
\newcommand{\derm}[3]{\frac{\diff^{#3}{#1}}{\diff{#2}}}
\newcommand{\prt}[2]{\frac{\partial{#1}}{\partial{#2}}}
\newcommand{\prtn}[3]{\frac{\partial^{#3}{#1}}{\partial{#2}^{#3}}}
\newcommand{\prtm}[3]{\frac{\partial^{#3}{#1}}{\partial{#2}}}
\newcommand{\modulo}[1]{~\pn{\mathrm{mod}~#1}}

\newcommand{\inj}{\hookrightarrow}
\newcommand{\injection}{\hookrightarrow}

\newcommand{\surj}{\twoheadrightarrow}
\newcommand{\surjection}{\twoheadrightarrow}

\newcommand{\bij}{\lhook\joinrel\twoheadrightarrow}
\newcommand{\bijection}{\lhook\joinrel\twoheadrightarrow}

\newcommand{\monic}{\hookrightarrow}
\newcommand{\monomorphism}{\hookrightarrow}

\newcommand{\epic}{\twoheadrightarrow}
\newcommand{\epimorphism}{\twoheadrightarrow}

\newcommand{\iso}{\lhook\joinrel\twoheadrightarrow}
\newcommand{\isomorphism}{\lhook\joinrel\twoheadrightarrow}
\newcommand{\immersion}{\looprightarrow}

\renewcommand{\O}[1]{\mathcal{O}\pn*{#1}}
\renewcommand{\P}[1]{\mathbb{P}\pn*{#1}}
\newcommand{\power}[1]{\mathcal{P}\pn*{#1}}
\newcommand{\successor}[1]{\mathcal{S}\pn*{#1}}
\newcommand{\C}{\mathbb{C}}
\newcommand{\N}{\mathbb{N}}
\newcommand{\Q}{\mathbb{Q}}
\newcommand{\R}{\mathbb{R}}
\newcommand{\Z}{\mathbb{Z}}

% these don't need {} after them since they should be followed by text
\newcommand{\cf}{\textit{c.f.},\ }
\newcommand{\eg}{\textit{e.g.},\ }
\newcommand{\ie}{\textit{i.e.},\ }
\newcommand{\aka}{\textit{a.k.a.}\ }
\newcommand{\viz}{\textit{viz.}\ }
\newcommand{\vide}{\textit{v.}\ }
\newcommand{\ifandonlyif}{\textit{iff}\ }

% these need {} after them
\newcommand{\etal}{\textit{et al.}}
\newcommand{\wff}{\textit{wff}}

\DeclareMathOperator{\lcm}{lcm}
\DeclareMathOperator*{\argmin}{arg\!\min}
\DeclareMathOperator*{\argmax}{arg\!\max}

\let\originalleft\left
\let\originalright\right
\renewcommand{\left}{\mathopen{}\mathclose\bgroup\originalleft}
\renewcommand{\right}{\aftergroup\egroup\originalright}

\newcommand{\zh}[1]{\begin{CJK}{UTF8}{gbsn}#1\end{CJK}}
\newcommand{\jp}[1]{\begin{CJK}{UTF8}{gbsn}#1\end{CJK}}

\DeclarePairedDelimiterX \inner[2]{\langle}{\rangle}{#1,#2}
\DeclarePairedDelimiter \bra{\langle}{\rvert}
\DeclarePairedDelimiter \ket{\lvert}{\rangle}
\DeclarePairedDelimiter \abs{\lvert}{\rvert}
\DeclarePairedDelimiter \cardinality{\lvert}{\rvert}
\DeclarePairedDelimiter \norm{\lVert}{\rVert}
\DeclarePairedDelimiter \set{\lbrace}{\rbrace}
\DeclarePairedDelimiter \seq{\langle}{\rangle}
\DeclarePairedDelimiter \pn{(}{)}
\DeclarePairedDelimiter \sq{[}{]}
\DeclarePairedDelimiter \curly{\lbrace}{\rbrace}
\DeclarePairedDelimiter \bracket{\langle}{\rangle}

\let\oldemptyset\emptyset
\let\emptyset\varnothing
\let\union\cup
\let\intersection\cap
\let\intersect\cap


\begin{document}

\title{Discrete Mathematics}
\author{Daniel Gonzalez Cedre}
\date{University of Notre Dame \\ Spring of 2023}
\maketitle

\setcounter{chapter}{6}
\chapter{Number Theory}

\section{Ancient Greece}

\begin{definition}[Divisibility]
    Given two integers $a, b \in \Z$,
    we say $a \div b$ \iffbydefn $\pn*{\exists k \in \Z}\pn*{ak = b}$.
    We read $a \div b$ as \emph{$a$ divides $b$}, meaning $\sfrac{b}{a} \in \Z$.
\end{definition}

\begin{lemma}[Initial object]
    If $x \in \Z$, then $1 \div x$.
\end{lemma}
\begin{proof}
    Let $x \in \Z$ and observe that $1 \cdot x = x$.
    Therefore, $1 \div x$ by definition.
\end{proof}

\begin{lemma}[Terminal object]
    If $x \in \Z$, then $x \div 0$.
\end{lemma}
\begin{proof}
    Let $x \in \Z$ and observe that $0 \cdot x = 0$.
    Therefore, $x \div 0$ by definition.
\end{proof}

\begin{lemma}[Divisibility is a Partial Order]\label{lem:divpartial}
    The following statements hold for all $a, b, c \in \Z$:
    \begin{enumerate}
        \item[\textsc{i}.]
            $a \div a$
            \begin{proof}
                Let $a \in \Z$ and observe that $1 \cdot a = a$.
                Therefore, $a \div a$ by definition.
            \end{proof}
        \item[\textsc{ii}.]
            $\pn*{\pn*{a \div b} \meet \pn*{b \div a}} \implies \abs{a} = \abs{b}$
            \begin{proof}
                Let $a, b \in Z$ and suppose $a \div b$ and $b \div a$.
                Then, there exist $k_1, k_2 \in \Z$ such that $ak_1 = b$ and $bk_2 = a$ by definition.
                But then $bk_2 = \pn*{ak_1}k_2 = a$, so $ak_1k_2 = a$, yielding $k_1k_2 = 1$.
                Since the only integers with multiplicative inverses are $1$ and $-1$,
                we have $\set*{k_1, k_2} \subseteq \set*{1, -1}$,
                so $a = b$ or $a = -b$.
                Thus, $\abs{a} = \abs{b}$.
            \end{proof}
        \item[\textsc{iii}.]
            $\pn*{\pn*{a \div b} \meet \pn*{b \div c}} \implies a \div c$
            \begin{proof}
                Let $a, b, c \in Z$ and suppose $a \div b$ and $b \div c$.
                Then, there exist $k_1, k_2 \in \Z$ such that $ak_1 = b$ and $bk_2 = c$.
                This yields $ak_1k_2 = c$.
                Since $k_1, k_2 \in \Z$,
                we observe $k_1k_2 \in \Z$ and conclude $a \divides c$ by definition.
            \end{proof}
    \end{enumerate}
    \vspace{-2ex}
\end{lemma}

\begin{lemma}[Useful facts]\label{lem:divalgebra}
    The following statements hold for all $a, b, c \in \Z$:
    \begin{enumerate}
        \item[\textsc{i}.]
            $\pn*{\pn*{a \div b} \meet \pn*{a \div c}} \implies a \div b + c$
        \item[\textsc{ii}.]
            $a \div b \implies \pn*{\forall \ell \in \Z}\pn*{a \div b\ell}$
        \item[\textsc{iii}.]
            $a \div b \implies \abs{a} \leq \abs{b}$
    \end{enumerate}
    The proofs of the above lemmata are left as exercises to the reader.
\end{lemma}

\begin{corollary}
    Given $a, b, c \in \Z$, if $a \div b$ and $a \div c$,
    then $\pn*{\forall \ell_1, \ell_2 \in \Z}\pn*{a \div \ell_1b + \ell_2c}$.
\end{corollary}
% \begin{proof}
%     Let $a, b, c \in \Z$ and suppose $a \div b$ and $a \div c$.
%     Let $\ell_1, \ell_2 \in \Z$.
%     From \autoref{lem:divalgebra}, we know $a \div b\ell_1$ and $a \div c\ell_2$,
%     implying $a \div b\ell_1 + c\ell_2$ by \autoref{lem:divalgebra}.
% \end{proof}

\begin{definition}[Primality]
    We say that a natural number $p \in \N$ is \emph{prime} \iffbydefn 
    $(p > 1)$ and $\pn*{\forall n \in \N}\pn*{n \div p \implies n \in \set*{1, p}}$.\\
    We say $n \in \N$ is \emph{composite} \iffbydefn $n$ is not prime.
\end{definition}

\begin{lemma}[Fundamental Lemma of Arithmetic]
    If $n \in \N$ and $n > 1$, then $\pn*{\exists p \in \N}\pn*{p \text{ is prime} \meet p \div n}$.
\end{lemma}
\begin{proof}
    \hl{TODO}
\end{proof}

\begin{theorem}[Fundamental Theorem of Arithmetic]
    Every natural number greater than $1$ has a \emph{unique} prime factorization.
    Formally, for every natural number $n \in \N_{\geq 2}$ greater than $1$,
    there exist \emph{unique, distinct} primes $p_1, \dots p_\ell \in \N_+$
    with \emph{unique} exponents $k_1, \dots k_\ell \in \N_+$ such that
    % $\pn*{\forall n \in \N_{\geq 2}}$
    % $\pn*{\exists! \ell \in \N_+}$
    % $\pn*{\exists! p_1, \dots p_\ell \in \N_+}$
    % $\pn*{\exists! k_1, \dots k_\ell \in \N_+}$
    \begin{enumerate}
        \item[\textsc{i.}]
            $\pn*{\forall i, j \in \set*{1, \dots \ell}}\pn*{i \neq j \implies p_i \neq p_j}$
        \item[\textsc{ii.}]
            $\pn*{\forall i \in \set*{1, \dots \ell}}\pn*{p_i \text{ is prime}}$
        \item[\textsc{iii.}]
            $n = p_1^{k_1} p_2^{k_2} \dots p_\ell^{k_\ell}$.
    \end{enumerate}
    % \[
    %     \pn*{%
    %         \pn*{\forall i, j \in \set*{1, \dots \ell}}\pn*{i \neq j \implies p_i \neq p_j}
    %         ~~\meet~~ \pn*{\forall i \in \set*{1, \dots \ell}}\pn*{p_i \text{ is prime}}
    %         ~~\meet~~ n = \prod_{i = 1}^{\ell}p_i^{k_i}
    %     }.
    % \]
\end{theorem}

\begin{theorem}[Euclid's Theorem]
    There are infinitely-many prime numbers.
\end{theorem}
\begin{proof}
    \hl{TODO}
\end{proof}

\begin{definition}[Greatest Common Divisor]
    Given two integers $a, b \in \Z$,
    we say that $g \in \Z$ is the \emph{greatest common divisor}
    (\aka \emph{greatest common factor})
    of $a$ and $b$ \iffbydefn
    \[
        \pn*{g \div a} \meet \pn*{g \div b} \meet
        \pn*{\forall h \in \Z}\pn*{\pn*{\pn*{h \div a} \meet \pn*{h \div b}} \implies h \div g}.
    \]
    Notice that, since $\pn*{\forall x}\pn*{1 \div x}$,
    every pair of integers shares a common factor.
    Since common factors of $a$ and $b$ are bounded above by $\min\set*{a, b}$,
    that means the set of all common factors of $a$ and $b$ is nonempty and bounded above,
    so it has a maximal element.
    Therefore, the greatest common divisor of any two integers always exists.
\end{definition}

\begin{definition}[Co-Primality]
    We say that two integers $a, b \in \Z$ are \emph{co-prime} \iffbydefn 
    their greatest common divisor is $1$.
\end{definition}

\begin{theorem}[Euclid's Division Theorem]
    If $a, b \in \Z$, then there exist two \emph{unique} integers $q, r \in \Z$ such that
    \[
        a = bq + r
        \text{ and } 0 \leq r < b.
    \]
    Here, $q$ is called the \emph{quotient} when $a$ is divided by $b$,
    and $r$ is the \emph{remainder}, as illustrated by
    $\sfrac{a}{b} = q + \sfrac{r}{b}$.
\end{theorem}

\begin{algorithm}[Euclid's Division Algorithm]\label{alg:division}
    We can find the greatest common divisor of two integers by recursively computing
    \begin{align*}
        \gcd\pn*{a, 0} &\defn a \\
        \gcd\pn*{a, b} &\defn \gcd\pn*{b, r}
        \begin{aligned}[t]
            \text{ where } &a = bq + r\\
            \text{ and } &0 \leq r < b\\
            \text{ and } &q, r \in \Z.
        \end{aligned}
    \end{align*}
    This algorithm correctly computes the greatest common divisor of two arbitrary integers.
\end{algorithm}

\newpage

\section{Modular Arithmetic}

\begin{definition}[Modular Congruence]
    Let $m \in \N_+$ and let $x, y \in \Z$.
    We say that $\modulo{x}{y}{m} \iffdefn m \divides x - y$.
    We read the sentence $\congruent{x}{y}{m}$ in English as
    ``$x$ is congruent to $y$ modulo $m$.''
    This expresses the idea that $x$ and $y$ have the \emph{same remainder}
    after division by $m$, as we can see below.
    \begin{align*}
        \begin{rcases}
            x = q_xm + r\\
            y = q_ym + r
        \end{rcases}
            &\iff x - y = \pn*{q_xm + r} - \pn*{q_ym + r} \\
            &\iff x - y = \pn*{q_x - q_y}m + (r - r) \\
            &\iff x - y = \pn*{q_x - q_y}m \\
            &\iff m \divides x - y
    \end{align*}
\end{definition}

\begin{exercise}
    Let $m \in \N_+$ and $w, x, y, z \in \Z$.
    The following are some useful facts about modular congruence.
    \begin{enumerate}
        \item[\textsc{i.}]
            $\modulo{x}{y}{m} \implies \modulo{x + z}{y + z}{m}$.
        \item[\textsc{ii.}]
            $\pn*{\pn*{\modulo{w}{z}{m}} \meet \pn*{\modulo{x}{y}{m}}}
                \implies \modulo{wx}{yz}{m}$.
    \end{enumerate}
\end{exercise}

\begin{theorem}[Modular Congruence is an Equivalence Relation]
    Let $m \in \N_+$ and $x, y, z \in \Z$.
    The following are true.
    \begin{enumerate}
        \item[\textsc{i.}]
            $\modulo{x}{x}{m}$
        \item[\textsc{ii.}]
            $\modulo{x}{y}{m} \implies \modulo{y}{x}{m}$
        \item[\textsc{iii.}]
            $\pn*{\pn*{\modulo{x}{y}{m}} \meet \pn*{\modulo{y}{z}{m}}}
                \implies \modulo{x}{z}{m}$
    \end{enumerate}
\end{theorem}

\begin{definition}[Modular Residue Classes]
    Let $m \in \N_+$ and let $a \in \Z$.
    The set of solutions to the \emph{linear congruence} $\modulo{x}{a}{m}$ is denoted by
    \[
        [a]_m \defn \set*{x \in \Z \suchthat \modulo{x}{a}{m}}.
    \]
    Each of these is known as an \emph{equivalence class} of \emph{residues} modulo $m$,
    indicating that all the integers in that class have remainder congruent to $a$
    after division by $m$.
\end{definition}

\begin{definition}[Cyclic Groups]
    Let $m \in \N_+$.
    We define the \emph{modular group} (\aka the \emph{cyclic group}) of size $m$ by
    \[
        \faktor{\Z}{m\Z} \defn \set*{[x]_m \suchthat x \in \Z},
    \]
    and we define \emph{addition}
    % $+: \pn*{\faktor{\Z}{m\Z} \times \faktor{\Z}{m\Z}} \to \faktor{\Z}{m\Z}$
    and \emph{multiplication}
    % $\cdot: \pn*{\faktor{\Z}{m\Z} \times \faktor{\Z}{m\Z}} \to \faktor{\Z}{m\Z}$
    on it by
    \begin{align*}
        [x]_m + [y]_m &\defn [x + y]_m \\
        [x]_m \cdot [y]_m &\defn [xy]_m.
    \end{align*}
\end{definition}

\end{document}
