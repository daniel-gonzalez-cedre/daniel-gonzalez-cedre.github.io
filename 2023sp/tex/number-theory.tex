% define new conditionals: \if<conditional>
\newif\ifbook  % whether or not to use book styles
\newif\ifdate  % whether or not to include the current date header
\newif\ifalgorithms  % whether or not to define the algorithm environment
\newif\ifindentproofs  % whether or not to hang-indent proof environments
\newif\ifindenttheorems  % whether or not to hang-indent theorem environments

% set the conditionals: \<conditional>true / \<conditional>false
\bookfalse
\datefalse
\algorithmsfalse
\indentproofstrue
\indenttheoremstrue

\ifbook
    \documentclass[letterpaper]{book}
    \usepackage{arydshln, chngcntr}
\else
    \documentclass[letterpaper]{article}
\fi

% math
\usepackage{amsmath, amsfonts, amssymb, amstext, amscd, amsthm, mathrsfs, mathtools, xfrac}
% fonts
\usepackage{bbm, CJKutf8, caption, dsfont, marvosym, stmaryrd}
% tables
\usepackage{booktabs, colortbl, makecell}
% colors
\usepackage{color, soul, xcolor}
% references
\usepackage{hyperref, xr-hyper, url}
% figures
\usepackage{graphicx, float, tikz}
% headers and footers
\usepackage{fancyhdr, lastpage}
% miscellaneous
\usepackage{enumerate, ifthen, lipsum, listings, makeidx, parskip, verbatim, xargs}
\usepackage[nodayofweek]{datetime}

\usepackage[left=2cm,top=2cm,right=2cm,bottom=2cm,bindingoffset=0cm]{geometry}
\usepackage[group-separator={,},group-minimum-digits={3}]{siunitx}
\usepackage[shortlabels]{enumitem}
\usepackage[math]{cellspace}
\cellspacetoplimit 1pt
\cellspacebottomlimit 1pt

\definecolor{gruvred}{HTML}{CC214D}
\definecolor{gruvorange}{HTML}{D65D0E}
\definecolor{gruvaqua}{HTML}{689D6A}
\definecolor{gruvpurple}{HTML}{B16286}

\hypersetup{
    colorlinks=true,
    linkcolor=gruvorange,
    citecolor=gruvaqua,
    urlcolor=gruvpurple
}

\allowdisplaybreaks
\newdateformat{verbosedate}{\ordinal{DAY} of \monthname[\THEMONTH], \THEYEAR}
\verbosedate

\pagestyle{fancy}
% \fancyfoot[C]{--~\thepage~--}
\fancyfoot[C]{\tiny \thepage\ / \pageref*{LastPage}}
\ifbook
    \fancypagestyle{plain}{%
        \fancyhead[L]{}
        \ifdate
            \fancyhead[R]{\textsc{\today}}
        \else
            \fancyhead[R]{}
        \fi
        \renewcommand{\headrulewidth}{0pt}
    }

    \let\cleardoublepage=\clearpage
\else
    \fancypagestyle{plain}{}
    \renewcommand{\headrulewidth}{0pt}
\fi

\delimitershortfall=-1pt

\newlist{detail}{itemize}{2}
\setlist[detail]{label={\boldmath$\cdot$},topsep=0pt,leftmargin=*,noitemsep}

\ifalgorithms
    \newcounter{nalg}[chapter]
    \renewcommand{\thenalg}{\thechapter.\arabic{nalg}}
    \DeclareCaptionLabelFormat{algocaption}{\it Algorithm \thenalg}

    \lstnewenvironment{algorithm}[1][]
    {
        \refstepcounter{nalg}
        \captionsetup{labelformat=algocaption,labelsep=colon}
        \lstset{
            mathescape=true,
            frame=tB,
            numbers=left,
            numberstyle=\tiny,
            basicstyle=\scriptsize,
            keywordstyle=\color{black}\bfseries\em,
            keywords={,input, output, return, datatype, function, in, if, else, elif, for, foreach, while, not, begin, end, true, false, null, break, continue, let, and, or, }
            numbers=left,
            xleftmargin=.04\textwidth,
            #1
        }
    }
    {}
\else
\fi

% new proof environment
\expandafter\let\expandafter\oldproof\csname\string\proof\endcsname
\let\oldendproof\endproof
\renewenvironment{proof}[1][\proofname]{%
    \vspace{-0.5\parskip}%
    \oldproof[#1]
}{%
    ~\\\qed
}

\newtheoremstyle{cedretheorem} % name
    {1ex}  % space above
    {-1ex}  % space below
    {\itshape} % body font
    {}  % indent amount
    {\bfseries}  % theorem head font
    {.\\}  % punctuation after theorem head
    {.5em}  % space after theorem head
    {}  % theorem head spec (can be left empty, meaning ‘normal’)

\newtheoremstyle{cedredefinition} % name
    {1ex}  % space above
    {-1ex}  % space below
    {} % body font
    {}  % indent amount
    {\bfseries}  % theorem head font
    {.\\}  % punctuation after theorem head
    {.5em}  % space after theorem head
    {}  % theorem head spec (can be left empty, meaning ‘normal’)

\newtheoremstyle{cedrenote} % name
    {}  % space above
    {}  % space below
    {} % body font
    {}  % indent amount
    {\bfseries}  % theorem head font
    {.}  % punctuation after theorem head
    {.5em}  % space after theorem head
    {}  % theorem head spec (can be left empty, meaning ‘normal’)

% style for theorems
\theoremstyle{cedretheorem}
\newtheorem{theorem}{Theorem}
\newtheorem{lemma}{Lemma}[theorem]
\newtheorem{proposition}{Proposition}[theorem]
\newtheorem{corollary}{Corollary}[theorem]
\newtheorem{conjecture}{Conjecture}[theorem]
\newtheorem*{claim}{Claim}
\newtheorem*{justification}{Justification}

% style for definitions
\theoremstyle{cedredefinition}
\newtheorem{axiom}{Axiom}
\newtheorem{definition}{Definition}
\newtheorem{notation}{Notation}[definition]
\newtheorem{exercise}{Exercise}[definition]
\newtheorem{example}{Example}[definition]
\newtheorem*{counterexample}{Counterexample}

% style for notes
\theoremstyle{cedrenote}
\newtheorem{idea}{Idea}[definition]
\newtheorem*{remark}{Remark}
\newtheorem*{note}{Note}

\newenvironment{case}[1][Case]
    {\quote\textbf{#1:}~\\}
    {\endquote}

\def\lstlistingautorefname{Algorithm}
\def\itemautorefname{Section}
\renewcommand{\chapterautorefname}{Chapter}
\renewcommand{\sectionautorefname}{Section}
\renewcommand{\theoremautorefname}{Theorem}
\newcommand{\axiomautorefname}{Axiom}
\newcommand{\lemmaautorefname}{Lemma}
\newcommand{\propositionautorefname}{Proposition}
\newcommand{\corollaryautorefname}{Corollary}
\newcommand{\claimautorefname}{Claim}
\newcommand{\conjectureautorefname}{Conjecture}
\newcommand{\justificationautorefname}{Justification}
\newcommand{\definitionautorefname}{Definition}
\newcommand{\notationautorefname}{Notation}
\newcommand{\exampleautorefname}{Example}
\newcommand{\counterexampleautorefname}{Counterexample}
\newcommand{\ideaautorefname}{Idea}

\ifbook
    \renewcommand{\theequation}{\thechapter.\arabic{equation}}
    \renewcommand{\thetheorem}{\thechapter.\arabic{theorem}}
    \renewcommand{\thelemma}{\thechapter.\arabic{lemma}}
    \renewcommand{\theproposition}{\thechapter.\arabic{proposition}}
    \renewcommand{\thecorollary}{\thechapter.\arabic{corollary}}
    \renewcommand{\theconjecture}{\thechapter.\arabic{conjecture}}
    % \renewcommand{\theclaim}{\thechapter.\arabic{claim}}
    % \renewcommand{\thejustification}{\thechapter.\arabic{justification}}
    \renewcommand{\thedefinition}{\thechapter.\arabic{definition}}
    \renewcommand{\thenotation}{\thechapter.\arabic{notation}}
    \renewcommand{\theexample}{\thechapter.\arabic{example}}
    % \renewcommand{\thecounterexample}{\thechapter.\arabic{counterexample}}
    \counterwithin*{equation}{chapter}
    \counterwithin*{theorem}{chapter}
    \counterwithin*{lemma}{chapter}
    \counterwithin*{proposition}{chapter}
    \counterwithin*{corollary}{chapter}
    \counterwithin*{conjecture}{chapter}
    % \counterwithin*{claim}{chapter}
    % \counterwithin*{justification}{chapter}
    \counterwithin*{definition}{chapter}
    \counterwithin*{notation}{chapter}
    \counterwithin*{exercise}{chapter}
    \counterwithin*{example}{chapter}
    % \counterwithin*{counterexample}{chapter}
\else
\fi

\newcommand*{\xline}[1][3em]{\rule[0.5ex]{#1}{0.55pt}}

\newcommand{\isomorphic}{\cong}
\newcommand{\iffdefn}{\mathrel{\vcentcolon\Leftrightarrow}}
\newcommand{\iffbydefn}{$\iffdefn{}$}
\newcommand{\niff}{\mathrel{{\ooalign{\hidewidth$\not\phantom{"}$\hidewidth\cr$\iff$}}}}
\renewcommand{\implies}{~\Rightarrow~}
\renewcommand{\iff}{~\Leftrightarrow~}
\renewcommand{\restriction}[1]{\downharpoonright_{#1}}
\renewcommand{\qedsymbol}{\sc q.e.d.}
\renewcommand{\leq}{\leqslant}
\renewcommand{\geq}{\geqslant}

\newcommand{\meet}{\wedge}
\newcommand{\join}{\vee}
\newcommand{\conjunct}{\wedge}
\newcommand{\disjunct}{\vee}
\newcommand{\defn}{\coloneqq}
\newcommand{\xor}{\oplus}
\newcommand{\nand}{\uparrow}
\newcommand{\nor}{\downarrow}

\newcommand{\compose}{\circ}
\newcommand{\divides}{~|~}
\newcommand{\notdivides}{\not|~}
\newcommand{\given}{~\middle|~}
\newcommand{\suchthat}{~\middle|~}
\newcommand{\contradiction}{~\smash{\text{\Large \Lightning}}~}

\newcommand{\conjugate}[1]{\overline{#1}}
\newcommand{\mean}[1]{\overline{#1}}

\newcommand*\diff{\mathop{}\!\mathrm{d}}
\newcommand{\integral}[1]{\smashoperator{\int_{#1}}}
\newcommand{\E}[1]{\mathbb{E}\crochets*{#1}}
\newcommand{\Esub}[2]{\mathbb{E}_{#1}\crochets*{#2}}
\newcommand{\var}[1]{\mathrm{Var}\parens*{#1}}
\newcommand{\cov}[2]{\mathrm{Cov}\parens*{#1, #2}}
\newcommand{\der}[2]{\frac{\diff{#1}}{\diff{#2}}}
\newcommand{\dern}[3]{\frac{\diff^{#3}{#1}}{\diff{#2}^{#3}}}
\newcommand{\derm}[3]{\frac{\diff^{#3}{#1}}{\diff{#2}}}
\newcommand{\prt}[2]{\frac{\partial{#1}}{\partial{#2}}}
\newcommand{\prtn}[3]{\frac{\partial^{#3}{#1}}{\partial{#2}^{#3}}}
\newcommand{\prtm}[3]{\frac{\partial^{#3}{#1}}{\partial{#2}}}
\newcommand{\modulo}[1]{~\parens{\mathrm{mod}~#1}}

\newcommand{\inj}{\hookrightarrow}
\newcommand{\injection}{\hookrightarrow}

\newcommand{\surj}{\twoheadrightarrow}
\newcommand{\surjection}{\twoheadrightarrow}

\newcommand{\bij}{\lhook\joinrel\twoheadrightarrow}
\newcommand{\bijection}{\lhook\joinrel\twoheadrightarrow}

\newcommand{\monic}{\hookrightarrow}
\newcommand{\monomorphism}{\hookrightarrow}

\newcommand{\epic}{\twoheadrightarrow}
\newcommand{\epimorphism}{\twoheadrightarrow}

\newcommand{\iso}{\lhook\joinrel\twoheadrightarrow}
\newcommand{\isomorphism}{\lhook\joinrel\twoheadrightarrow}
\newcommand{\immersion}{\looprightarrow}

\renewcommand{\O}[1]{\mathcal{O}\parens*{#1}}
\renewcommand{\P}[1]{\mathcal{P}\parens*{#1}}
\newcommand{\C}{\mathbb{C}}
\newcommand{\N}{\mathbb{N}}
\newcommand{\Q}{\mathbb{Q}}
\newcommand{\R}{\mathbb{R}}
\newcommand{\Z}{\mathbb{Z}}

\newcommand{\century}{c.\ }
\newcommand{\ca}{\textit{ca.}\ }
\newcommand{\cf}{\textit{c.f.},\ }
\newcommand{\eg}{\textit{e.g.},\ }
\newcommand{\ie}{\textit{i.e.},\ }
\newcommand{\aka}{\textit{a.k.a.}\ }
\newcommand{\viz}{\textit{viz.}\ }
\newcommand{\vide}{\textit{v.}\ }
\newcommand{\etal}{\textit{et al.}\ }

\DeclareMathOperator{\lcm}{lcm}
\DeclareMathOperator*{\argmin}{arg\!\min}
\DeclareMathOperator*{\argmax}{arg\!\max}

\let\originalleft\left
\let\originalright\right
\renewcommand{\left}{\mathopen{}\mathclose\bgroup\originalleft}
\renewcommand{\right}{\aftergroup\egroup\originalright}

\newcommand{\zh}[1]{\begin{CJK}{UTF8}{gbsn}#1\end{CJK}}
\newcommand{\jp}[1]{\begin{CJK}{UTF8}{gbsn}#1\end{CJK}}

\DeclarePairedDelimiterX \inner[2]{\langle}{\rangle}{#1,#2}
\DeclarePairedDelimiter \bra{\langle}{\rvert}
\DeclarePairedDelimiter \ket{\lvert}{\rangle}
\DeclarePairedDelimiter \abs{\lvert}{\rvert}
\DeclarePairedDelimiter \cardinality{\lvert}{\rvert}
\DeclarePairedDelimiter \norm{\lVert}{\rVert}
\DeclarePairedDelimiter \set{\lbrace}{\rbrace}
\DeclarePairedDelimiter \seq{\langle}{\rangle}
\DeclarePairedDelimiter \parens{(}{)}
\DeclarePairedDelimiter \crochets{[}{]}
\DeclarePairedDelimiter \brackets{\langle}{\rangle}

\let\oldemptyset\emptyset
\let\emptyset\varnothing
\let\union\cup
\let\intersection\cap
\let\intersect\cap


\begin{document}

\title{Discrete Mathematics}
\author{Daniel Gonzalez Cedre}
\date{University of Notre Dame \\ Spring of 2023}
\maketitle

\setcounter{chapter}{6}
\chapter{Number Theory}

\section{Ancient Greece}

\begin{definition}[Divisibility]
    Given two integers $a, b \in \Z$,
    we say $a \div b$ \iffbydefn $\pn*{\exists k \in \Z}\pn*{ak = b}$.
    We read $a \div b$ as \emph{$a$ divides $b$}, meaning $\sfrac{b}{a} \in \Z$.
\end{definition}

\begin{lemma}[Initial object]
    If $x \in \Z$, then $1 \div x$.
\end{lemma}
\begin{proof}
    Let $x \in \Z$ and observe that $1 \cdot x = x$.
    Therefore, $1 \div x$ by definition.
\end{proof}

\begin{lemma}[Terminal object]
    If $x \in \Z$, then $x \div 0$.
\end{lemma}
\begin{proof}
    Let $x \in \Z$ and observe that $0 \cdot x = 0$.
    Therefore, $x \div 0$ by definition.
\end{proof}

\begin{lemma}[Reflexivity]
    $\pn*{\forall x \in \Z}\pn*{x \div x}$.
\end{lemma}

\begin{lemma}[Anti-Symmetry]
    $\pn*{\forall x \in \Z}\pn*{x \div x}$.
\end{lemma}
\begin{proof}
    Let $x \in \Z$ and observe that $1 \cdot x = x$.
    Therefore, $x \div x$ by definition.
\end{proof}

\begin{lemma}[Divisibility is a Partial Order]\label{lem:divpartial}
    The following statements hold for all $a, b, c \in \Z$:
    \begin{enumerate}
        \item[\textsc{i}.]
            $a \div a$
            \begin{proof}
                Let $a \in \Z$ and observe that $1 \cdot a = a$.
                Therefore, $a \div a$ by definition.
            \end{proof}
        \item[\textsc{ii}.]
            $\pn*{a \div b} ~\meet~ \pn*{b \div a} \implies \abs{a} = \abs{b}$
            \begin{proof}
                Let $a, b \in Z$ and suppose $a \div b$ and $b \div a$.
                Then, there exist $k_1, k_2 \in \Z$ such that $ak_1 = b$ and $bk_2 = a$ by definition.
                But then $bk_2 = \pn*{ak_1}k_2 = a$, so $ak_1k_2 = a$, yielding $k_1k_2 = 1$.
                Since the only integers with multiplicative inverses are $1$ and $-1$,
                we have $\set*{k_1, k_2} \subseteq \set*{1, -1}$,
                so $a = b$ or $a = -b$.
                Thus, $\abs{a} = \abs{b}$.
            \end{proof}
        \item[\textsc{iii}.]
            $\pn*{a \div b ~\meet~ b \div c} \implies a \div c$
            \begin{proof}
                Let $a, b, c \in Z$ and suppose $a \div b$ and $b \div c$.
                Then, there exist $k_1, k_2 \in \Z$ such that $ak_1 = b$ and $bk_2 = c$.
                This yields $ak_1k_2 = c$.
                Since $k_1, k_2 \in \Z$,
                we observe $k_1k_2 \in \Z$ and conclude $a \divides c$ by definition.
            \end{proof}
    \end{enumerate}
\end{lemma}

\newpage

\begin{lemma}[Useful facts]\label{lem:divalgebra}
    The following statements hold for all $a, b, c \in \Z$:
    \begin{enumerate}
        \item[\textsc{i}.]
            $\pn*{a \div b ~\meet~ a \div c} \implies a \div b + c$
        \item[\textsc{ii}.]
            $a \div b \implies \pn*{\forall \ell \in \Z}\pn*{a \div b\ell}$
        \item[\textsc{iii}.]
            $a \div b \implies \abs{a} \leq \abs{b}$
    \end{enumerate}
    The proofs of the above lemmata are left as exercises to the reader.
\end{lemma}

\begin{corollary}
    Given $a, b, c \in \Z$, if $a \div b$ and $a \div c$,
    then $\pn*{\forall \ell_1, \ell_2 \in \Z}\pn*{a \div \ell_1b + \ell_2c}$.
\end{corollary}
\begin{proof}
    Let $a, b, c \in \Z$ and suppose $a \div b$ and $a \div c$.
    Let $\ell_1, \ell_2 \in \Z$.
    From \autoref{lem:divalgebra}, we know $a \div b\ell_1$ and $a \div c\ell_2$,
    implying $a \div b\ell_1 + c\ell_2$ by \autoref{lem:divalgebra}.
\end{proof}

\begin{definition}[Prime Numbers]
    We say that a natural number $p \in \N$ is \emph{prime} \iffbydefn 
    $(p > 1)$ and $\pn*{\forall n \in \N}\pn*{n \div p \implies n \in \set*{1, p}}$.\\
    We say $n \in \N$ is \emph{composite} \iffbydefn $n$ is not prime.
\end{definition}

\begin{lemma}[Fundamental Lemma of Arithmetic]
    If $n \in \N$ and $n > 1$, then $\pn*{\exists p \in \N}\pn*{p \text{ is prime} \meet p \div n}$.
\end{lemma}

\begin{theorem}[Fundamental Theorem of Arithmetic]
    Every natural number greater than $1$ has a \emph{unique} prime factorization.
    Formally, for every natural number $n \in \N_{\geq 2}$ greater than $1$,
    there exist \emph{unique, distinct} primes $p_1, \dots p_\ell \in \N_+$
    with \emph{unique} exponents $k_1, \dots k_\ell \in \N_+$ such that
    % $\pn*{\forall n \in \N_{\geq 2}}$
    % $\pn*{\exists! \ell \in \N_+}$
    % $\pn*{\exists! p_1, \dots p_\ell \in \N_+}$
    % $\pn*{\exists! k_1, \dots k_\ell \in \N_+}$
    \begin{enumerate}
        \item[\textsc{i.}]
            $\pn*{\forall i, j \in \set*{1, \dots \ell}}\pn*{i \neq j \implies p_i \neq p_j}$
        \item[\textsc{ii.}]
            $\pn*{\forall i \in \set*{1, \dots \ell}}\pn*{p_i \text{ is prime}}$
        \item[\textsc{iii.}]
            $n = p_1^{k_1} p_2^{k_2} \dots p_\ell^{k_\ell}$.
    \end{enumerate}
    % \[
    %     \pn*{%
    %         \pn*{\forall i, j \in \set*{1, \dots \ell}}\pn*{i \neq j \implies p_i \neq p_j}
    %         ~~\meet~~ \pn*{\forall i \in \set*{1, \dots \ell}}\pn*{p_i \text{ is prime}}
    %         ~~\meet~~ n = \prod_{i = 1}^{\ell}p_i^{k_i}
    %     }.
    % \]
\end{theorem}

\begin{theorem}[Euclid's Theorem]
    There are infinitely-many prime numbers.
\end{theorem}

\end{document}
