% define new conditionals: \if<conditional>
\newif\ifbook  % whether or not to use book styles
\newif\ifdate  % whether or not to include the current date header
\newif\ifalgorithms  % whether or not to define the algorithm environment
\newif\ifdaggerfootnotes  % whether or not to have fnsymbol footnotes or numerical ones

% set the conditionals: \<conditional>true / \<conditional>false
\bookfalse
\datefalse
\algorithmsfalse
\daggerfootnotestrue

\ifbook
    \documentclass[letterpaper]{book}
    \usepackage{arydshln, chngcntr}
\else
    \documentclass[letterpaper]{article}
\fi

% math
\usepackage{amsmath, amsfonts, amssymb, amstext, amscd, amsthm, mathrsfs, mathtools, xfrac}
% fonts
\usepackage{bbm, CJKutf8, caption, dsfont, marvosym, stmaryrd}
% tables
\usepackage{booktabs, colortbl, makecell}
% colors
\usepackage{color, soul, xcolor}
% references
\usepackage{xr-hyper, hyperref, url}
% figures
\usepackage{graphicx, float, subcaption, tikz}
% headers and footers
\usepackage{fancyhdr, lastpage}
% miscellaneous
\usepackage{enumerate, ifthen, lipsum, listings, makeidx, parskip, ulem, verbatim, xargs}
\usepackage[nodayofweek]{datetime}

\usepackage[left=2cm,top=2cm,right=2cm,bottom=2cm,bindingoffset=0cm]{geometry}
\usepackage[group-separator={,},group-minimum-digits={3}]{siunitx}
\usepackage[shortlabels]{enumitem}
\setlist[enumerate]{topsep=0ex,itemsep=0ex,partopsep=1ex,parsep=1ex}
\setlist[itemize]{topsep=0ex,itemsep=0ex,partopsep=1ex,parsep=1ex}

\usepackage[math]{cellspace}
\cellspacetoplimit 1pt
\cellspacebottomlimit 1pt

\definecolor{gruvred}{HTML}{CC214D}
\definecolor{gruvorange}{HTML}{D65D0E}
\definecolor{gruvaqua}{HTML}{689D6A}
\definecolor{gruvpurple}{HTML}{B16286}
\definecolor{colorblack}{HTML}{252422}
\definecolor{colorgrey}{HTML}{f4efef}
\definecolor{colorblue}{HTML}{045275}
\definecolor{colorteal}{HTML}{089099}
\definecolor{colorgreen}{HTML}{7ccba2}
\definecolor{coloryellow}{HTML}{ffc61e}  % fcde9c % ffc61e  % b8860b
\definecolor{colororange}{HTML}{f0746e}
\definecolor{colorred}{HTML}{dc3977}
\definecolor{colorpurple}{HTML}{7c1d6f}

\hypersetup{
    colorlinks=true,
    linkcolor=gruvorange,
    citecolor=gruvaqua,
    urlcolor=gruvpurple
}

\allowdisplaybreaks
\newdateformat{verbosedate}{\ordinal{DAY} of \monthname[\THEMONTH], \THEYEAR}
\verbosedate

\pagestyle{fancy}
% \fancyfoot[C]{--~\thepage~--}
\fancyfoot[C]{\tiny \thepage\ / \pageref*{LastPage}}
\ifbook
    \fancypagestyle{plain}{%
        \fancyhead[L]{}
        \ifdate
            \fancyhead[R]{\textsc{\today}}
        \else
            \fancyhead[R]{}
        \fi
        \renewcommand{\headrulewidth}{0pt}
    }

    \let\cleardoublepage=\clearpage
\else
    \fancypagestyle{plain}{}
    \renewcommand{\headrulewidth}{0pt}
\fi

\ifdaggerfootnotes
    \renewcommand{\thefootnote}{\fnsymbol{footnote}}
\else
\fi

\delimitershortfall=-1pt
\normalem

\newlist{detail}{itemize}{2}
\setlist[detail]{label={\boldmath$\cdot$},topsep=0pt,leftmargin=*,noitemsep}

\ifalgorithms
    \newcounter{nalg}[chapter]
    \renewcommand{\thenalg}{\thechapter.\arabic{nalg}}
    \DeclareCaptionLabelFormat{algocaption}{\it Algorithm \thenalg}

    \lstnewenvironment{algorithm}[1][]
    {
        \refstepcounter{nalg}
        \captionsetup{labelformat=algocaption,labelsep=colon}
        \lstset{
            mathescape=true,
            frame=tB,
            numbers=left,
            numberstyle=\tiny,
            basicstyle=\scriptsize,
            keywordstyle=\color{black}\bfseries\em,
            keywords={,input, output, return, datatype, function, in, if, else, elif, for, foreach, while, not, begin, end, true, false, null, break, continue, let, and, or, }
            numbers=left,
            xleftmargin=.04\textwidth,
            #1
        }
    }
    {}
\else
\fi

\ifindentproofs
    % begin new proof environment
    \expandafter\let\expandafter\oldproof\csname\string\proof\endcsname
    \let\oldendproof\endproof

    \renewenvironment{proof}[1][\proofname]{%
        \ifindenttheorems
            \vspace{-\abovedisplayskip}
        \else
        \fi
        \oldproof[#1]\quote~\vspace{-\parskip}

    }{%
        %\endquote\oldendproof
        \endquote\vspace{-\parskip}\qed
    }
    % end new proof environment
\else
\fi

\ifindenttheorems
    \newtheorem{pretheorem}{Theorem}
    \newtheorem{prelemma}{Lemma}
    \newtheorem{preproposition}{Proposition}
    \newtheorem{precorollary}{Corollary}
    \newtheorem{preclaim}{Claim}
    \newtheorem{preconjecture}{Conjecture}
    \newtheorem{prejustification}{Justification}

    \newtheorem{preaxiom}{Axiom}
    \newtheorem{predefinition}{Definition}
    \newtheorem{prenotation}{Notation}
    \newtheorem{preexercise}{Exercise}
    \newtheorem{preexample}{Example}
    \newtheorem{precounterexample}{Counterexample}

    \newtheorem{preidea}{Idea}
    \newtheorem*{preremark}{Remark}
    \newtheorem*{prenote}{Note}

    % theorem
    \NewDocumentEnvironment{theorem}{O{} O{}}
        {\begin{pretheorem}[#1]~#2\quote\vspace{-0.75\parskip}}
        {\endquote\end{pretheorem}}
    % lemma
    \NewDocumentEnvironment{lemma}{O{} O{}}
        {\begin{prelemma}[#1]~#2\quote\vspace{-0.75\parskip}}
        {\endquote\end{prelemma}}
    % proposition
    \NewDocumentEnvironment{proposition}{O{} O{}}
        {\begin{preproposition}[#1]~#2\quote\vspace{-0.75\parskip}}
        {\endquote\end{preproposition}}
    % corollary
    \NewDocumentEnvironment{corollary}{O{} O{}}
        {\begin{precorollary}[#1]~#2\quote\vspace{-0.75\parskip}}
        {\endquote\end{precorollary}}
    % claim
    \NewDocumentEnvironment{claim}{O{} O{}}
        {\begin{preclaim}[#1]~#2\quote\vspace{-0.75\parskip}}
        {\endquote\end{preclaim}}
    % conjecture
    \NewDocumentEnvironment{conjecture}{O{} O{}}
        {\begin{preconjecture}[#1]~#2\quote\vspace{-0.75\parskip}}
        {\endquote\end{preconjecture}}
    % justification
    \NewDocumentEnvironment{justification}{O{} O{}}
        {\begin{prejustification}[#1]~#2\quote\vspace{-0.75\parskip}}
        {\endquote\end{prejustification}}

    % axiom
    \NewDocumentEnvironment{axiom}{O{} O{}}
        {\begin{preaxiom}[#1]~#2\quote\normalfont\vspace{-0.75\parskip}}
        {\endquote\end{preaxiom}}
    % definition
    \NewDocumentEnvironment{definition}{O{} O{}}
        {\begin{predefinition}[#1]~#2\quote\normalfont\vspace{-0.75\parskip}}
        {\endquote\end{predefinition}}
    % notation
    \NewDocumentEnvironment{notation}{O{} O{}}
        {\begin{prenotation}[#1]~#2\quote\normalfont\vspace{-0.75\parskip}}
        {\endquote\end{prenotation}}
    % exercise
    \NewDocumentEnvironment{exercise}{O{} O{}}
        {\begin{preexercise}[#1]~#2\quote\normalfont\vspace{-0.75\parskip}}
        {\endquote\end{preexercise}}
    % example
    \NewDocumentEnvironment{example}{O{} O{}}
        {\begin{preexample}[#1]~#2\quote\normalfont\vspace{-0.75\parskip}}
        {\endquote\end{preexample}}
    % counterexample
    \NewDocumentEnvironment{counterexample}{O{} O{}}
        {\begin{precounterexample}[#1]~#2\quote\normalfont\vspace{-0.75\parskip}}
        {\endquote\end{precounterexample}}

    % idea
    \NewDocumentEnvironment{idea}{O{} O{}}
        {\begin{preidea}[#1]~#2\normalfont}
        {\end{preidea}}
    % remark
    \NewDocumentEnvironment{remark}{O{} O{}}
        {\begin{preremark}[#1]~#2\normalfont}
        {\end{preremark}}
    % note
    \NewDocumentEnvironment{note}{O{} O{}}
        {\begin{prenote}[#1]~#2\normalfont}
        {\end{prenote}}
\else
    \theoremstyle{thm}% style for theorems
    \newtheorem{theorem}{Theorem}
    \newtheorem{lemma}{Lemma}
    \newtheorem{proposition}{Proposition}
    \newtheorem{corollary}{Corollary}
    \newtheorem{claim}{Claim}
    \newtheorem{conjecture}{Conjecture}
    \newtheorem{justification}{Justification}

    \theoremstyle{dfn}% style for definitions
    \newtheorem{axiom}{Axiom}
    \newtheorem{definition}{Definition}
    \newtheorem{notation}{Notation}
    \newtheorem{exercise}{Exercise}
    \newtheorem{example}{Example}
    \newtheorem{counterexample}{Counterexample}

    \theoremstyle{rmk}% style for remarks
    \newtheorem{idea}{Idea}
    \newtheorem*{remark}{Remark}
    \newtheorem*{note}{Note}
\fi

\newenvironment{case}[1][Case]
    {\textbf{#1:}\quote\vspace{-0.75\parskip}}
    {\endquote}

\def\lstlistingautorefname{Algorithm}
\def\itemautorefname{Section}
\renewcommand{\chapterautorefname}{Chapter}
\renewcommand{\sectionautorefname}{Section}
\newcommand{\pretheoremautorefname}{Theorem}
\newcommand{\preaxiomautorefname}{Axiom}
\newcommand{\prelemmaautorefname}{Lemma}
\newcommand{\prepropositionautorefname}{Proposition}
\newcommand{\precorollaryautorefname}{Corollary}
\newcommand{\preclaimautorefname}{Claim}
\newcommand{\preconjectureautorefname}{Conjecture}
\newcommand{\prejustificationautorefname}{Justification}
\newcommand{\predefinitionautorefname}{Definition}
\newcommand{\prenotationautorefname}{Notation}
\newcommand{\preexampleautorefname}{Example}
\newcommand{\precounterexampleautorefname}{Counterexample}
\newcommand{\preideaautorefname}{Idea}
\newcommand{\axiomautorefname}{Axiom}
\newcommand{\lemmaautorefname}{Lemma}
\newcommand{\propositionautorefname}{Proposition}
\newcommand{\corollaryautorefname}{Corollary}
\newcommand{\claimautorefname}{Claim}
\newcommand{\conjectureautorefname}{Conjecture}
\newcommand{\justificationautorefname}{Justification}
\newcommand{\definitionautorefname}{Definition}
\newcommand{\notationautorefname}{Notation}
\newcommand{\exampleautorefname}{Example}
\newcommand{\counterexampleautorefname}{Counterexample}
\newcommand{\ideaautorefname}{Idea}

\ifbook
    \renewcommand{\theequation}{\thechapter.\arabic{equation}}
    \renewcommand{\thepretheorem}{\thechapter.\arabic{pretheorem}}
    \renewcommand{\theprelemma}{\thechapter.\arabic{prelemma}}
    \renewcommand{\thepreproposition}{\thechapter.\arabic{preproposition}}
    \renewcommand{\theprecorollary}{\thechapter.\arabic{precorollary}}
    \renewcommand{\thepreclaim}{\thechapter.\arabic{preclaim}}
    \renewcommand{\thepreconjecture}{\thechapter.\arabic{preconjecture}}
    \renewcommand{\theprejustification}{\thechapter.\arabic{prejustification}}
    \renewcommand{\thepredefinition}{\thechapter.\arabic{predefinition}}
    \renewcommand{\theprenotation}{\thechapter.\arabic{prenotation}}
    \renewcommand{\thepreexample}{\thechapter.\arabic{preexample}}
    \renewcommand{\theprecounterexample}{\thechapter.\arabic{precounterexample}}
    \counterwithin*{equation}{chapter}
    \counterwithin*{pretheorem}{chapter}
    \counterwithin*{prelemma}{chapter}
    \counterwithin*{preproposition}{chapter}
    \counterwithin*{precorollary}{chapter}
    \counterwithin*{preclaim}{chapter}
    \counterwithin*{preconjecture}{chapter}
    \counterwithin*{prejustification}{chapter}
    \counterwithin*{predefinition}{chapter}
    \counterwithin*{prenotation}{chapter}
    \counterwithin*{preexercise}{chapter}
    \counterwithin*{preexample}{chapter}
    \counterwithin*{precounterexample}{chapter}
\else
\fi

\newcommand*{\xline}[1][3em]{\rule[0.5ex]{#1}{0.55pt}}

\newcommand{\isomorphic}{\cong}
\newcommand{\iffdefn}{~\mathrel{\vcentcolon\Leftrightarrow}~}
\newcommand{\iffbydefn}{\(\mathrel{\vcentcolon\Leftrightarrow}\)\ }
\newcommand{\niff}{\mathrel{{\ooalign{\hidewidth$\not\phantom{"}$\hidewidth\cr$\iff$}}}}
\renewcommand{\implies}{\Rightarrow}
\renewcommand{\iff}{\Leftrightarrow}
\newcommand{\proves}{\vdash}
\newcommand{\satisfies}{\models}
\renewcommand{\qedsymbol}{\sc q.e.d.}

\renewcommand{\restriction}[1]{\downharpoonright_{#1}}
\renewcommand{\leq}{\leqslant}
\renewcommand{\geq}{\geqslant}

\newcommand{\meet}{\wedge}
\newcommand{\join}{\vee}
\newcommand{\conjunct}{\wedge}
\newcommand{\disjunct}{\vee}
\newcommand{\bigmeet}{\bigwedge}
\newcommand{\bigjoin}{\bigvee}
\newcommand{\bigconjunct}{\bigwedge}
\newcommand{\bigdisjunct}{\bigvee}
\newcommand{\defn}{\coloneqq}
\newcommand{\xor}{\oplus}
\newcommand{\nand}{\uparrow}
\newcommand{\nor}{\downarrow}

\newcommand{\compose}{\circ}
\newcommand{\divides}{~|~}
\newcommand{\notdivides}{\not|~}
\newcommand{\given}{~\middle|~}
\newcommand{\suchthat}{~\middle|~}
\newcommand{\contradiction}{~\smash{\text{\raisebox{-0.6ex}{\Large \Lightning}}}~}

\newcommand{\conjugate}[1]{\overline{#1}}
\newcommand{\mean}[1]{\overline{#1}}

\newcommand*\diff{\mathop{}\!\mathrm{d}}
\newcommand{\integral}[1]{\smashoperator{\int_{#1}}}
\newcommand{\E}[1]{\mathbb{E}\sq*{#1}}
\newcommand{\Esub}[2]{\mathbb{E}_{#1}\sq*{#2}}
\newcommand{\var}[1]{\mathrm{Var}\pn*{#1}}
\newcommand{\cov}[2]{\mathrm{Cov}\pn*{#1, #2}}
\newcommand{\der}[2]{\frac{\diff{#1}}{\diff{#2}}}
\newcommand{\dern}[3]{\frac{\diff^{#3}{#1}}{\diff{#2}^{#3}}}
\newcommand{\derm}[3]{\frac{\diff^{#3}{#1}}{\diff{#2}}}
\newcommand{\prt}[2]{\frac{\partial{#1}}{\partial{#2}}}
\newcommand{\prtn}[3]{\frac{\partial^{#3}{#1}}{\partial{#2}^{#3}}}
\newcommand{\prtm}[3]{\frac{\partial^{#3}{#1}}{\partial{#2}}}
\newcommand{\modulo}[1]{~\pn{\mathrm{mod}~#1}}

\newcommand{\inj}{\hookrightarrow}
\newcommand{\injection}{\hookrightarrow}

\newcommand{\surj}{\twoheadrightarrow}
\newcommand{\surjection}{\twoheadrightarrow}

\newcommand{\bij}{\lhook\joinrel\twoheadrightarrow}
\newcommand{\bijection}{\lhook\joinrel\twoheadrightarrow}

\newcommand{\monic}{\hookrightarrow}
\newcommand{\monomorphism}{\hookrightarrow}

\newcommand{\epic}{\twoheadrightarrow}
\newcommand{\epimorphism}{\twoheadrightarrow}

\newcommand{\iso}{\lhook\joinrel\twoheadrightarrow}
\newcommand{\isomorphism}{\lhook\joinrel\twoheadrightarrow}
\newcommand{\immersion}{\looprightarrow}

\renewcommand{\O}[1]{\mathcal{O}\pn*{#1}}
\renewcommand{\P}[1]{\mathbb{P}\pn*{#1}}
\newcommand{\power}[1]{\mathcal{P}\pn*{#1}}
\newcommand{\successor}[1]{\mathcal{S}\pn*{#1}}
\newcommand{\C}{\mathbb{C}}
\newcommand{\N}{\mathbb{N}}
\newcommand{\Q}{\mathbb{Q}}
\newcommand{\R}{\mathbb{R}}
\newcommand{\Z}{\mathbb{Z}}

% these don't need {} after them since they should be followed by text
\newcommand{\cf}{\textit{c.f.},\ }
\newcommand{\eg}{\textit{e.g.},\ }
\newcommand{\ie}{\textit{i.e.},\ }
\newcommand{\aka}{\textit{a.k.a.}\ }
\newcommand{\viz}{\textit{viz.}\ }
\newcommand{\vide}{\textit{v.}\ }
\newcommand{\ifandonlyif}{\textit{iff}\ }

% these need {} after them
\newcommand{\etal}{\textit{et al.}}
\newcommand{\wff}{\textit{wff}}

\DeclareMathOperator{\lcm}{lcm}
\DeclareMathOperator*{\argmin}{arg\!\min}
\DeclareMathOperator*{\argmax}{arg\!\max}

\let\originalleft\left
\let\originalright\right
\renewcommand{\left}{\mathopen{}\mathclose\bgroup\originalleft}
\renewcommand{\right}{\aftergroup\egroup\originalright}

\newcommand{\zh}[1]{\begin{CJK}{UTF8}{gbsn}#1\end{CJK}}
\newcommand{\jp}[1]{\begin{CJK}{UTF8}{gbsn}#1\end{CJK}}

\DeclarePairedDelimiterX \inner[2]{\langle}{\rangle}{#1,#2}
\DeclarePairedDelimiter \bra{\langle}{\rvert}
\DeclarePairedDelimiter \ket{\lvert}{\rangle}
\DeclarePairedDelimiter \abs{\lvert}{\rvert}
\DeclarePairedDelimiter \cardinality{\lvert}{\rvert}
\DeclarePairedDelimiter \norm{\lVert}{\rVert}
\DeclarePairedDelimiter \set{\lbrace}{\rbrace}
\DeclarePairedDelimiter \seq{\langle}{\rangle}
\DeclarePairedDelimiter \pn{(}{)}
\DeclarePairedDelimiter \sq{[}{]}
\DeclarePairedDelimiter \curly{\lbrace}{\rbrace}
\DeclarePairedDelimiter \bracket{\langle}{\rangle}

\let\oldemptyset\emptyset
\let\emptyset\varnothing
\let\union\cup
\let\intersection\cap
\let\intersect\cap


\externaldocument{../propositional-logic/propositional-logic}

\begin{document}

\title{Discrete Mathematics}
\author{Daniel Gonzalez Cedre}
\date{University of Notre Dame \\ Spring of 2023}
\maketitle

\setcounter{chapter}{1}
\chapter{First-Order Logic}\label{chap:firstorder}

\section{Predicates \& Quantification}

\begin{definition}[Universe of Discourse]
    Our \emph{universe of discourse}, usually denoted by the letter $\Omega$,
    denotes the collection of objects under consideration.
    This specifies the objects that are admissible as inputs to predicates,
    meaning these are the objects we can actually make concrete, descriptive claims about
    using our formal language.
\end{definition}

\begin{definition}[Predicate]
    Given a universe of discourse $\Omega$ and a non-negative integer $n$,
    we say that $\varphi(x_1, \dots x_n)$ is an $n$-ary \emph{predicate}
    \iffbydefn the substitution of an object $\omega_i$ from $\Omega$ for each $x_i$ in $\varphi$
    results in a proposition $\varphi(\omega_1, \dots \omega_n)$.
    These are sometimes referred to as \emph{propositional functions}.
    Notice that if $n = 0$ then the $0$-ary predicate is simply a proposition.

    The placeholders $x_1, \dots x_n$ in $\varphi$ are called \emph{variables},
    and the process of assigning objects to these variables is called
    \emph{instantiation} or \emph{variable assignment}.
\end{definition}
\begin{example}\label{ex:predicate}
    If we choose a universe of discourse $\Omega$ consisting of some collection of people, then

    \begin{minipage}{.45\linewidth}
        \vspace{-\parskip-\abovedisplayskip}
        \begin{align*}
            \mu(x) &\defn \text{``}x \text{ is a mathematician.''}\\
            \varepsilon(x) &\defn \text{``}x \text{ loves espresso.''}
        \end{align*}
    \end{minipage}%
    \begin{minipage}{.45\linewidth}
        \vspace{-\parskip-\abovedisplayskip}
        \begin{align*}
            \gamma(x, y) &\defn \text{``}x \text{ drinks a Guinness with } y\text{.''}\\
            \alpha(x, y, z) &\defn \text{``}x\text{, }y\text{, and } z \text{ are colleagues.''}
        \end{align*}
    \end{minipage}

    are unary, binary, and ternary predicates respectively.
\end{example}

\begin{definition}[Universal Quantifier]
    Given a unary predicate $\varphi(x)$,
    the \emph{universal quantification} of the variable $x$ in $\varphi$
    is denoted by $\forall x \pn*{\varphi(x)}$
    and expresses that $\varphi(\omega)$ is true for any arbitrary $\omega$ from our universe of discourse.
\end{definition}
\begin{example}
    Using the definitions from \autoref{ex:predicate},
    we can rewrite ``Every mathematician loves espresso'' as
    $\forall x \pn*{\varepsilon(x)}$ or as $\forall x \pn*{\mu(x) \rightarrow \varepsilon(x)}$
    depending on whether our universe $\Omega$
    is the collection of all mathematicians or of all people.
\end{example}

\begin{definition}[Existential Quantifier]
    Given a unary predicate $\varphi(x)$,
    the \emph{existential quantification} of the variable $x$ in $\varphi$
    is denoted by $\exists x \pn*{\varphi(x)}$
    and expresses that $\varphi(\omega)$ is true for at least one $\omega$ from our universe of discourse.
\end{definition}
\begin{example}
    Using the definitions from \autoref{ex:predicate},
    we can rewrite ``Some mathematician loves espresso'' as
    $\exists x \pn*{\varepsilon(x)}$ or as $\exists x \pn*{\mu(x) \meet \varepsilon(x)}$
    depending on whether our universe $\Omega$
    is the collection of all mathematicians or of all people.
\end{example}

\newpage

\section{Rules of Inference}

\begin{definition}[Rule of Inference]
    A \emph{rule of inference} is a construction in the meta-language that tells us
    how we're allowed to take previous statements in our formal language and derive new statements from them.
    They essentially describe the allowable algebraic manipulations we can make to the symbols in our language,
    and in this way they function analogously to the instruction set of a computer
    (if you're familiar with that).
    Another way to think of this is that they are truth-preserving axioms for our formal system;
    they are fundamental assumptions about the semantics of our formal language
    that allow us to reinterpret the statements we formulate in it.
    They usually take one of two forms: equivalences and inferences.
    For each of these rules, there is an \emph{underlying tautology},
    which is a tautological sentence in our formal language expressing the same idea as the rule.

    An \emph{equivalence rule} would take the form $\varphi \iff \psi$,
    where $\varphi$ and $\psi$ are sentences,
    indicating that any instance of $\varphi$ can be replaced by an instance of $\psi$.
    The underlying tautology of such a rule is $p \leftrightarrow q$.
    A popular alternative notation is $\varphi \equiv \psi$.

    An \emph{inference rule} would take the form $\Gamma \implies \psi$
    % or, equivalently, $\bigmeet_{i = 1}^{n}\varphi_i \implies \psi$,
    where $\Gamma$ is a collection of sentences
    $\varphi_1, \dots \varphi_n$ in our formal language
    and $\psi$ is some other sentence,
    indicating that whenever we write down all of the sentences in $\Gamma$,
    we can then write down $\psi$ afterward.
    The underlying tautology of such a rule is
    $\varphi_1 \meet \dots \varphi_n \implies \psi$.
    A popular alternative notation is $\Gamma \proves \psi$, or $\varphi_1, \dots \varphi_n \proves \psi$,
    indicating that $\psi$ is deducible from the sentences in $\Gamma$.

    We will stick to the $\iff$ and $\implies$ notation in this course
    since it is the most popular modern notation in use among practicing mathematicians.
    However, keep in mind that certain people
    (in particular, modern computer scientists, logicians, and set/model theorists)
    may instead use the $\equiv$ and $\proves$ notation.
\end{definition}

% The rules in the two tables below encompass the rules we have for \emph{unquantified} statements in our language.
% Notice that these encompass the axioms for a Boolean (and Heyting) algebra from the previous chapter.

\begin{center}
    \vspace{-\parskip-\abovedisplayskip}
    \begin{minipage}[t]{.4\linewidth}
        \begin{table}[H]
            \centering
            \label{tab:rulesequivalence}
            \begin{tabular}{|Cc|Cc|} \hline
                \multicolumn{2}{|c|}{\thead{Equivalence Rules}} \\\hline
                \thead{Identity} & \thead{$p \meet \top \iff p$ \\ $p \join \bot \iff p$} \\ \hline
                \thead{Complement \\ (\aka Negation)} & \thead{$p \meet \neg p \iff \bot$ \\ $p \join \neg p \iff \top$} \\ \hline
                \thead{Commutativity} & \thead{$p \meet q \iff q \meet p$ \\ $p \join q \iff q \join p$} \\ \hline
                \thead{Associativity} & \thead{$p \meet (q \meet r) \iff (p \meet q) \meet r$ \\ $p \join (q \join r) \iff (p \join q) \join r$} \\ \hline
                \thead{Distributivity} & \thead{$p \conjunct (q \disjunct r) \iff (p \conjunct q) \disjunct (p \conjunct r)$ \\ $p \disjunct (q \conjunct r) \iff (p \disjunct q) \conjunct (p \disjunct r)$} \\ \hline
                \thead{Conditional \\ Disintegration} & \thead{$p \rightarrow q \iff \neg p \join q$} \\ \hline
                \thead{Biconditional \\ Disintegration} & \thead{$p \leftrightarrow q \iff (p \rightarrow q) \meet (q \rightarrow p)$} \\ \hline
            \end{tabular}
        \end{table}
    \end{minipage}%
    \begin{minipage}[t]{.07\linewidth}
        ~
    \end{minipage}%
    \begin{minipage}[t]{.53\linewidth}
        \begin{table}[H]
            \centering
            \label{tab:rulesinference}
            \begin{tabular}{|Cc|l|l|} \hline
                \multicolumn{3}{|c|}{\thead{Unquantified Inference Rules}} \\\hline
                \thead{Truth Axiom} & \thead[l]{We can always write $\top$ \\ without any prior assumptions.} & \thead[l]{$\begin{array}{c} ~ \\ \cline{1-1} \top\end{array}$} \\ \hline
                \thead{Deduction \\ -- -- -- -- -- -- -- -- -- -- \\ \aka Conditional \\ Introduction} & \thead[l]{If by assuming $p$ \\ we are able to derive $q$,\\ then we can derive $p \rightarrow q$.} & \thead[l]{$\begin{array}{c} p \implies q \\ \cline{1-1} p \rightarrow q\end{array}$} \\ \hline
                \thead{Modus Ponens} & \thead[l]{If we know $p \rightarrow q$ \\  and we have $p$,\\ then we can derive $q$.} & \thead[l]{$\begin{array}{l} p \\ p \rightarrow q \\ \cline{1-1} q \end{array}$} \\ \hline
                \thead{Modus Tollens} & \thead[l]{If we know $p \rightarrow q$ \\ and we have $\neg q$,\\ then we can derive $\neg p$.} & \thead[l]{$\begin{array}{l} \neg q \\ p \rightarrow q \\ \cline{1-1} \neg p \end{array}$} \\ \hline
                \thead{Reductio \\ ad Absurdum \\ -- -- -- -- -- -- -- -- -- -- \\ \aka Proof by \\ Contradiction} & \thead[l]{If by assuming $\neg p$ \\ we can derive both $q$ and $\neg q$,\\ then we can derive $p$.} & \thead[l]{$\begin{array}{l} \neg p \implies q \\ \neg p \implies \neg q \\ \cline{1-1} p \end{array}$} \\ \hline
            \end{tabular}
        \end{table}
    \end{minipage}
\end{center}

\begin{table}[H]
    \centering
    \label{tab:rulesquantifiers}
    \begin{tabular}{|Cc|l|l|} \hline
        \multicolumn{3}{|c|}{\thead{Quantified Inference Rules}} \\\hline
        \thead{Universal Instantiation \\ ($\forall$ Elimination)} & \thead[l]{If we know $\forall x \pn*{\varphi(x)}$, \\ then we can derive $\varphi(t)$ for any $t$.} & \thead[l]{$\begin{array}{l} \forall x \pn*{\varphi(x)} \\ \cline{1-1} \varphi(t) \text{ for any } t\end{array}$} \\ \hline
        \thead{Universal Generalization \\ ($\forall$ Introduction)} & \thead[l]{If we know $\varphi(t)$ for an arbitrary $t$, \\ then we can derive $\forall x \pn*{\varphi(x)}$.} & \thead[l]{$\begin{array}{l} \varphi(t) \text{ for an arbitrary } t \\ \cline{1-1} \forall x \pn*{\varphi(x)} \end{array}$} \\ \hline
        \thead{Existential Instantiation \\ ($\exists$ Elimination)} & \thead[l]{If we know $\exists x \pn*{\varphi(x)}$, then we can derive \\ $\varphi(t)$ for some $t$ that hasn't yet appeared.} & \thead[l]{$\begin{array}{l} \exists x \pn*{\varphi(x)} \\ \cline{1-1} \varphi(t) \text{ for some } t \text{ that hasn't yet appeared}\end{array}$} \\ \hline
        \thead{Existential Generalization \\ ($\exists$ Introduction)} & \thead[l]{If we know $\varphi(t)$ for some particular $t$, \\ then we can derive $\exists x \pn*{\varphi(x)}$.} & \thead[l]{$\begin{array}{l} \varphi(t) \text{ for some particular } t \\ \cline{1-1} \exists x \pn*{\varphi(x)} \end{array}$} \\ \hline
    \end{tabular}
\end{table}

\newpage

\begin{lemma}\label{lem:forall}
    Let $\varphi$ be a predicate.
    If we know $\forall x \pn*{\neg \varphi(x)}$, then we have $\neg \exists x \pn*{\varphi(x)}$.
\end{lemma}
\begin{proof}
    Suppose that $\forall x \pn*{\neg \varphi(x)}$.
    Assume, towards a contradiction, that $\exists x \pn*{\varphi(x)}$.
    Then, we know that $\varphi(t)$ for some $t$.
    From our first assumption, we know that $\neg \varphi(s)$ for any $s$;
    in particular, we know that $\neg \varphi(t)$.
    But then, we obtain $\varphi(t) \meet \neg \varphi(t)$, which is a contradiction. \contradiction
    Therefore, we can conclude $\neg \exists x \pn*{\varphi(x)}$.
\end{proof}

\begin{lemma}\label{lem:negexists}
    Let $\varphi$ be a predicate.
    If we know $\neg \exists x \pn*{\varphi(x)}$, then we have $\forall x \pn*{\neg \varphi(x)}$.
\end{lemma}
\begin{proof}
    Suppose that $\neg \exists x \pn*{\varphi(x)}$ and let $t$ be an arbitrary object.
    Towards a contradiction, assume that $\varphi(t)$.
    That would then mean $\exists x \pn*{\varphi(x)}$.
    However, we already assumed $\neg \exists x \pn*{\varphi(x)}$, which is absurd. \contradiction
    Therefore, we must have $\neg \varphi(t)$.
    Since $t$ was arbitrary, we can then conclude $\forall x \pn*{\neg \varphi(x)}$.
\end{proof}

\begin{lemma}\label{lem:exists}
    Let $\varphi$ be a predicate.
    If we know $\exists x \pn*{\neg \varphi(x)}$, then we have $\neg \forall x \pn*{\varphi(x)}$.
\end{lemma}
\begin{proof}
    Suppose that $\exists x \pn*{\neg \varphi(x)}$.
    Then, we know that $\neg \varphi(t)$ for some $t$.
    Assume, towards a contradiction, that $\forall x \pn*{\varphi(x)}$.
    Then, we have $\varphi(s)$ for any $s$;
    in particular, we know $\varphi(t)$.
    But then, we have $\neg \varphi(t) \meet \varphi(t)$, giving us a contradiction. \contradiction
    Therefore, we can conclude $\neg \forall x \pn*{\varphi(x)}$.
\end{proof}

\begin{lemma}\label{lem:negforall}
    Let $\varphi$ be a predicate.
    If we know $\neg \forall x \pn*{\varphi(x)}$, then we have $\exists x \pn*{\neg \varphi(x)}$.
\end{lemma}
\begin{proof}
    Suppose $\neg \forall x \pn*{\varphi(x)}$.
    Towards a contradiction, assume that $\neg \exists x \pn*{\neg \varphi(x)}$.
    Then by \autoref{lem:negexists}, we get $\forall x \pn*{\neg \neg \varphi(x)}$,
    which is equivalent to $\forall x \pn*{\varphi(x)}$.
    However, this contradicts our assumption that $\neg \forall x \pn*{\varphi(x)}$. \contradiction
    Therefore, we can conclude that $\neg \neg \exists x \pn*{\varphi(x)}$,
    which is equivalent to $\exists x \pn*{\varphi(x)}$.
\end{proof}

\begin{theorem}[Negation of Quantifiers]
    For any predicate $\varphi$, we have
    \begin{align*}
        \neg \forall x \pn*{\varphi(x)} &\iff \exists x \pn*{\neg \varphi(x))}\\
        \neg \exists x \pn*{\varphi(x)} &\iff \forall x \pn*{\neg \varphi(x))}
    \end{align*}
\end{theorem}
\begin{proof}
    This follows from \autoref{lem:forall}, \hyperref[lem:negexists]{2.2}, \hyperref[lem:exists]{2.3},
    and \hyperref[lem:negforall]{2.4}.
    The details are left as an exercise for the reader.
\end{proof}

\begin{remark}[Order of Quantification]
    Notice that the \emph{order} of quantification often matters when writing or interpreting an expression,
    meaning $\forall x \exists y \pn*{\varphi(x, y)} \niff \exists y \forall x \pn*{\varphi(x, y)}$.
    However, if two instances of the \emph{same} quantifier occur immediately adjacent to each other,
    then they will commute: $\forall x \forall y \pn*{\varphi(x, y)} \iff \forall y \forall x \pn*{\varphi(x, y)}$.
\end{remark}

\newpage

\section{Validity of Arguments}

\begin{definition}[Validity]
    Let $\varphi_1, \dots \varphi_n$ and $\psi$ be sentences in our language.
    Consider an argument of the form
    \[
    \begin{array}{ r l }
                & \varphi_1 \\
                & \vdots \\
                & \varphi_n \\
        \cline{1-2}
        \therefore & \psi
    \end{array}
    \]
    We say that such an argument is \emph{valid} \iffbydefn
    either of the following (equivalent) criteria is satisfied:
    \begin{enumerate}
        \item[I.]\label{validi}
            The sentence
            $\pn*{\varphi_1 \meet \dots \varphi_n} \rightarrow \psi$
            is a tautology.
        \item[II.]\label{validii}
            We have $\varphi_1, \dots \varphi_n \implies \psi$.
    \end{enumerate}
    Condition \hyperref[validi]{I.\ }states that \emph{any} truth assignment
    that makes $\varphi_1 \meet \dots \varphi_n$ true must also make $\psi$ true.
    Condition \hyperref[validi]{II.\ }states that,
    from the assumption that $\varphi_1, \dots \varphi_n$ all hold,
    we \emph{have a derivation}
    that $\psi$ is the case by using the rules of equivalence and inference.
\end{definition}

\begin{definition}[Invalidity]
    Conversely, we say the argument is \emph{invalid} \iffbydefn
    either of the following (equivalent) criteria is satisfied: 
    \begin{enumerate}
        \item[I.]
            The sentence
            $\pn*{\varphi_1 \meet \dots \varphi_n} \rightarrow \psi$
            is \emph{not} a tautology.
        \item[II.]
            We have $\varphi_1, \dots \varphi_n \nimplies \psi$.
    \end{enumerate}
    Condition \hyperref[validi]{I.\ }states that \emph{there is a} truth assignment
    that makes $\varphi_1 \meet \dots \varphi_n$ true and also $\psi$ false.
    Condition \hyperref[validi]{II.\ }states that,
    from the assumption that $\varphi_1, \dots \varphi_n$ all hold,
    we \emph{can not have a derivation}
    that $\psi$ is the case by using the rules of equivalence and inference.
\end{definition}

\begin{theorem}[Church's Theorem]
    Validity of arguments in the first-order logic is undecidable.

    This means that there is no Turing machine that,
    given an arbitrary sentence in the first-order logic,
    is able to correctly decide whether or not it has a proof in a finite amount of time.
\end{theorem}

\begin{definition}[Term]
    Given a universe of discourse $\Omega$,
    we say that $x$ is a \emph{term} if it is either a specific member of $\Omega$
    or a variable symbol denoting a not-yet-determined member of $\Omega$.
    In class, we have been referring to these as \emph{objects}.
\end{definition}

\begin{definition}
    In our language, given a universe of discourse $\Omega$, we have the following symbols:
    \begin{list}{$\cdot$}{}
        \item
            \emph{Constant symbols:} terms denoting specific members of $\Omega$
            (\eg $7$, Socrates, $\R$, $\varnothing$).
        \item
            \emph{Variable symbols:} terms denoting members of $\Omega$ that are not-yet determined,
            usually written with a single letter from the Latin or Greek alphabets
            with or without natural number subscripts
            (\eg $x_1$, $\varphi_{23}$, $y$, $\beta$, $A$, $\Gamma$).
        \item
            \emph{Predicate symbols:} $n$-ary predicates $\psi\pn*{x_1, \dots x_n}$
            taking the terms $x_1, \dots x_n$ as arguments and returning a proposition once all variables are eliminated
            (\eg $=$, $\leq$, $\in$, $\subseteq$, $\isomorphic$).
        \item
            \emph{Functional symbols:} we haven't encountered any of these yet;
            they work similarly to predicates in that they take $n$ terms as arguments,
            but they return a specific object of $\Omega$ as their output
            once all free variables are eliminated
            (\eg $+$, $\times$, $\gcd(x, y)$, $\abs{x}$, ``$x$ is the father of $y$'').
        \item
            \emph{Grouping symbols:} the simple curved parentheses ( and ), used for grouping expressions.
        \item
            \emph{Logical connectives:} the unary connective $\neg$
            and the binary connectives $\meet, \join, \rightarrow, \leftrightarrow$,
            which form a Boolean algebra along with $\top$ and $\bot$.
        \item
            \emph{Quantifier symbols:} the universal quantifier $\forall$
            and the existential quantifier $\exists$.
    \end{list}
    Once we specify some basic predicate and functional symbols,
    we can then define new predicates and functionals on the basis of the already-existing ones
    as the need arises.
\end{definition}

\begin{definition}[Atomic Formul{\ae}]
    We say $\varphi$ is an \emph{atomic formula} \iffbydefn
    we have $\varphi \iff \psi\pn*{x_1, \dots x_n}$ for some $n$-ary predicate $\psi$
    with $x_1, \dots x_n$ being terms.
    Note that atomic formul{\ae} may or may not contain free variables.
\end{definition}

\begin{definition}[Well-Formed Formul{\ae}]
    We define \emph{well-formed formul{\ae}} recursively (\ie inductively) using the atomic formul{\ae} as our basis.
    We say that $\varphi$ is a \emph{well-formed formula} (abbreviated {\wff}) \iffbydefn
    any of the following holds:
    \begin{enumerate}
        \item[I.]
            $\varphi$ is an atomic formula
        \item[II.]
            $\varphi \iff \neg \pn*{\psi}$, where $\psi$ is a {\wff}
        \item[III.]
            $\varphi \iff \pn*{\psi} \meet \pn*{\chi}$, where $\psi$ is a {\wff}
        \item[IV.]
            $\varphi \iff \pn*{\psi} \join \pn*{\chi}$, where $\psi$ and $\chi$ are {\wff}
        \item[V.]
            $\varphi \iff \pn*{\psi} \rightarrow \pn*{\chi}$, where $\psi$ and $\chi$ are {\wff}
        \item[VI.]
            $\varphi \iff \pn*{\psi} \leftrightarrow \pn*{\chi}$, where $\psi$ and $\chi$ are {\wff}
        \item[VII.]
            $\varphi \iff \forall x \pn*{\psi}$, where $\psi$ is {\wff}
        \item[VIII.]
            $\varphi \iff \exists x \pn*{\psi}$, where $\psi$ is {\wff}
    \end{enumerate}
\end{definition}

\begin{definition}[Sentence]
    We say that a {\wff} $\varphi$ is a \emph{sentence} \iffbydefn it contains no free variables.
\end{definition}

\end{document}
