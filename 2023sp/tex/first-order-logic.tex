% define new conditionals: \if<conditional>
\newif\ifbook  % whether or not to use book styles
\newif\ifdate  % whether or not to include the current date header
\newif\ifalgorithms  % whether or not to define the algorithm environment
\newif\ifindentproofs  % whether or not to hang-indent proof environments
\newif\ifindenttheorems  % whether or not to hang-indent theorem environments

% set the conditionals: \<conditional>true / \<conditional>false
\bookfalse
\datefalse
\algorithmsfalse
\indentproofstrue
\indenttheoremstrue

\ifbook
    \documentclass[letterpaper]{book}
    \usepackage{arydshln, chngcntr}
\else
    \documentclass[letterpaper]{article}
\fi

% math
\usepackage{amsmath, amsfonts, amssymb, amstext, amscd, amsthm, mathrsfs, mathtools, xfrac}
% fonts
\usepackage{bbm, CJKutf8, caption, dsfont, marvosym, stmaryrd}
% tables
\usepackage{booktabs, colortbl, makecell}
% colors
\usepackage{color, soul, xcolor}
% references
\usepackage{hyperref, xr-hyper, url}
% figures
\usepackage{graphicx, float, tikz}
% headers and footers
\usepackage{fancyhdr, lastpage}
% miscellaneous
\usepackage{enumerate, ifthen, lipsum, listings, makeidx, parskip, verbatim, xargs}
\usepackage[nodayofweek]{datetime}

\usepackage[left=2cm,top=2cm,right=2cm,bottom=2cm,bindingoffset=0cm]{geometry}
\usepackage[group-separator={,},group-minimum-digits={3}]{siunitx}
\usepackage[shortlabels]{enumitem}
\usepackage[math]{cellspace}
\cellspacetoplimit 1pt
\cellspacebottomlimit 1pt

\definecolor{gruvred}{HTML}{CC214D}
\definecolor{gruvorange}{HTML}{D65D0E}
\definecolor{gruvaqua}{HTML}{689D6A}
\definecolor{gruvpurple}{HTML}{B16286}

\hypersetup{
    colorlinks=true,
    linkcolor=gruvorange,
    citecolor=gruvaqua,
    urlcolor=gruvpurple
}

\allowdisplaybreaks
\newdateformat{verbosedate}{\ordinal{DAY} of \monthname[\THEMONTH], \THEYEAR}
\verbosedate

\pagestyle{fancy}
% \fancyfoot[C]{--~\thepage~--}
\fancyfoot[C]{\tiny \thepage\ / \pageref*{LastPage}}
\ifbook
    \fancypagestyle{plain}{%
        \fancyhead[L]{}
        \ifdate
            \fancyhead[R]{\textsc{\today}}
        \else
            \fancyhead[R]{}
        \fi
        \renewcommand{\headrulewidth}{0pt}
    }

    \let\cleardoublepage=\clearpage
\else
    \fancypagestyle{plain}{}
    \renewcommand{\headrulewidth}{0pt}
\fi

\delimitershortfall=-1pt

\newlist{detail}{itemize}{2}
\setlist[detail]{label={\boldmath$\cdot$},topsep=0pt,leftmargin=*,noitemsep}

\ifalgorithms
    \newcounter{nalg}[chapter]
    \renewcommand{\thenalg}{\thechapter.\arabic{nalg}}
    \DeclareCaptionLabelFormat{algocaption}{\it Algorithm \thenalg}

    \lstnewenvironment{algorithm}[1][]
    {
        \refstepcounter{nalg}
        \captionsetup{labelformat=algocaption,labelsep=colon}
        \lstset{
            mathescape=true,
            frame=tB,
            numbers=left,
            numberstyle=\tiny,
            basicstyle=\scriptsize,
            keywordstyle=\color{black}\bfseries\em,
            keywords={,input, output, return, datatype, function, in, if, else, elif, for, foreach, while, not, begin, end, true, false, null, break, continue, let, and, or, }
            numbers=left,
            xleftmargin=.04\textwidth,
            #1
        }
    }
    {}
\else
\fi

% new proof environment
\expandafter\let\expandafter\oldproof\csname\string\proof\endcsname
\let\oldendproof\endproof
\renewenvironment{proof}[1][\proofname]{%
    \vspace{-0.5\parskip}%
    \oldproof[#1]
}{%
    ~\\\qed
}

\newtheoremstyle{cedretheorem} % name
    {1ex}  % space above
    {-1ex}  % space below
    {\itshape} % body font
    {}  % indent amount
    {\bfseries}  % theorem head font
    {.\\}  % punctuation after theorem head
    {.5em}  % space after theorem head
    {}  % theorem head spec (can be left empty, meaning ‘normal’)

\newtheoremstyle{cedredefinition} % name
    {1ex}  % space above
    {-1ex}  % space below
    {} % body font
    {}  % indent amount
    {\bfseries}  % theorem head font
    {.\\}  % punctuation after theorem head
    {.5em}  % space after theorem head
    {}  % theorem head spec (can be left empty, meaning ‘normal’)

\newtheoremstyle{cedrenote} % name
    {}  % space above
    {}  % space below
    {} % body font
    {}  % indent amount
    {\bfseries}  % theorem head font
    {.}  % punctuation after theorem head
    {.5em}  % space after theorem head
    {}  % theorem head spec (can be left empty, meaning ‘normal’)

% style for theorems
\theoremstyle{cedretheorem}
\newtheorem{theorem}{Theorem}
\newtheorem{lemma}{Lemma}[theorem]
\newtheorem{proposition}{Proposition}[theorem]
\newtheorem{corollary}{Corollary}[theorem]
\newtheorem{conjecture}{Conjecture}[theorem]
\newtheorem*{claim}{Claim}
\newtheorem*{justification}{Justification}

% style for definitions
\theoremstyle{cedredefinition}
\newtheorem{axiom}{Axiom}
\newtheorem{definition}{Definition}
\newtheorem{notation}{Notation}[definition]
\newtheorem{exercise}{Exercise}[definition]
\newtheorem{example}{Example}[definition]
\newtheorem*{counterexample}{Counterexample}

% style for notes
\theoremstyle{cedrenote}
\newtheorem{idea}{Idea}[definition]
\newtheorem*{remark}{Remark}
\newtheorem*{note}{Note}

\newenvironment{case}[1][Case]
    {\quote\textbf{#1:}~\\}
    {\endquote}

\def\lstlistingautorefname{Algorithm}
\def\itemautorefname{Section}
\renewcommand{\chapterautorefname}{Chapter}
\renewcommand{\sectionautorefname}{Section}
\renewcommand{\theoremautorefname}{Theorem}
\newcommand{\axiomautorefname}{Axiom}
\newcommand{\lemmaautorefname}{Lemma}
\newcommand{\propositionautorefname}{Proposition}
\newcommand{\corollaryautorefname}{Corollary}
\newcommand{\claimautorefname}{Claim}
\newcommand{\conjectureautorefname}{Conjecture}
\newcommand{\justificationautorefname}{Justification}
\newcommand{\definitionautorefname}{Definition}
\newcommand{\notationautorefname}{Notation}
\newcommand{\exampleautorefname}{Example}
\newcommand{\counterexampleautorefname}{Counterexample}
\newcommand{\ideaautorefname}{Idea}

\ifbook
    \renewcommand{\theequation}{\thechapter.\arabic{equation}}
    \renewcommand{\thetheorem}{\thechapter.\arabic{theorem}}
    \renewcommand{\thelemma}{\thechapter.\arabic{lemma}}
    \renewcommand{\theproposition}{\thechapter.\arabic{proposition}}
    \renewcommand{\thecorollary}{\thechapter.\arabic{corollary}}
    \renewcommand{\theconjecture}{\thechapter.\arabic{conjecture}}
    % \renewcommand{\theclaim}{\thechapter.\arabic{claim}}
    % \renewcommand{\thejustification}{\thechapter.\arabic{justification}}
    \renewcommand{\thedefinition}{\thechapter.\arabic{definition}}
    \renewcommand{\thenotation}{\thechapter.\arabic{notation}}
    \renewcommand{\theexample}{\thechapter.\arabic{example}}
    % \renewcommand{\thecounterexample}{\thechapter.\arabic{counterexample}}
    \counterwithin*{equation}{chapter}
    \counterwithin*{theorem}{chapter}
    \counterwithin*{lemma}{chapter}
    \counterwithin*{proposition}{chapter}
    \counterwithin*{corollary}{chapter}
    \counterwithin*{conjecture}{chapter}
    % \counterwithin*{claim}{chapter}
    % \counterwithin*{justification}{chapter}
    \counterwithin*{definition}{chapter}
    \counterwithin*{notation}{chapter}
    \counterwithin*{exercise}{chapter}
    \counterwithin*{example}{chapter}
    % \counterwithin*{counterexample}{chapter}
\else
\fi

\newcommand*{\xline}[1][3em]{\rule[0.5ex]{#1}{0.55pt}}

\newcommand{\isomorphic}{\cong}
\newcommand{\iffdefn}{\mathrel{\vcentcolon\Leftrightarrow}}
\newcommand{\iffbydefn}{$\iffdefn{}$}
\newcommand{\niff}{\mathrel{{\ooalign{\hidewidth$\not\phantom{"}$\hidewidth\cr$\iff$}}}}
\renewcommand{\implies}{~\Rightarrow~}
\renewcommand{\iff}{~\Leftrightarrow~}
\renewcommand{\restriction}[1]{\downharpoonright_{#1}}
\renewcommand{\qedsymbol}{\sc q.e.d.}
\renewcommand{\leq}{\leqslant}
\renewcommand{\geq}{\geqslant}

\newcommand{\meet}{\wedge}
\newcommand{\join}{\vee}
\newcommand{\conjunct}{\wedge}
\newcommand{\disjunct}{\vee}
\newcommand{\defn}{\coloneqq}
\newcommand{\xor}{\oplus}
\newcommand{\nand}{\uparrow}
\newcommand{\nor}{\downarrow}

\newcommand{\compose}{\circ}
\newcommand{\divides}{~|~}
\newcommand{\notdivides}{\not|~}
\newcommand{\given}{~\middle|~}
\newcommand{\suchthat}{~\middle|~}
\newcommand{\contradiction}{~\smash{\text{\Large \Lightning}}~}

\newcommand{\conjugate}[1]{\overline{#1}}
\newcommand{\mean}[1]{\overline{#1}}

\newcommand*\diff{\mathop{}\!\mathrm{d}}
\newcommand{\integral}[1]{\smashoperator{\int_{#1}}}
\newcommand{\E}[1]{\mathbb{E}\crochets*{#1}}
\newcommand{\Esub}[2]{\mathbb{E}_{#1}\crochets*{#2}}
\newcommand{\var}[1]{\mathrm{Var}\parens*{#1}}
\newcommand{\cov}[2]{\mathrm{Cov}\parens*{#1, #2}}
\newcommand{\der}[2]{\frac{\diff{#1}}{\diff{#2}}}
\newcommand{\dern}[3]{\frac{\diff^{#3}{#1}}{\diff{#2}^{#3}}}
\newcommand{\derm}[3]{\frac{\diff^{#3}{#1}}{\diff{#2}}}
\newcommand{\prt}[2]{\frac{\partial{#1}}{\partial{#2}}}
\newcommand{\prtn}[3]{\frac{\partial^{#3}{#1}}{\partial{#2}^{#3}}}
\newcommand{\prtm}[3]{\frac{\partial^{#3}{#1}}{\partial{#2}}}
\newcommand{\modulo}[1]{~\parens{\mathrm{mod}~#1}}

\newcommand{\inj}{\hookrightarrow}
\newcommand{\injection}{\hookrightarrow}

\newcommand{\surj}{\twoheadrightarrow}
\newcommand{\surjection}{\twoheadrightarrow}

\newcommand{\bij}{\lhook\joinrel\twoheadrightarrow}
\newcommand{\bijection}{\lhook\joinrel\twoheadrightarrow}

\newcommand{\monic}{\hookrightarrow}
\newcommand{\monomorphism}{\hookrightarrow}

\newcommand{\epic}{\twoheadrightarrow}
\newcommand{\epimorphism}{\twoheadrightarrow}

\newcommand{\iso}{\lhook\joinrel\twoheadrightarrow}
\newcommand{\isomorphism}{\lhook\joinrel\twoheadrightarrow}
\newcommand{\immersion}{\looprightarrow}

\renewcommand{\O}[1]{\mathcal{O}\parens*{#1}}
\renewcommand{\P}[1]{\mathcal{P}\parens*{#1}}
\newcommand{\C}{\mathbb{C}}
\newcommand{\N}{\mathbb{N}}
\newcommand{\Q}{\mathbb{Q}}
\newcommand{\R}{\mathbb{R}}
\newcommand{\Z}{\mathbb{Z}}

\newcommand{\century}{c.\ }
\newcommand{\ca}{\textit{ca.}\ }
\newcommand{\cf}{\textit{c.f.},\ }
\newcommand{\eg}{\textit{e.g.},\ }
\newcommand{\ie}{\textit{i.e.},\ }
\newcommand{\aka}{\textit{a.k.a.}\ }
\newcommand{\viz}{\textit{viz.}\ }
\newcommand{\vide}{\textit{v.}\ }
\newcommand{\etal}{\textit{et al.}\ }

\DeclareMathOperator{\lcm}{lcm}
\DeclareMathOperator*{\argmin}{arg\!\min}
\DeclareMathOperator*{\argmax}{arg\!\max}

\let\originalleft\left
\let\originalright\right
\renewcommand{\left}{\mathopen{}\mathclose\bgroup\originalleft}
\renewcommand{\right}{\aftergroup\egroup\originalright}

\newcommand{\zh}[1]{\begin{CJK}{UTF8}{gbsn}#1\end{CJK}}
\newcommand{\jp}[1]{\begin{CJK}{UTF8}{gbsn}#1\end{CJK}}

\DeclarePairedDelimiterX \inner[2]{\langle}{\rangle}{#1,#2}
\DeclarePairedDelimiter \bra{\langle}{\rvert}
\DeclarePairedDelimiter \ket{\lvert}{\rangle}
\DeclarePairedDelimiter \abs{\lvert}{\rvert}
\DeclarePairedDelimiter \cardinality{\lvert}{\rvert}
\DeclarePairedDelimiter \norm{\lVert}{\rVert}
\DeclarePairedDelimiter \set{\lbrace}{\rbrace}
\DeclarePairedDelimiter \seq{\langle}{\rangle}
\DeclarePairedDelimiter \parens{(}{)}
\DeclarePairedDelimiter \crochets{[}{]}
\DeclarePairedDelimiter \brackets{\langle}{\rangle}

\let\oldemptyset\emptyset
\let\emptyset\varnothing
\let\union\cup
\let\intersection\cap
\let\intersect\cap


\externaldocument{../propositional-logic/propositional-logic}

\begin{document}

\title{Discrete Mathematics}
\author{Daniel Gonzalez Cedre}
\date{University of Notre Dame \\ Spring of 2023}
\maketitle

\datetrue

\setcounter{chapter}{1}
\chapter{First-Order Logic}\label{chap:firstorder}

\section{Predicates \& Quantification}

\begin{definition}[Universe of Discourse]
    Our \emph{universe of discourse}, usually denoted by the letter \(\Omega\),
    denotes the collection of objects under consideration.
    This specifies the objects that are admissible as inputs to predicates,
    meaning these are the objects we can actually make concrete, descriptive claims about
    using our formal language.
\end{definition}

\begin{definition}[Predicate]
    Given a universe of discourse \(\Omega\) and a non-negative integer \(n\),
    we say that \(\varphi(x_1, \dots x_n)\) is an \(n\)-ary \emph{predicate}
    \iffbydefn the substitution of an object \(\omega_i\) from \(\Omega\) for each \(x_i\) in \(\varphi\)
    results in a proposition \(\varphi(\omega_1, \dots \omega_n)\).
    These are sometimes referred to as \emph{propositional functions}.
    Notice that if \(n = 0\) then the \(0\)-ary predicate is simply a proposition.

    The placeholders \(x_1, \dots x_n\) in \(\varphi\) are called \emph{variables},
    and the process of assigning objects to these variables is called
    \emph{instantiation} or \emph{variable assignment}.
\end{definition}
\begin{example}\label{ex:predicate}
    If we choose a universe of discourse \(\Omega\) consisting of some collection of people, then

    \begin{minipage}{.45\linewidth}
        \vspace{-\parskip-\abovedisplayskip}
        \begin{align*}
            \mu(x) &\defn \text{``}x \text{ is a mathematician.''}\\
            \varepsilon(x) &\defn \text{``}x \text{ loves espresso.''}
        \end{align*}
    \end{minipage}%
    \begin{minipage}{.45\linewidth}
        \vspace{-\parskip-\abovedisplayskip}
        \begin{align*}
            \gamma(x, y) &\defn \text{``}x \text{ drinks a Guinness with } y\text{.''}\\
            \alpha(x, y, z) &\defn \text{``}x\text{, }y\text{, and } z \text{ are colleagues.''}
        \end{align*}
    \end{minipage}

    are unary, binary, and ternary predicates respectively.
\end{example}

\begin{definition}[Universal Quantifier]
    Given a unary predicate \(\varphi(x)\),
    the \emph{universal quantification} of the variable \(x\) in \(\varphi\)
    is denoted by \(\forall x \pn*{\varphi(x)}\)
    and expresses that \(\varphi(\omega)\) is true for any arbitrary \(\omega\) from our universe of discourse.
\end{definition}
\begin{example}
    Using the definitions from \autoref{ex:predicate},
    we can rewrite ``Every mathematician loves espresso'' as
    \(\forall x \pn*{\varepsilon(x)}\) or as \(\forall x \pn*{\mu(x) \rightarrow \varepsilon(x)}\)
    depending on whether our universe \(\Omega\)
    is the collection of all mathematicians or the collection of all people.
\end{example}

\begin{definition}[Existential Quantifier]
    Given a unary predicate \(\varphi(x)\),
    the \emph{existential quantification} of the variable \(x\) in \(\varphi\)
    is denoted by \(\exists x \pn*{\varphi(x)}\)
    and expresses that \(\varphi(\omega)\) is true for at least one \(\omega\) from our universe of discourse.
\end{definition}
\begin{example}
    Using the definitions from \autoref{ex:predicate},
    we can rewrite ``Some mathematician loves espresso'' as
    \(\exists x \pn*{\varepsilon(x)}\) or as \(\exists x \pn*{\mu(x) \meet \varepsilon(x)}\)
    depending on whether our universe \(\Omega\)
    is the collection of all mathematicians or the collection of all people.
\end{example}

\section{Rules of Inference}

\begin{definition}[Rule of Inference]
    A \emph{rule of inference} is a construction in the meta-language that tells us
    how we're allowed to take previous statements in our formal language and derive new statements from them.
    They essentially describe the allowable algebraic manipulations we can make to the symbols in our language,
    and in this way they function analogously to the instruction set of a computer
    (if you're familiar with that).

    Another way to think of this is that they are truth-preserving axioms for our formal system;
    they are fundamental assumptions about the semantics of our formal language
    that allow us to reinterpret the statements we formulate in it.
    They usually take one of two forms: equivalences and inferences.
    For each of these rules, there is an \emph{underlying tautology},
    which is a tautological sentence in our formal language expressing the same idea as the rule.

    An \emph{equivalence rule} would take the form \(\varphi \iff \psi\),
    where \(\varphi\) and \(\psi\) are sentences,
    indicating that any instance of \(\varphi\) can be replaced by an instance of \(\psi\).
    The underlying tautology of such a rule is \(p \leftrightarrow q\).
    \\
    A popular alternative notation is \(\varphi \equiv \psi\).

    An \emph{inference rule} would take the form \(\Gamma \implies \psi\)
    % or, equivalently, \(\bigmeet_{i = 1}^{n}\varphi_i \implies \psi\),
    where \(\Gamma\) is a collection of sentences
    \(\varphi_1, \dots \varphi_n\) in our formal language
    and \(\psi\) is some other sentence,
    indicating that whenever we write down all of the sentences in \(\Gamma\),
    we can then write down \(\psi\) afterward.
    The underlying tautology of such a rule is
    \(\varphi_1 \meet \dots \varphi_n \implies \psi\).
    \\
    A popular alternative notation is \(\Gamma \proves \psi\), or \(\varphi_1, \dots \varphi_n \proves \psi\),
    indicating that we can deduce \(\psi\) from the sentences in \(\Gamma\).
\end{definition}

The axioms for a Boolean algebra given in
\autoref{def:boolean} in \autoref{chap:propositional}
are examples of \emph{equivalence rules of inference},
and we will take these as part of our collection of rules of inference for the formal language we are building.

\begin{table}[H]
    \centering
    \label{tab:rulesequivalence}
    \begin{tabular}{|Cc|Cc|Cc|} \hline
        \multicolumn{3}{|c|}{\thead{Equivalence Rules}} \\\hline
        \thead{Identity} & \thead{\(p \meet \top \iff p\) \\ \(p \join \bot \iff p\)} & \thead{\(p \meet \top \equiv p\) \\ \(p \join \bot \equiv p\)} \\ \hline
        \thead{Complement \\ (\aka Negation)} & \thead{\(p \meet \neg p \iff \bot\) \\ \(p \join \neg p \iff \top\)} & \thead{\(p \meet \neg p \equiv \bot\) \\ \(p \join \neg p \equiv \top\)} \\ \hline
        \thead{Commutativity} & \thead{\(p \meet q \iff q \meet p\) \\ \(p \join q \iff q \join p\)} & \thead{\(p \meet q \equiv q \meet p\) \\ \(p \join q \equiv q \join p\)} \\ \hline
        \thead{Associativity} & \thead{\(p \meet (q \meet r) \iff (p \meet q) \meet r\) \\ \(p \join (q \join r) \iff (p \join q) \join r\)} & \thead{\(p \meet (q \meet r) \equiv (p \meet q) \meet r\) \\ \(p \join (q \join r) \equiv (p \join q) \join r\)} \\ \hline
        \thead{Distributive Laws} & \thead{\(p \conjunct (q \disjunct r) \iff (p \conjunct q) \disjunct (p \conjunct r)\) \\ \(p \disjunct (q \conjunct r) \iff (p \disjunct q) \conjunct (p \disjunct r)\)} & \thead{\(p \conjunct (q \disjunct r) \equiv (p \conjunct q) \disjunct (p \conjunct r)\) \\ \(p \disjunct (q \conjunct r) \equiv (p \disjunct q) \conjunct (p \disjunct r)\)} \\ \hline
    \end{tabular}
\end{table}
\begin{table}[H]
    \centering
    \label{tab:rulesinference}
    \begin{tabular}{|Cc|Cc|Cc|} \hline
        \multicolumn{3}{|c|}{\thead{Inference Rules}} \\\hline
        \thead{Deduction} & \thead{If by assuming \(p\) we are able to derive \(q\),\\ then we can derive \(p \rightarrow q\)} & \thead[l]{\(p \proves q\) \\[1ex] \hline \\[-2ex] \(p \rightarrow q\)} \\ \hline
        \thead{Modus Ponens} & \thead{If we know that \(p \rightarrow q\) and we have \(p\),\\ then we can derive \(q\)} & \thead[l]{\(p\) \\ \(p \rightarrow q\) \\[1ex] \hline \\[-2ex] \(q\)} \\ \hline
        \thead{Modus Tollens} & \thead{If we know that \(p \rightarrow q\) and we have \(\neg q\),\\ then we can derive \(\neg p\)} & \thead[l]{\(\neg q\) \\ \(p \rightarrow q\) \\[1ex] \hline \\[-2ex] \(\neg p\)} \\ \hline
        \thead{Reductio ad Absurdum} & \thead{If by assuming \(p\) we can derive both \(q\) and \(\neg q\),\\ then we can derive \(\neg p\)} & \thead[l]{\(p \proves q\) \\ \(p \proves \neg q\) \\[1ex] \hline \\[-2ex] \(\neg p\)} \\ \hline
    \end{tabular}
\end{table}

% \begin{definition}[Modus Ponens]
%     If we have written down two sentences of the form \(\varphi\) and \(\varphi \rightarrow \psi\),
%     then we can write down \(\psi\).
% 
%     This can also be 
% \end{definition}
% 
% \begin{definition}[Deduction]
% \end{definition}

\end{document}
