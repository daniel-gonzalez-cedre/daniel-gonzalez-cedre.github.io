\input{config}

\title{Problem Set 8}
\author[Daniel Gonzalez Cedre]{Discrete Mathematics}
\publisher{University of Notre Dame}
\date{Due on the \red{4\textsuperscript{th} of April, 2024}}

\begin{document}

\maketitle

\begin{enumerate}
  \item[(10 pts) \quad 1.]
    There are $31$ people registered for our class this semester.
    Prove that two of those people were born on the same day of the month.

  \item[(10 pts) \quad 2.]
    As of the 28\textsuperscript{th} of March, 2024, there are over 8.1 billion people living on Earth.%
    \sidenote{For simplicity, you may assume the human population will only monotonically increase over time from the point this problem set was assigned.}
    A person's heart will beat no more than $7 \times 10^9$ times over their lifespan.
    Show that there are two currently-living people on Earth whose hearts have beat the exact same number of times. 

  \item[(20 pts) \quad 3.]
    Let $n \in \naturals$ and consider $\mathcal{A} \subseteq \naturals$ such that $\cardinality{\mathcal{A}} = n + 1$.
    Prove there exist $x, y \in \mathcal{A}$ with $x \neq y$ such that $n \divides x - y$.

  \item[(20 pts) \quad 4.]
    Consider $\mathcal{S} \defeq \set{3, 4, 7, 8, 9, 10, 12, 15, 18, 19, 27, 28}$ and $\mathcal{X} \subseteq \mathcal{S}$ with $\cardinality{\mathcal{X}} \geq 9$.
    Show that there exist three \emph{distinct} elements $x_1, x_2, x_3 \in \mathcal{X}$ such that $x_1 + x_2 + x_3 = 35$.
    % Consider $\mathcal{S} \defeq \set{1, 3, 4, 6, 7, 8, 9, 10, 11, 13, 14 16, 17, 26, 30}$ and let $\mathcal{X} \subseteq \mathcal{S}$ such that $\cardinality{\mathcal{X}} \geq 9$.
    % Show there exist three \emph{distinct} elements $x_1, x_2, x_3 \in \mathcal{X}$ such that $x_1 + x_2 + x_3 = 35$.
    % Consider $\mathcal{S} \defeq \set{1, 5, 6, 9, 11, 14, 16, 19, 21, 24, 29}$ and let $\mathcal{X} \subseteq \mathcal{S}$ such that $\cardinality{\mathcal{X}} \geq 6$.
    % Show there exist three \emph{distinct} elements $x_1, x_2, x_3 \in \mathcal{X}$ such that $x_1 + x_2 + x_3 = 35$.

  \item[(20 pts) \quad 5.]
    Recall that $\binom{n}{0} = \binom{n}{n} = 1$ for all $n, k \in \naturals$ when $k \leq n$.
    % Let $n, k \in \naturals$ with $k \leq n$ and recall that $\binom{n}{}$
    \begin{enumerate}
      \item
        Show $\binom{n}{k} = \binom{n}{n - k}$ for all $n, k \in \naturals$ where $k \leq n$.
        % Show $\displaystyle\binom{n}{k} = \binom{n}{n - k}$ for all $n, k \in \naturals$ where $k \leq n$.

      \item
        Show $\binom{n + 1}{k + 1} = \binom{n}{k + 1} + \binom{n}{k}$ for all $n, k \in \naturals$ where $k \leq n$.
        % Show $\displaystyle\binom{n + 1}{k + 1} = \binom{n}{k + 1} + \binom{n}{k}$ for all $n, k \in \naturals$ where $k \leq n$.

    \end{enumerate}

  \item[(20 pts) \quad 6.]
    Prove that $\cardinality{\power{X}} = 2^{\cardinality{X}}$ for any finite set $X$.

\end{enumerate}

\end{document}
