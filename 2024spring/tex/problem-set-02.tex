\input{config}
% define new conditionals: \if<conditional>
\newif\ifbook  % whether or not to use book styles
\newif\ifdate  % whether or not to include the current date header
\newif\ifalgorithms  % whether or not to define the algorithm environment
\newif\ifindentproofs  % whether or not to hang-indent proof environments
\newif\ifindenttheorems  % whether or not to hang-indent theorem environments

% set the conditionals: \<conditional>true / \<conditional>false
\bookfalse
\datefalse
\algorithmsfalse
\indentproofstrue
\indenttheoremstrue

\ifbook
    \documentclass[letterpaper]{book}
    \usepackage{arydshln, chngcntr}
\else
    \documentclass[letterpaper]{article}
\fi

% math
\usepackage{amsmath, amsfonts, amssymb, amstext, amscd, amsthm, mathrsfs, mathtools, xfrac}
% fonts
\usepackage{bbm, CJKutf8, caption, dsfont, marvosym, stmaryrd}
% tables
\usepackage{booktabs, colortbl, makecell}
% colors
\usepackage{color, soul, xcolor}
% references
\usepackage{hyperref, xr-hyper, url}
% figures
\usepackage{graphicx, float, tikz}
% headers and footers
\usepackage{fancyhdr, lastpage}
% miscellaneous
\usepackage{enumerate, ifthen, lipsum, listings, makeidx, parskip, verbatim, xargs}
\usepackage[nodayofweek]{datetime}

\usepackage[left=2cm,top=2cm,right=2cm,bottom=2cm,bindingoffset=0cm]{geometry}
\usepackage[group-separator={,},group-minimum-digits={3}]{siunitx}
\usepackage[shortlabels]{enumitem}
\usepackage[math]{cellspace}
\cellspacetoplimit 1pt
\cellspacebottomlimit 1pt

\definecolor{gruvred}{HTML}{CC214D}
\definecolor{gruvorange}{HTML}{D65D0E}
\definecolor{gruvaqua}{HTML}{689D6A}
\definecolor{gruvpurple}{HTML}{B16286}

\hypersetup{
    colorlinks=true,
    linkcolor=gruvorange,
    citecolor=gruvaqua,
    urlcolor=gruvpurple
}

\allowdisplaybreaks
\newdateformat{verbosedate}{\ordinal{DAY} of \monthname[\THEMONTH], \THEYEAR}
\verbosedate

\pagestyle{fancy}
% \fancyfoot[C]{--~\thepage~--}
\fancyfoot[C]{\tiny \thepage\ / \pageref*{LastPage}}
\ifbook
    \fancypagestyle{plain}{%
        \fancyhead[L]{}
        \ifdate
            \fancyhead[R]{\textsc{\today}}
        \else
            \fancyhead[R]{}
        \fi
        \renewcommand{\headrulewidth}{0pt}
    }

    \let\cleardoublepage=\clearpage
\else
    \fancypagestyle{plain}{}
    \renewcommand{\headrulewidth}{0pt}
\fi

\delimitershortfall=-1pt

\newcommand*{\xline}[1][3em]{\rule[0.5ex]{#1}{0.55pt}}

\newcommand{\isomorphic}{\cong}
\newcommand{\iffdefn}{\mathrel{\vcentcolon\Leftrightarrow}}
\newcommand{\iffbydefn}{$\iffdefn{}$}
\newcommand{\niff}{\mathrel{{\ooalign{\hidewidth$\not\phantom{"}$\hidewidth\cr$\iff$}}}}
\renewcommand{\implies}{~\Rightarrow~}
\renewcommand{\iff}{~\Leftrightarrow~}
\renewcommand{\restriction}[1]{\downharpoonright_{#1}}
\renewcommand{\qedsymbol}{\sc q.e.d.}
\renewcommand{\leq}{\leqslant}
\renewcommand{\geq}{\geqslant}

\newcommand{\meet}{\wedge}
\newcommand{\join}{\vee}
\newcommand{\conjunct}{\wedge}
\newcommand{\disjunct}{\vee}
\newcommand{\defn}{\coloneqq}
\newcommand{\xor}{\oplus}
\newcommand{\nand}{\uparrow}
\newcommand{\nor}{\downarrow}

\newcommand{\compose}{\circ}
\newcommand{\divides}{~|~}
\newcommand{\notdivides}{\not|~}
\newcommand{\given}{~\middle|~}
\newcommand{\suchthat}{~\middle|~}
\newcommand{\contradiction}{~\smash{\text{\Large \Lightning}}~}

\newcommand{\conjugate}[1]{\overline{#1}}
\newcommand{\mean}[1]{\overline{#1}}

\newcommand*\diff{\mathop{}\!\mathrm{d}}
\newcommand{\integral}[1]{\smashoperator{\int_{#1}}}
\newcommand{\E}[1]{\mathbb{E}\crochets*{#1}}
\newcommand{\Esub}[2]{\mathbb{E}_{#1}\crochets*{#2}}
\newcommand{\var}[1]{\mathrm{Var}\parens*{#1}}
\newcommand{\cov}[2]{\mathrm{Cov}\parens*{#1, #2}}
\newcommand{\der}[2]{\frac{\diff{#1}}{\diff{#2}}}
\newcommand{\dern}[3]{\frac{\diff^{#3}{#1}}{\diff{#2}^{#3}}}
\newcommand{\derm}[3]{\frac{\diff^{#3}{#1}}{\diff{#2}}}
\newcommand{\prt}[2]{\frac{\partial{#1}}{\partial{#2}}}
\newcommand{\prtn}[3]{\frac{\partial^{#3}{#1}}{\partial{#2}^{#3}}}
\newcommand{\prtm}[3]{\frac{\partial^{#3}{#1}}{\partial{#2}}}
\newcommand{\modulo}[1]{~\parens{\mathrm{mod}~#1}}

\newcommand{\inj}{\hookrightarrow}
\newcommand{\injection}{\hookrightarrow}

\newcommand{\surj}{\twoheadrightarrow}
\newcommand{\surjection}{\twoheadrightarrow}

\newcommand{\bij}{\lhook\joinrel\twoheadrightarrow}
\newcommand{\bijection}{\lhook\joinrel\twoheadrightarrow}

\newcommand{\monic}{\hookrightarrow}
\newcommand{\monomorphism}{\hookrightarrow}

\newcommand{\epic}{\twoheadrightarrow}
\newcommand{\epimorphism}{\twoheadrightarrow}

\newcommand{\iso}{\lhook\joinrel\twoheadrightarrow}
\newcommand{\isomorphism}{\lhook\joinrel\twoheadrightarrow}
\newcommand{\immersion}{\looprightarrow}

\renewcommand{\O}[1]{\mathcal{O}\parens*{#1}}
\renewcommand{\P}[1]{\mathcal{P}\parens*{#1}}
\newcommand{\C}{\mathbb{C}}
\newcommand{\N}{\mathbb{N}}
\newcommand{\Q}{\mathbb{Q}}
\newcommand{\R}{\mathbb{R}}
\newcommand{\Z}{\mathbb{Z}}

\newcommand{\century}{c.\ }
\newcommand{\ca}{\textit{ca.}\ }
\newcommand{\cf}{\textit{c.f.},\ }
\newcommand{\eg}{\textit{e.g.},\ }
\newcommand{\ie}{\textit{i.e.},\ }
\newcommand{\aka}{\textit{a.k.a.}\ }
\newcommand{\viz}{\textit{viz.}\ }
\newcommand{\vide}{\textit{v.}\ }
\newcommand{\etal}{\textit{et al.}\ }

\DeclareMathOperator{\lcm}{lcm}
\DeclareMathOperator*{\argmin}{arg\!\min}
\DeclareMathOperator*{\argmax}{arg\!\max}

\let\originalleft\left
\let\originalright\right
\renewcommand{\left}{\mathopen{}\mathclose\bgroup\originalleft}
\renewcommand{\right}{\aftergroup\egroup\originalright}

\newcommand{\zh}[1]{\begin{CJK}{UTF8}{gbsn}#1\end{CJK}}
\newcommand{\jp}[1]{\begin{CJK}{UTF8}{gbsn}#1\end{CJK}}

\DeclarePairedDelimiterX \inner[2]{\langle}{\rangle}{#1,#2}
\DeclarePairedDelimiter \bra{\langle}{\rvert}
\DeclarePairedDelimiter \ket{\lvert}{\rangle}
\DeclarePairedDelimiter \abs{\lvert}{\rvert}
\DeclarePairedDelimiter \cardinality{\lvert}{\rvert}
\DeclarePairedDelimiter \norm{\lVert}{\rVert}
\DeclarePairedDelimiter \set{\lbrace}{\rbrace}
\DeclarePairedDelimiter \seq{\langle}{\rangle}
\DeclarePairedDelimiter \parens{(}{)}
\DeclarePairedDelimiter \crochets{[}{]}
\DeclarePairedDelimiter \brackets{\langle}{\rangle}

\let\oldemptyset\emptyset
\let\emptyset\varnothing
\let\union\cup
\let\intersection\cap
\let\intersect\cap

\newlist{detail}{itemize}{2}
\setlist[detail]{label={\boldmath$\cdot$},topsep=0pt,leftmargin=*,noitemsep}

\ifalgorithms
    \newcounter{nalg}[chapter]
    \renewcommand{\thenalg}{\thechapter.\arabic{nalg}}
    \DeclareCaptionLabelFormat{algocaption}{\it Algorithm \thenalg}

    \lstnewenvironment{algorithm}[1][]
    {
        \refstepcounter{nalg}
        \captionsetup{labelformat=algocaption,labelsep=colon}
        \lstset{
            mathescape=true,
            frame=tB,
            numbers=left,
            numberstyle=\tiny,
            basicstyle=\scriptsize,
            keywordstyle=\color{black}\bfseries\em,
            keywords={,input, output, return, datatype, function, in, if, else, elif, for, foreach, while, not, begin, end, true, false, null, break, continue, let, and, or, }
            numbers=left,
            xleftmargin=.04\textwidth,
            #1
        }
    }
    {}
\else
\fi

% new proof environment
\expandafter\let\expandafter\oldproof\csname\string\proof\endcsname
\let\oldendproof\endproof
\renewenvironment{proof}[1][\proofname]{%
    \vspace{-0.5\parskip}%
    \oldproof[#1]
}{%
    ~\\\qed
}

\newtheoremstyle{cedretheorem} % name
    {1ex}  % space above
    {-1ex}  % space below
    {\itshape} % body font
    {}  % indent amount
    {\bfseries}  % theorem head font
    {.\\}  % punctuation after theorem head
    {.5em}  % space after theorem head
    {}  % theorem head spec (can be left empty, meaning ‘normal’)

\newtheoremstyle{cedredefinition} % name
    {1ex}  % space above
    {-1ex}  % space below
    {} % body font
    {}  % indent amount
    {\bfseries}  % theorem head font
    {.\\}  % punctuation after theorem head
    {.5em}  % space after theorem head
    {}  % theorem head spec (can be left empty, meaning ‘normal’)

\newtheoremstyle{cedrenote} % name
    {}  % space above
    {}  % space below
    {} % body font
    {}  % indent amount
    {\bfseries}  % theorem head font
    {.}  % punctuation after theorem head
    {.5em}  % space after theorem head
    {}  % theorem head spec (can be left empty, meaning ‘normal’)

% style for theorems
\theoremstyle{cedretheorem}
\newtheorem{theorem}{Theorem}
\newtheorem{lemma}{Lemma}[theorem]
\newtheorem{proposition}{Proposition}[theorem]
\newtheorem{corollary}{Corollary}[theorem]
\newtheorem{conjecture}{Conjecture}[theorem]
\newtheorem*{claim}{Claim}
\newtheorem*{justification}{Justification}

% style for definitions
\theoremstyle{cedredefinition}
\newtheorem{axiom}{Axiom}
\newtheorem{definition}{Definition}
\newtheorem{notation}{Notation}[definition]
\newtheorem{exercise}{Exercise}[definition]
\newtheorem{example}{Example}[definition]
\newtheorem*{counterexample}{Counterexample}

% style for notes
\theoremstyle{cedrenote}
\newtheorem{idea}{Idea}[definition]
\newtheorem*{remark}{Remark}
\newtheorem*{note}{Note}

\newenvironment{case}[1][Case]
    {\quote\textbf{#1:}~\\}
    {\endquote}

\def\lstlistingautorefname{Algorithm}
\def\itemautorefname{Section}
\renewcommand{\chapterautorefname}{Chapter}
\renewcommand{\sectionautorefname}{Section}
\renewcommand{\theoremautorefname}{Theorem}
\newcommand{\axiomautorefname}{Axiom}
\newcommand{\lemmaautorefname}{Lemma}
\newcommand{\propositionautorefname}{Proposition}
\newcommand{\corollaryautorefname}{Corollary}
\newcommand{\claimautorefname}{Claim}
\newcommand{\conjectureautorefname}{Conjecture}
\newcommand{\justificationautorefname}{Justification}
\newcommand{\definitionautorefname}{Definition}
\newcommand{\notationautorefname}{Notation}
\newcommand{\exampleautorefname}{Example}
\newcommand{\counterexampleautorefname}{Counterexample}
\newcommand{\ideaautorefname}{Idea}

\ifbook
    \renewcommand{\theequation}{\thechapter.\arabic{equation}}
    \renewcommand{\thetheorem}{\thechapter.\arabic{theorem}}
    \renewcommand{\thelemma}{\thechapter.\arabic{lemma}}
    \renewcommand{\theproposition}{\thechapter.\arabic{proposition}}
    \renewcommand{\thecorollary}{\thechapter.\arabic{corollary}}
    \renewcommand{\theconjecture}{\thechapter.\arabic{conjecture}}
    % \renewcommand{\theclaim}{\thechapter.\arabic{claim}}
    % \renewcommand{\thejustification}{\thechapter.\arabic{justification}}
    \renewcommand{\thedefinition}{\thechapter.\arabic{definition}}
    \renewcommand{\thenotation}{\thechapter.\arabic{notation}}
    \renewcommand{\theexample}{\thechapter.\arabic{example}}
    % \renewcommand{\thecounterexample}{\thechapter.\arabic{counterexample}}
    \counterwithin*{equation}{chapter}
    \counterwithin*{theorem}{chapter}
    \counterwithin*{lemma}{chapter}
    \counterwithin*{proposition}{chapter}
    \counterwithin*{corollary}{chapter}
    \counterwithin*{conjecture}{chapter}
    % \counterwithin*{claim}{chapter}
    % \counterwithin*{justification}{chapter}
    \counterwithin*{definition}{chapter}
    \counterwithin*{notation}{chapter}
    \counterwithin*{exercise}{chapter}
    \counterwithin*{example}{chapter}
    % \counterwithin*{counterexample}{chapter}
\else
\fi


\title{Problem Set 2}
\author[Daniel Gonzalez Cedre]{Discrete Mathematics}
\publisher{University of Notre Dame}
\date{Due on the \red{4\textsuperscript{th} of February, 2024}}

\begin{document}

\maketitle

% \section*{In this problem set, the variables $p$, $q$, and $r$ refer to propositions.}
% \section*{You may rely on the theorems listed in the margin.}
\marginnote{%
  You may rely on the following theorems throughout this problem set in addition to the axioms of classical logic.
  \begin{itemize}
    \item
      \emph{Uniqueness of Negations}
    \item
      $\neg \top \equiv \bot$ and $\neg \bot \equiv \top$
    \item
      \emph{Double Negation}
    \item
      \emph{Idempotency}
    \item
      \emph{Domination}
    \item
      \emph{De Morgan's Laws}
  \end{itemize}
}

We say that a propositional formula is a \defn{tautology} if it is logically equivalent to $\top$ under any assignment of truth values to its variables.

\begin{enumerate}
  % \item
      % For the following problems, do not appeal to truth tables.
      % \marginnote{These are called the Absorption Laws.}
      % \begin{enumerate}
          % \item
              % Show
              % $p \meet (p \join q) \equiv p$
              % for any $p$ and $q$.
          % \item
              % Show
              % $p \join (p \meet q) \equiv q$
              % for any $p$ and $q$.
      % \end{enumerate}
  \item[(5 pts) \quad 1.]
    Consider the following proof of $p \lif (q \lif r) \equiv (p \lif q) \lif r$.
    \begin{mdframed}
      \vspace{2ex}
      \begin{proof}
        Observe the following chain of reasoning.
        \begin{align*}
          p \lif (q \lif r)
            &\equiv p \lor \neg(q \lif r)
              &\quad
              &\text{by \emph{conditional disintegration}} \\
            &\equiv p \lor \neg(q \lor \neg r)
              &\quad
              &\text{by \emph{conditional disintegration}} \\
            &\equiv p \lor \neg q \lor \neg r
              &\quad
              &\text{by \emph{associativity}} \\
            % &\equiv (p \lor \neg q) \lor \neg r
              % &\quad
              % &\text{by \emph{associativity}} \\
            &\equiv (p \lor \neg q) \lor \neg r
              &\quad
              &\text{by \emph{associativity}} \\
            &\equiv (p \lif q) \lor \neg r
              &\quad
              &\text{by \emph{conditional disintegration}} \\
            &\equiv (p \lif q) \lif r
              &\quad
              &\text{by \emph{conditional disintegration}}
        \end{align*}
        Therefore, $p \lif (q \lif r) \equiv (p \lif q) \lif r$.
      \end{proof}
    \end{mdframed}
    Find all of the mistakes, if any, in this proof, and \emph{explain why.}
  \item[(40 pts) \quad 2.]
    Prove the claims below \emph{without truth tables} for all propositions $p, q, r$.
    \begin{enumerate}
      \item
        $p \rightarrow q \equiv \neg q \rightarrow \neg p$.
      % \item
        % $(p \rightarrow q) \meet (\neg p \rightarrow q) \equiv q$.
      % \item
        % $(p \rightarrow r) \join (q \rightarrow r) \equiv (p \meet q) \rightarrow r$.
      % \item
        % $(p \rightarrow q) \meet (p \rightarrow r) \equiv p \rightarrow (q \meet r)$.
      \item
        $(p \meet (p \rightarrow q)) \rightarrow q$ is a tautology.
      \item
        $(\neg q \meet (p \rightarrow q)) \rightarrow \neg p$ is a tautology.
      \item
        $(p \rightarrow q) \rightarrow ((p \rightarrow \neg q) \rightarrow \neg p)$ is a tautology.
    \end{enumerate}
  \item[(40 pts) \quad 3.]
    In this problem, we will progressively establish that the alternative axioms Hilbert proposed are all tautologies \emph{without truth tables.}.
    Here, the variables $p$, $q$, and $r$ all represent arbitrary propositions.
    \begin{enumerate}
      \item
        Show $p \lif p$ is a tautology.
      \item
        Show $(p \lif q) \lif (\neg q \lif \neg p)$ is a tautology.
      \item
        Show $p \lif (q \lif p)$ is a tautology.
      \item
        Show $(p \lif (q \lif r)) \lif ((p \lif q) \lif (p \lif r))$ is a tautology.
    \end{enumerate}
  % \item[(10 pts) \quad 4.]
  % \item[(20 pts)~~~~3.]
    % The variables $p$, $q$, $r$ refer to propositions
    % in the following problems.%
    % \marginnote{Make sure to read section 1.4 of the lecture notes on
                % logical nonequivalence.}
    % \begin{enumerate}
      % \item
        % Show that
        % $(p \rightarrow q) \rightarrow r$
        % is \emph{not} equivalent to
        % $p \rightarrow (q \rightarrow r)$.
      % \item
        % Show that
        % $\neg (p \rightarrow q)$
        % is \emph{not} equivalent to
        % $\neg p \rightarrow \neg q$.
    % \end{enumerate}
  % \item[(10 pts)~~~~3.]
    % Given%
    % \marginnote{Make sure to look at the \eorange{recursive} definition
                % of a proposition in the lecture notes.
                % Your proof should involve going through
                % the cases in that definition.}
    % an arbitrary proposition $p$,
    % show that there exists a proposition $q$
    % consisting only of $\neg$ and $\meet$
    % such that $p \equiv q$.
  \item[(10 pts)~~~~4.]
    Show that $\neg$ and $\land$ are sufficient to express \emph{any} proposition.
  \item[(5 pts)~~~~5.]
    Is there a \emph{single connective} capable of expressing \emph{any} proposition?%
    \sidenote{%
      This does not necessarily have to be one of $\neg$, $\meet$, $\join$, $\rightarrow$, nor $\leftrightarrow$.
      You can define new logical connectives using truth tables.
    }
    Justify your answer with a proof.
    % Is there a \emph{single} logical connective such that any propositional formula $\varphi$ can be rewritten equivalently using \emph{only} this connective?
    % %
    % such that, for any proposition $p$,
    % it is possible to find a proposition $q$ consisting of
    % \emph{only that} logical connective
    % with the property that $p \equiv q$?
    % It must be the \emph{same} logical connective across all propositions.\\
    % Justify your answer with a proof or argument.
\end{enumerate}

\end{document}
