\input{config}

\title{Problem Set 9}
\author[Daniel Gonzalez Cedre]{Discrete Mathematics}
\publisher{University of Notre Dame}
\date{Due on the \red{15\textsuperscript{th} of April, 2024}}

\usepackage{xskak}

\begin{document}

\maketitle

\section{As always, all answers must be justified with a proof.}

\begin{enumerate}
  \item[(10 pts) \quad 1.]
    A \defn{palindrome} of length $k \in \naturals$ over an alphabet $X$ is a string $s: k \to X$ such that $(\forall i \in k)(s(i) = s(k - 1 - i))$.
    % henlo
    % \sidenote{\url{https://youtu.be/zR0jWzzfqPE?si=oXosDA0K4gxN_hHX&t=312}}
    How many possible palindromic words of length $k \in \naturals$ are there in the English language?%
    \sidenote{For simplicity, you may assume words are the same regardless of capitalization.}

  \item[(20 pts) \quad 2.]
     Given a natural number $k \in \naturals$, how many ordered pairs $(a, b)$ are there such that $a, b \in \naturals$ and $a + b = k$?

  \item[(20 pts) \quad 3.]
    Let $n \in \naturals_+$ and suppose you have an $n \times n$ chess board.
    We say two pieces on the board \defn{threaten each other} if it is possible for one to capture the other by moving to occupy its square on the next move.
    \begin{marginfigure}
      \centering
      \newchessgame[setwhite={rc4}, addblack={rd5}]
      \chessboard[smallboard,
                  showmover=false,
                  color=blue,
                  pgfstyle=border,
                  markfield=c4,
                  colorbackfields={c1,c2,c3,c5,c6,c7,c8,
                                  a4,b4,d4,e4,f4,g4,h4}]
      \caption{%
        Two \chr{rooks} placed on an $8 \times 8$ chess board so that they do not threaten each other.
        The \high{movement pattern} for the white \chr{rook} is highlighted above.
      }\label{fig:rook}
    \end{marginfigure}
    % \begin{marginfigure}
      % \centering
      % \newchessgame[setwhite={be4}, addblack={bc3}]
      % \chessboard[smallboard,
                  % showmover=false,
                  % color=gold,
                  % pgfstyle=border,
                  % markfield=e4,
                  % colorbackfields={b1,c2,d3,f5,g6,h7,
                                  % a8,b7,c6,d5,f3,g2,h1}]
      % \caption{%
        % Two \chr{bishops} placed on an $8 \times 8$ chess board so that they do not threaten each other.
        % The \high{movement pattern} for the white \chr{bishop} is highlighted above.
      % }\label{fig:bishop}
    % \end{marginfigure}
    % \begin{marginfigure}
      % \centering
      % \newchessgame[setwhite={qc4}, addblack={qf6}]
      % \chessboard[smallboard,
                  % showmover=false,
                  % color=gold,
                  % pgfstyle=border,
                  % markfield=c4,
                  % colorbackfields={c1,c2,c3,c5,c6,c7,c8,
                                  % a4,b4,d4,e4,f4,g4,h4},
                  % colorbackfields={a2,b3,d5,e6,f7,g8,
                                  % a6,b5,d3,e2,f1}]
      % % \showboard
      % \caption{}\label{fig:queen}
    % \end{marginfigure}

    In how many ways can $n \in \naturals_+$ \chr{rooks} possibly be arranged on an $n \times n$ chess board so that no two \chr{rooks} threaten each other?

    % \begin{enumerate}
      % \item
        % % The \texttt{rook} in chess is a piece that only moves along the \emph{ranks} and \emph{files} of the board as shown in \cref{fig:rook}.
        % How many ways can $n \in \naturals_+$ \chr{rooks} possibly be arranged on an $n \times n$ chess board so that no two \chr{rooks} threaten each other?
      % \item
        % % The \texttt{bishop} in chess is a piece that only moves along the \emph{diagonals} of the board as shown in \cref{fig:bishop}.
        % \sout{How many ways can $n \in \naturals_+$ \chr{bishops} possibly be arranged on an $n \times n$ chess board so that no two \chr{bishops} threaten each other?}
    % \end{enumerate}

  \item[(20 pts) \quad 4.]
    We can write $4$ as a sum of positive integers in 8 distinct ways.
    \begin{equation*}
      \begin{tabular}{cccccccc}
                      &\quad& 1 + 1 + 2 &\quad& 1 + 3           \\
        1 + 1 + 1 + 1 &\quad& 1 + 2 + 1 &\quad& 2 + 2 &\quad& 4 \\
                      &\quad& 2 + 1 + 1 &\quad& 3 + 1           \\
      \end{tabular}
    \end{equation*}
    % \begin{equation*}
      % 4 =
      % \begin{cases}
        % & 1 + 1 + 1 + 1 \\
        % & 1 + 1 + 2 \\
        % & 1 + 2 + 1 \\
        % & 2 + 1 + 1 \\
        % & 1 + 3 \\
        % & 3 + 1 \\
        % & 2 + 2 \\
        % & 4
      % \end{cases}
    % \end{equation*}
    % \begin{align*}
      % 4 &= 1 + 1 + 1 + 1 \\
        % &= 1 + 1 + 2 \\
        % &= 1 + 2 + 1 \\
        % &= 2 + 1 + 1 \\
        % &= 1 + 3 \\
        % &= 3 + 1 \\
        % &= 2 + 2 \\
        % &= 4
      % % 4 &= 4 \\
        % % &= 3 + 1 \\
        % % &= 1 + 3 \\
        % % &= 2 + 2 \\
        % % &= 2 + 1 + 1 \\
        % % &= 1 + 2 + 1 \\
        % % &= 1 + 1 + 2 \\
        % % &= 1 + 1 + 1 + 1 \\
    % \end{align*}
    Given a natural number $n \in \naturals$, how many distinct ways are there to write $n$ as a sum of positive integers?

  \item[(30 pts) \quad 5.]
    A hermetic monk in meditation is repeatedly ascending and descending a ladder with $n \in \naturals$ rungs.
    The burdens of wisdom and devotion have left the monk's body frail, so he can only move by one or two rungs at-a-time.
    % Suddenly, a terrifying Word reverberates through his soul.
    One day, as the monk ascends the ladder, a sudden tempest blows through his soul, and out of the whirlwind a fathomless voice calls out to him.
    \begin{quote}
      \emph{``In how many ways can you do this?''}
    \end{quote}
    The monk ponders this question.
    After many years, he finds the answer, is enlightened, and immediately dies.
    What was the answer?
    % The monk ponders this question for many years until the answer eventually comes to him.
    % In this moment, the monk is enlightened and immediately dies.

    % \emph{Strong induction} is a variant of mathematical induction
    % with a \emph{stronger} inductive hypothesis.
    % Given a predicate $\varphi$ with the goal of proving $(\forall n \in \naturals)(\varphi(n))$, the base case remains the same.
    % In the inductive step, however, instead of making the \emph{``weak''} assumption $\varphi(k)$ for an arbitrary $k \in \naturals$, we make the \emph{``strong''} assumption $\varphi(\ell)$ \emph{for all} natural numbers $\ell$ \emph{less than or equal to} $k$.
    \emph{Hint:} It may be helpful to use \emph{strong induction} to prove your result; as a reminder, the scheme of strong induction is given below for an arbitrary predicate $\varphi$ of one free variable.
    % The \emph{schema of strong induction} is stated below.
      \begin{equation*}
        \left(\varphi(0) \land
          \left(\forall k \in \naturals\right)
          \left(\left(\forall \ell \in \naturals\right)
            \left(\ell \leq k \implies \varphi\left(\ell\right)\right)
        \implies \varphi(k + 1)\right)\right)
        \implies \left(\forall n \in \naturals\right)\left(\varphi(n)\right)
      \end{equation*}

\end{enumerate}

\end{document}
